% The next command tells RStudio to do "Compile PDF" on book.Rnw,
% instead of this chapter, thereby eliminating the need to switch back to book.Rnw 
% before making the book.
%!TEX root = HSB.Rnw

This section provides a detailed derivation of the comprehensive rebound framework,
beginning with some typical energy, efficiency, and cost relationships.


%++++++++++++++++++++++++++++++
\subsection{Energy and cost relationships}
\label{sec:energy_and_cost_relationships}
%++++++++++++++++++++++++++++++

With the definitions of Appendix~\ref{sec:nomenclature}, four typical relationships emerge.
First, the consumption rate of the energy service~($\rate{q}_s$)
is the product of final-to-services efficiency~($\eta$) and
the rate of consumption of final energy by the service-producing machine~($\rate{E}_s$).
Typical units for automotive transport are provided.
%
\begin{align} \label{eq:typ_qs_eta_Edot}
  \rate{q}_s &= \eta \rate{E}_s \\
  \text{pass-km/yr} &= [\text{pass-km/MJ}] [\text{MJ/yr}] \nonumber
\end{align}
%
Second, the energy service price~($p_s$) is the ratio of energy price~($p_E$) to the final-to-services efficiency~($\eta$).
%
\begin{align} \label{eq:ps_pE_eta}
  p_s &= \frac{p_E}{\eta} \\
  \text{\$/pass-km} &= \frac{\$/\text{MJ}}{\text{pass-km/MJ}} \nonumber
\end{align}
%
Third, cost rates~($\rate{C}$) are the product of energy price~($p_E$) and device energy consumption~($\rate{E}_s$).
%
\begin{align}
  \rate{C} &= p_E \rate{E}_s \\
  \text{\$/yr} &= [\text{\$/MJ}][\text{MJ/yr}] \nonumber
\end{align}
%
Fourth, energy rates are the product of cost rates~($\rate{C}$) and energy intensity of the economy~($I_E$).
%
\begin{align}
  \rate{E} &= \rate{C} I_E \\
  \text{MJ/yr} &= [\text{\$/yr}][\text{MJ/\$}] \nonumber
\end{align}


%++++++++++++++++++++++++++++++
\subsection{Rebound relationships}
\label{sec:rebound_relationships}
%++++++++++++++++++++++++++++++

Energy rebound is defined as
%
\begin{equation} \label{eq:Re_def}
  Re \equiv 1 - \frac{\text{actual final energy savings rate}}{\text{expected final energy savings rate}} \; ,
\end{equation}
%
where both actual and expected energy savings rates are expected positive.
Final energy ``takeback'' is the rate at which energy savings are eroded by rebound effects,
and the actual final energy savings rate is the expected energy savings rate less takeback.
Note that takeback can be negative, indicating in increase in the energy savings rate
relative to the expected energy savings rate.
Thus,
%
\begin{equation}
  Re = 1 - \frac{\text{expected energy savings rate} - \text{takeback}}{\text{expected energy savings rate}} \; ,
\end{equation}
%
or
%
\begin{equation}
  Re = 1 - 1 + \frac{\text{takeback}}{\text{expected energy savings rate}} \; .
\end{equation}
%
Simplifying gives
%
\begin{equation} \label{eq:Re_takeback}
  Re = \frac{\text{takeback}}{\text{expected energy savings rate}} \; .
\end{equation}
%
Thus, energy rebound can be given by either Eq.~\ref{eq:Re_def} or Eq.~\ref{eq:Re_takeback}.

We define rebound at the final energy level, 
because that is the point of energy purchase by the device owner.
To simplify derivations, 
we choose not to apply final-to-primary energy multipliers
to final energy rates in the numerators and denominators of rebound expressions
derived from Eqs.~\ref{eq:Re_def} and~\ref{eq:Re_takeback};
they divide out anyway.


%++++++++++++++++++++++++++++++
\subsection{Relationships for stages}
\label{sec:relationships_for_stages}
%++++++++++++++++++++++++++++++

For each energy rebound effect in Fig.~\ref{fig:flowchart},
energy and financial analysis must be performed.
The purposes of the analyses are to determine for each effect
%
\begin{enumerate*}[label={(\alph*)}]

  \item a definition for energy rebound~($Re$) and

  \item an equation for net income~($\rate{N}$).

\end{enumerate*}

Analysis of each stage involves a set of assumptions and constraints
as shown in Table~\ref{tab:analysis_assumptions}.
In Table~\ref{tab:analysis_assumptions}, 
relationships for \lceffect{} 
embodied energy rates, 
capital expenditure rates, and 
OM and disposal expenditure rates
are typical, and
inequalities could switch direction for a specific EEU.
\Prodeffect{} relationships are given for a single device only.
If the EEU is deployed at scale across the economy, 
the energy service consumption rate~($\rate{q}_s$), 
device energy consumption rate~($\rate{E}_s$), 
embodied energy rate~($\rate{E}_{emb}$),
capital expenditure rate~($\rate{C}_{cap}$), and 
OM and disposal expenditure rate~($\rate{C}_{\OMd}$)
will all increase.

% The next command tells RStudio to do "Compile PDF" on HSB.Rnw,
% instead of this file, thereby eliminating the need to switch back to HSB.Rnw 
% before building the paper.
%!TEX root = ../HSB.Rnw

\begin{landscape}

\begin{table}
\centering
\caption{Assumptions and constraints for analysis of rebound effects.}
\label{tab:analysis_assumptions}

\begin{tabular}{r c c c c c}
\toprule
% Parameter & \DevEffect{} & \LcEffect & \SubEffect & \IncEffect & \MacroEffect \\
Parameter & \EmplEffect{} & \SubEffect & \IncEffect & \MacroEffect \\
\midrule
Energy price                     & $\bempl{p}_E  = \aempl{p}_E$         
                                 & $\bsub{p}_E   = \asub{p}_E$ 
                                 & $\binc{p}_E   = \ainc{p}_E$ 
                                 & $\bprod{p}_E  = \aprod{p}_E$ \\
%
Energy service efficiency        & $\bempl{\eta}  < \aempl{\eta}$         
                                 & $\bsub{\eta}   = \asub{\eta}$ 
                                 & $\binc{\eta}   = \ainc{\eta}$ 
                                 & $\bprod{\eta}  = \aprod{\eta}$ \\
%
Energy service price             & $\bempl{p}_s  > \aempl{p}_s$          
                                 & $\bsub{p}_s   = \asub{p}_s$ 
                                 & $\binc{p}_s   = \ainc{p}_s$  
                                 & $\bprod{p}_s  = \aprod{p}_s$ \\
%
Other goods price                & $\bempl{p}_o  = \aempl{p}_o$          
                                 & $\bsub{p}_o   = \asub{p}_o$ 
                                 & $\binc{p}_o   = \ainc{p}_o$  
                                 & $\bprod{p}_o  = \aprod{p}_o$ \\
%
Energy service consumption rate  & $\rbempl{q}_s  = \raempl{q}_s$         
                                 & $\rbsub{q}_s   < \rasub{q}_s$ 
                                 & $\rbinc{q}_s   < \rainc{q}_s$ 
                                 & $\rbprod{q}_s  = \raprod{q}_s$ \\
%
Other goods consumption rate     & $\rbempl{q}_o  = \raempl{q}_o$         
                                 & $\rbsub{q}_o   > \rasub{q}_o$ 
                                 & $\rbinc{q}_o   < \rainc{q}_o$ 
                                 & $\rbprod{q}_o  = \raprod{q}_o$ \\
%
Device energy consumption rate   & $\rbempl{E}_s  > \raempl{E}_s$
                                 & $\rbsub{E}_s   < \rasub{E}_s$ 
                                 & $\rbinc{E}_s   < \rainc{E}_s$ 
                                 & $\rbprod{E}_s  = \raprod{E}_s$ \\
%
Embodied energy rate             & $\rbempl{E}_{emb}  < \raempl{E}_{emb}$ 
                                 & $\rbsub{E}_{emb}   = \rasub{E}_{emb}$ 
                                 & $\rbinc{E}_{emb}   = \rainc{E}_{emb}$ 
                                 & $\rbprod{E}_{emb}  = \raprod{E}_{emb}$ \\
%
Capital expenditure rate         & $\rbempl{C}_{cap}  < \raempl{C}_{cap}$ 
                                 & $\rbsub{C}_{cap}   = \rasub{C}_{cap}$ 
                                 & $\rbinc{C}_{cap}   = \rainc{C}_{cap}$ 
                                 & $\rbprod{C}_{cap}  = \raprod{C}_{cap}$ \\
%
Maint.\ and disp.\ expenditure rate & $\rbempl{C}_{\md}  < \raempl{C}_{\md}$ 
                                 & $\rbsub{C}_{\md}      = \rasub{C}_{\md}$ 
                                 & $\rbinc{C}_{\md}      = \rainc{C}_{\md}$ 
                                 & $\rbprod{C}_{\md}     = \raprod{C}_{\md}$ \\
%
Energy service expenditure rate  & $\rbempl{C}_s  > \raempl{C}_s$
                                 & $\rbsub{C}_s   < \rasub{C}_s$ 
                                 & $\rbinc{C}_s   < \rainc{C}_s$ 
                                 & $\rbprod{C}_s  = \raprod{C}_s$ \\
%
Other goods expenditure rate     & $\rbempl{C}_o  = \raempl{C}_o$         
                                 & $\rbsub{C}_o   > \rasub{C}_o$ 
                                 & $\rbinc{C}_o   < \rainc{C}_o$ 
                                 & $\rbprod{C}_o  = \raprod{C}_o$ \\
%
Income                           & $\rbempl{M} = \raempl{M}$         
                                 & $\rbsub{M}  = \rasub{M}$ 
                                 & $\rbinc{M}  = \rainc{M}$ 
                                 & $\rbprod{M} = \raprod{M}$  \\
%
Net income (freed cash)          & 0 = $\rbempl{N} <   \raempl{N}$         
                                 & $\rbsub{N}      \ne \rasub{N}$ 
                                 & $\rbinc{N}      >   \rainc{N} = 0$ 
                                 & $\rbprod{N}     =   \raprod{N} = 0$  \\
\bottomrule
\end{tabular}


\end{table}

\end{landscape}



%++++++++++++++++++++++++++++++
\subsection{Derivations}
\label{sec:derivations}
%++++++++++++++++++++++++++++++

Derivations for rebound definitions and net income equations
are presented in Tables~\ref{tab:deveffect}--\ref{tab:prodeffect},
one for each stage in Fig.~\ref{fig:flowchart}.
Energy and financial analyses are shown side by side, because
each informs the other.

% Derivation tables

% The next command tells RStudio to do "Compile PDF" on book.Rnw,
% instead of this chapter, thereby eliminating the need to switch back to book.Rnw 
% before making the book.
%!TEX root = HSB.Rnw

% This file contains the derivation for device effect rebound and net income.

\begin{landscape}

\linespread{1}

%%%%%%%%%%%%%%%%%%%%%%%%%%%%%%%%%%%
%%%%%%%%%% Device Effect %%%%%%%%%%
%%%%%%%%%%%%%%%%%%%%%%%%%%%%%%%%%%%
% \derivheader{\inlinebox[\devcolor]{\bf{Device Effect}}}
\derivheader{\bf{\DevEffect}}

\sectionsep{}

%%%%%%%%%% Before Device Effect %%%%%%%%%%
\derivsection{before}
{
% Original energy
\begin{equation} \label{eq:E_acct_orig}
  \Eacctorig{}
\end{equation}
}
{
% Original financial
\begin{equation} \label{eq:M_acct_orig}
  \Macctorig{}
\end{equation}
}

\sectionsep{}

%%%%%%%%%% After Device Effect %%%%%%%%%%
\derivsection{after ($\adev{ }$)}
{
% After device energy
\begin{equation} \label{eq:E_acct_adev}
  \Eacctadev{}
\end{equation}
}
{
% After device financial
\begin{equation} \label{eq:M_acct_adev}
  \Macctadev{}
\end{equation}
}

\sectionsep{}

%%%%%%%%%% Derivations for Device Effect %%%%%%%%%%
\derivsection{}
% Device effect: energy differences
{
%%%%%%%%%% Energy Device Effect %%%%%%%%%%
~
  
Take differences to obtain the change in energy consumption, $\Delta \radev{E} \equiv \radev{E} - \rbdev{E}$.
%
\begin{equation}
  \Delta \radev{E} = \Delta \radev{E}_s
                     + \cancelto{0}{\Delta \radev{E}_{emb}} 
                     + (\cancelto{0}{\Delta \radev{C}_{\OMd}} 
                     + \cancelto{0}{\Delta \radev{C}_o}) I_E
\end{equation}
%
Thus, 
%
\begin{equation}
\Delta \radev{E} = \Delta \radev{E}_s \; .
\end{equation}
%
Define
%
\begin{equation} \label{eq:Sdot_def}
\Sdot \equiv -\Delta \radev{E}_s \; ,
\end{equation}
%
such that
%
\begin{equation}
\Delta \radev{E} = -\Sdot \; .
\end{equation}
%
Device effect rebound ($Re_{dev}$) is obtained from 
%
\begin{equation}
Re_{dev} = 1 - \frac{\mathrm{actual \; energy \; savings \; rate}}{\mathrm{expected \; energy \; savings \; rate}} \; .
\end{equation}
%
\begin{equation}
Re_{dev} = 1 - \frac{-\Delta \radev{E}}{\Sdot} = 1 - \frac{\rate{S}_{dev}}{\Sdot} \; ,
\end{equation}
%
and
%
\begin{equation}
Re_{dev} = 0 \; .
\end{equation}
%
After only the device effect and before any other effects, 
there are no energy takebacks. 
Thus, $Re_{dev} = 0$.
%
}
{
%%%%%%%%%% Financial Device Effect %%%%%%%%%%
~
    
Use the monetary constraint ($\rbdev{M} = \radev{M}$) to obtain
%
\begin{align}
  p_E \rbdev{E}_s &+ \cancel{\rbdev{C}_{cap}} + \cancel{\rbdev{C}_{\OMd}} + \cancel{\rbdev{C}_o} + \cancelto{0}{\rbdev{N}} \nonumber \\
                  &= p_E \radev{E}_s + \cancel{\radev{C}_{cap}} + \cancel{\radev{C}_{\OMd}} + \cancel{\radev{C}_o}  + \radev{N} \; .
\end{align}
%
Prior to the device effect, there are no monetary savings from the EEU ($\rbdev{N} = 0$).
For the device effect itself, 
changes in capital; operations, maintenance, and disposal, and other costs are neglected
($\radev{C}_{cap} = \rbdev{C}_{cap}$, $\radev{C}_{\OMd} = \rbdev{C}_{\OMd}$, and $\radev{C}_o = \rbdev{C}_o$).
(Changes in $\rate{C}_{cap}$, $\rate{C}_{\OMd}$, and $\rate{C}_o$ 
are assigned later to the life cycle effect.)
Solving for $\Delta \radev{N} \equiv \radev{N} - \cancelto{0}{\rbdev{N}}$ gives 
%
\begin{equation}
  \Delta \radev{N} = \radev{N} = p_E(\rbdev{E}_s - \radev{E}_s) = p_E (-\Delta \radev{E}_s) \; .
\end{equation}
%
Substituting Eq.~\ref{eq:Sdot_def} gives
%
\begin{equation}
  \Delta \radev{N} = \radev{N} = p_E \Sdot \; .
\end{equation}
%
$\Delta \radev{N}$ represents the gross income (energy cost savings, $\rate{G}$) from the EEU, 
before any energy takeback, thus
%
\begin{equation} \label{eq:G_dot}
  \rate{G} = \Delta \radev{N} = \radev{N} = p_E \Sdot \; .
\end{equation}
}

\end{landscape}



% The next command tells RStudio to do "Compile PDF" on book.Rnw,
% instead of this chapter, thereby eliminating the need to switch back to book.Rnw 
% before making the book.
%!TEX root = HSB.Rnw

% This file contains the derivation for lifecycle effect rebound and net income.

\begin{landscape}

\linespread{1}

%%%%%%%%%%%%%%%%%%%%%%%%%%%%%%%%%%%%%%%
%%%%%%%%%% Life cycle Effect %%%%%%%%%%
%%%%%%%%%%%%%%%%%%%%%%%%%%%%%%%%%%%%%%%
\derivheader{\refstepcounter{table} Table~\thetable \label{tab:lceffect}. \bf{\LcEffect}}

\sectionsep{}

%%%%%%%%%% Before Life Cycle Effect %%%%%%%%%%
\derivsection{before ($\blc{})$}
{
% Before life cycle energy
\begin{equation}
  \Eacctblc{}
\end{equation}
}
{
% Before life cycle financial
\begin{equation}
  \Macctblc{}
\end{equation}
}

\sectionsep{}

%%%%%%%%%% After Life Cycle Effect %%%%%%%%%%
\derivsection{after ($\alc{})$}
{
% After life cycle energy
\begin{equation}
  \Eacctalc{}
\end{equation}
}
{
% After life cycle financial
\begin{equation}
  \Macctalc{}
\end{equation}
}

\sectionsep{}

%%%%%%%%%% Derivations for Life cycle Effect %%%%%%%%%%
\derivsection{}
{
%%%%%%%%%% Energy Life cycle Effect %%%%%%%%%%
~
    
Take differences to obtain the change in energy consumption, $\Delta \ralc{E} \equiv \ralc{E} - \rblc{E}$.
%
\begin{equation}
  \Delta \ralc{E} = \cancelto{0}{\Delta \ralc{E}_s}
                    + \Delta \ralc{E}_{emb}
                    + (\Delta \ralc{C}_{\OMd} + \cancelto{0}{\Delta \ralc{C}_o}) I_E
\end{equation}
%
Thus, 
%
\begin{equation}
  \Delta \ralc{E} = \Delta \ralc{E}_{emb} + \Delta \ralc{C}_{\OMd} I_E \; .
\end{equation}
%
All terms are energy takeback terms.
Divide by $\Sdot$
to create rebound terms.
%
\begin{equation}
  \frac{\Delta \ralc{E}}{\Sdot} = \frac{\Delta \ralc{E}_{emb}}{\Sdot} + \frac{\Delta \ralc{C}_{\OMd} I_E}{\Sdot}
\end{equation}
%
Define
$Re_{\life} \equiv \frac{\Delta \ralc{E}}{\Sdot}$, 
$Re_{emb} \equiv \frac{\Delta \ralc{E}_{emb}}{\Sdot}$, and
$Re_{\OMd} \equiv \frac{\Delta \ralc{C}_{\OMd} I_E}{\Sdot}$,
such that
%
\begin{equation}
  Re_{\life} = Re_{emb} + Re_{\OMd} \; .
\end{equation}
%
}
{
%%%%%%%%%% Financial Life cycle Effect %%%%%%%%%%
~

Use the monetary constraint ($\rblc{M} = \ralc{M}$) to obtain
%
\begin{align} \label{M_dot_bal_lc}
  \cancel{p_E \rblc{E}_s} &+ \rblc{C}_{cap} + \rblc{C}_{\OMd} + \cancel{\rblc{C}_o} + \rblc{N} \nonumber \\
                          &= \cancel{p_E \ralc{E}_s} + \ralc{C}_{cap} + \ralc{C}_{\OMd} + \cancel{\ralc{C}_o}  + \ralc{N} \; .
\end{align}
%
For the life cycle effect, 
there is no change in device energy consumption rate or expenditures on other good in the economy 
($\ralc{E}_s = \rblc{E}_s$, $\ralc{C}_o = \rblc{C}_o$).
(The change in device energy consumption rate is accounted in the device effect.
Changes in the rate of other purchases are accounted in substitution and income effects.)
Solving for $\Delta \ralc{N} \equiv \ralc{N} - \rblc{N}$ gives
%
\begin{equation}
  \Delta \ralc{N} = - \Delta \ralc{C}_{cap} - \Delta \ralc{C}_{\OMd} \; .
\end{equation}
%
Using $\ralc{N} = \rblc{N} + \Delta \ralc{N}$, we find that
%
\begin{equation}
  \ralc{N} = \rate{G} - \Delta \ralc{C}_{cap} - \Delta \ralc{C}_{\OMd} \; .
\end{equation}
%
}

\end{landscape}



% The next command tells RStudio to do "Compile PDF" on HSB.Rnw,
% instead of this file, thereby eliminating the need to switch back to HSB.Rnw 
% before building the paper.
%!TEX root = ../HSB.Rnw

% This file contains the derivation for substitution effect rebound and net income.

\begin{landscape}

\linespread{1}

%%%%%%%%%%%%%%%%%%%%%%%%%%%%%%%%%%%%%%%%%
%%%%%%%%%% Substitution Effect %%%%%%%%%%
%%%%%%%%%%%%%%%%%%%%%%%%%%%%%%%%%%%%%%%%%
\derivheader{\refstepcounter{table} Table~\thetable \label{tab:subeffect}. \bf{\SubEffect}}

\sectionsep{}

%%%%%%%%%% Before Substitution Effect %%%%%%%%%%
\derivsection{before ($\bsub{})$}
{
% Before substitution energy
\begin{equation}
  \Eacctbsub{}
\end{equation}
}
{
% Before substitution financial
\begin{equation}
  \Macctbsub{}
\end{equation}
}

\sectionsep{}


%%%%%%%%%% After Substitution Effect %%%%%%%%%%
\derivsection{after ($\asub{~})$}
{
% After substitution energy
\begin{equation}
  \Eacctasub{}
\end{equation}
}
{
% After substitution financial
\begin{equation}
  \Macctasub{}
\end{equation}
}

\sectionsep{}

%%%%%%%%%% Derivations for Substitution Effect %%%%%%%%%%
\derivsection{}
% Substitution effect: energy differences
{
%%%%%%%%%% Energy Substitution Effect %%%%%%%%%%
~
  
Take differences to obtain the change in energy consumption, $\Delta \rasub{E} \equiv \rasub{E} - \rbsub{E}$.
%
\begin{equation}
  \Delta \rasub{E} = \Delta \rasub{E}_s 
                      + \cancelto{0}{\Delta \rasub{E}_{emb}} 
                      + (\cancelto{0}{\Delta \rasub{C}_{\OMd}} + \Delta \rasub{C}_o) I_E
\end{equation}
%
Thus, 
%
\begin{equation}
  \Delta \rasub{E} = \Delta \rasub{E}_s + \Delta \rasub{C}_o I_E \; .
\end{equation}
%
All terms are energy takeback terms.
Divide by $\Sdot$
to create rebound terms.
%
\begin{equation}
    \frac{\Delta \rasub{E}}{\Sdot} = \frac{\Delta \rasub{E}_s}{\Sdot} + \frac{\Delta \rasub{C}_o I_E}{\Sdot}
\end{equation}
%
Define 
$Re_{sub} \equiv \frac{\Delta \rasub{E}}{\Sdot}$, 
$Re_{dsub} \equiv \frac{\Delta \rasub{E}_s}{\Sdot}$, and
$Re_{isub} \equiv \frac{\Delta \rasub{C}_o I_E}{\Sdot}$,
such that
%
\begin{equation} \label{eq:Re_sub_def}
  Re_{sub} = Re_{dsub} + Re_{isub} \; .
\end{equation}

}
{
%%%%%%%%%% Financial Substitution Effect %%%%%%%%%%
~
  
Use the monetary constraint ($\rbsub{M} = \rasub{M}$) to obtain
%
\begin{align}
  p_E \raempl{E}_s &+ \cancel{\raempl{C}_{cap}} + \cancel{\raempl{C}_{\OMd}} + \raempl{C}_o + \raempl{N} \nonumber \\
                   &= p_E \rasub{E}_s + \cancel{\rasub{C}_{cap}} + \cancel{\rasub{C}_{\OMd}} + \rasub{C}_o + \rasub{N} \; .
\end{align}
%
For the substitution effect, there is no change in capital or operations, maintenance, and disposal costs
($\rasub{C}_{cap} = \rbsub{C}_{cap}$ and $\rasub{C}_{\OMd} = \rbsub{C}_{\OMd}$).
Solving for $\Delta \rasub{N} \equiv \rasub{N} - \rbsub{N}$ gives
%
\begin{equation}
  \Delta \rasub{N} = - p_E \Delta \rasub{E}_s - \Delta \rasub{C}_o \; .
\end{equation}
%
With $\rasub{N} = \rbsub{N} + \Delta \rasub{N}$, 
we find that
%
\begin{equation} \label{eq:N_dot_after_sub}
  \rasub{N} = \rate{G} - \Delta \rbsub{C}_{cap} - \Delta \rbsub{C}_{\OMd} - p_E \Delta \rasub{E}_s - \Delta \rasub{C}_o \; .
\end{equation}
%
}

\end{landscape}



% The next command tells RStudio to do "Compile PDF" on book.Rnw,
% instead of this chapter, thereby eliminating the need to switch back to book.Rnw 
% before making the book.
%!TEX root = HSB.Rnw

% This file contains the derivation for income effect rebound and net income.

\begin{landscape}

\linespread{1}

%%%%%%%%%%%%%%%%%%%%%%%%%%%%%%%%%%%
%%%%%%%%%% Income Effect %%%%%%%%%%
%%%%%%%%%%%%%%%%%%%%%%%%%%%%%%%%%%%
\derivheader{\refstepcounter{table} Table~\thetable \label{tab:inceffect}. \bf{\IncEffect}}

\sectionsep{}

%%%%%%%%%% Before Income Effect %%%%%%%%%%
\derivsection{before ($\binc{~})$}
{
% Before income energy
\begin{equation}
\Eacctbinc{}
\end{equation}
}
{
% Before income financial
\begin{equation}
\Macctbinc{}
\end{equation}
}

\sectionsep{}

%%%%%%%%%% After Income Effect %%%%%%%%%%
\derivsection{after ($\ainc{~})$}
{
% After income energy
\begin{equation}
\Eacctainc{}
\end{equation}
}
{
% After income financial
\begin{equation}
\Macctainc{}
\end{equation}
}

\sectionsep{}

%%%%%%%%%%% Derivations for Income Effect %%%%%%%%%%
\derivsection{}
% Income effect: energy differences
{
%%%%%%%%%% Energy Income Effect %%%%%%%%%%
~

Take differences to obtain the change in energy consumption, $\Delta \rainc{E} \equiv \rainc{E} - \rbinc{E}$.
%
\begin{equation}
  \Delta \rainc{E} = \Delta \rainc{E}_s 
                     + \cancelto{0}{\Delta \rainc{E}_{emb}}
                     + (\cancelto{0}{\Delta \rainc{C}_{\OMd}} + \Delta \rainc{C}_o) I_E
\end{equation}
%
Thus, 
%
\begin{equation}
  \Delta \rainc{E} = \Delta \rainc{E}_s + \Delta \rainc{C}_o I_E
\end{equation}
%
All terms are energy takeback terms.
Divide by $\Sdot$
to create rebound terms.
%
\begin{equation}
  \frac{\Delta \rainc{E}}{\Sdot} = \frac{\Delta \rainc{E}_s}{\Sdot} + \frac{\Delta \rainc{C}_o I_E}{\Sdot}
\end{equation}
%
Define 
$Re_{inc} \equiv \frac{\Delta \rainc{E}}{\Sdot}$, 
$Re_{dinc} \equiv \frac{\Delta \rainc{E}_s}{\Sdot}$, and 
$Re_{iinc} \equiv \frac{\Delta \rainc{C}_o I_E}{\Sdot}$,
such that
%
\begin{equation}
  Re_{inc} = Re_{dinc} + Re_{iinc} \; .
\end{equation}
%
}
{
%%%%%%%%%% Financial Income Effect %%%%%%%%%%
~

Use the monetary constraint ($\rbinc{M} = \rainc{M}$) to obtain
%
\begin{align}
  p_E \rasub{E}_s &+ \cancel{\rasub{C}_{cap}} + \cancel{\rasub{C}_{\OMd}} + \rasub{C}_o + \rasub{N} \nonumber \\
                  &= p_E \rainc{E}_s + \cancel{\rainc{C}_{cap}} + \cancel{\rainc{C}_{\OMd}} + \rainc{C}_o + \cancelto{0}{\rainc{N}} \; .
\end{align}
%
For the income effect, there is no change in capital or operations, maintainance, and disposal costs
($\rasub{C}_{cap} = \rbsub{C}_{cap}$ and $\rasub{C}_{\OMd} = \rbsub{C}_{\OMd}$).
Notably, $\rainc{N} = 0$,
because it is assumed that all net income ($\rbinc{N}$) is spent on
more energy service ($\rainc{E}_s > \rbinc{E}_s$)
and
additional purchases in the economy ($\rainc{C}_o > \rbinc{C}_o$).
Solving for $\rbinc{N}$ gives 
%
\begin{equation}
  \rbinc{N} = p_E \Delta \rainc{E}_s + \Delta \rainc{C}_o \; ,
\end{equation}
%
the budget constraint for the income effect.
}
\end{landscape}



% The next command tells RStudio to do "Compile PDF" on HSB.Rnw,
% instead of this file, thereby eliminating the need to switch back to HSB.Rnw 
% before building the paper.
%!TEX root = ../HSB.Rnw

% This file contains the derivation for income effect rebound and net income.

\begin{landscape}

\linespread{1}

%%%%%%%%%%%%%%%%%%%%%%%%%%%%%%%%%%%%%%%%%
%%%%%%%%%% Productivity Effect %%%%%%%%%%
%%%%%%%%%%%%%%%%%%%%%%%%%%%%%%%%%%%%%%%%%
\derivheader{\refstepcounter{table} Table~\thetable \label{tab:prodeffect}. \bf{\MacroEffect}}

\sectionsep{}

%%%%%%%%%% Before Productivity Effect %%%%%%%%%%
\derivsection{before ($\bprod{~})$}
{
% Before productivity energy
\begin{equation}
  \rbprod{E}
\end{equation}
}
{
% Before productivity financial
}

\sectionsep{}

%%%%%%%%%% After Productivity Effect %%%%%%%%%%
\derivsection{after ($\aprod{~})$}
{
% After productivity energy
\begin{equation}
\raprod{E}
\end{equation}
}
{
% After productivity financial
}

\sectionsep{}

%%%%%%%%%%% Derivations for productivty Effect %%%%%%%%%%
\derivsection{}
% Productivity effect: energy differences
{
%%%%%%%%%% Energy Productivity Effect %%%%%%%%%%
~

Take differences to obtain the change in energy consumption,
%
\begin{equation}
  \Delta \raprod{E} \equiv \raprod{E} - \rbprod{E} \; .
\end{equation}
%
The energy change due to the productivity effect ($\Delta \raprod{E}$) 
is a scalar multiple ($k$) of net income ($\rbinc{N}$), 
assumed to be spent at the energy intensity of the economy ($I_E$).
%
\begin{equation}
  \Delta \raprod{E} = k \rbinc{N} I_E
\end{equation}
%
All terms are energy takeback terms.
Divide by $\Sdot$
to create rebound terms.
%
\begin{equation}
  \frac{\Delta \raprod{E}}{\Sdot} = \frac{k \rbinc{N} I_E}{\Sdot}
\end{equation}
%
Define 
$Re_{\macro} \equiv \frac{\Delta \raprod{E}}{\Sdot}$, 
such that
%
\begin{equation}
  Re_{\macro} = \frac{k \rbinc{N} I_E}{\Sdot} \; . \tag{\ref{eq:Re_prod_def}}
\end{equation}
%
}
{
%%%%%%%%%% Financial Productivity Effect %%%%%%%%%%
~
\centering

N/A
}
\end{landscape}



%++++++++++++++++++++++++++++++
\subsection{Rebound expressions}
\label{sec:rebound_expressions}
%++++++++++++++++++++++++++++++

All that remains is to determine expressions for each rebound term, 
beginning with the expected energy savings rate~($\Sdot$), which
appears in the denominator of all rebound expressions.


%------------------------------
\subsubsection{Expected energy savings ($\Sdot$)} 
\label{sec:Sdot}
%------------------------------

$\Sdot$ is the reduction of energy consumption rate
by the device due to the EEU.
No other effects are considered.
%
\begin{equation}
  \Sdot \equiv \rbdev{E}_s - \radev{E}_s
\end{equation}
%
The final energy consumption rates ($\rbdev{E}_s$ and $\radev{E}_s$) 
can be written as the typical relationship of Eq.~\ref{eq:typ_qs_eta_Edot} in the forms
$\rbdev{E}_s = \frac{\rbdev{q}_s}{\bdev{\eta}}$ and 
$\radev{E}_s = \frac{\radev{q}_s}{\adev{\eta}}$. 
%
\begin{equation}
  \Sdot = \frac{\rbdev{q}_s}{\bdev{\eta}} - \frac{\radev{q}_s}{\adev{\eta}}
\end{equation}
%
With reference to Table~\ref{tab:analysis_assumptions}, 
we use $\radev{q}_s = \rbdev{q}_s$ and $\adev{\eta} = \aprod{\eta}$ to obtain
%
\begin{equation}
  \Sdot = \frac{\rbdev{q}_s}{\bdev{\eta}} - \frac{\rbdev{q}_s}{\aprod{\eta}} \; .
\end{equation}
%
When the EEU increases efficiency such that $\aprod{\eta} > \bdev{\eta}$,
expected energy savings grows ($\Sdot > 0$)
as the rate of final energy consumption declines,
as expected.
As $\aprod{\eta} \rightarrow \infty$,
all final energy consumption is eliminated ($\radev{E}_s \rightarrow 0$), and
$\Sdot = \frac{\rbdev{q}_s}{\bdev{\eta}} = \rbdev{E}_s$.
(Of course, $\aprod{\eta} \rightarrow \infty$ is impossible. 
See \citet{Paoli:2020aa} for a recent discussion of upper limits to device efficiencies.)

After rearrangement and using $\rbdev{E}_s = \frac{\rbdev{q}_s}{\bdev{\eta}}$, 
we obtain a convenient form
%
\begin{equation} \label{eq:Sdot}
  \Sdot = \Sdoteqn \; .
\end{equation}


%------------------------------
\subsubsection{\Deveffect} 
\label{sec:Re_dev}
%------------------------------

As shown in Eq.~\ref{eq:Re_dev0}, 
rebound from the device effect is $Re_{dev} = 0$.


%------------------------------
\subsubsection{\Lceffect{}} 
\label{sec:Re_lc}
%------------------------------

Rebound effects can occur at any point in the life cycle of an energy conversion device,
from manufacturing and distribution 
to the use phase (operations and maintenance),
and finally to disposal.
For simplicity, we group operations and maintenance with disposal to form
two distinct life cycle rebound effects:
%
\begin{enumerate*}[label={(\alph*)}]
	
  \item an embodied energy effect ($Re_{emb}$) and 
  
  \item an operations, maintenance, and disposal effect ($Re_{\OMd}$).
    
\end{enumerate*}


%..............................
\paragraph{$Re_{emb}$}
\label{sec:Re_emb}
%..............................

We define embodied energy consistent with the energy/exergy analysis literature
to be the sum of all final energy consumed (direct and indirect)
in the production of the device.
The EEU
causes the embodied final energy of the device to change
from $\rblc{E}_{emb}$ to $\ralc{E}_{emb}$.

Energy is embodied in the device during manufacturing and distribution supply chains
prior to consumer acquisition of the device.
No energy is embodied in the device while in service.
However, for simplicity, we spread all embodied energy
over the lifetime of the device,
an equal amount assigned to each period.
We later take the same approach to capital costs and
operations, maintenance, and disposal costs.
A justification for spreading embodied energy purchase costs comes from considering
staggered device replacements by many consumers across several years.
In the aggregate, staggered replacements
work out to about the same embodied energy in every period.

Thus, we allocate embodied energy over the life of the original and upgraded devices
($\blc{t}$ and $\alc{t}$, respectively)
to obtain embodied energy rates, such that
$\rblc{E}_{emb} = \blc{E}_{emb} / \blc{t}$
and 
$\ralc{E}_{emb} = \alc{E}_{emb} / \alc{t}$.
The change in embodied final energy due to the EEU (expressed as a rate) is given by
$\Delta \ralc{E}_{emb} = \ralc{E}_{emb} - \rblc{E}_{emb}$.
After substitution and algebraic rearrangement,
the change in embodied energy rate due to the EEU can be expressed as
%
\begin{equation} \label{eq:delta_embodied}
  \Delta \ralc{E}_{emb} = \left( \frac{\alc{E}_{emb}}{\blc{E}_{emb}}
  \frac{\blc{t}}{\alc{t}} - 1 \right) \rblc{E}_{emb} \, .
\end{equation}

$\Delta \ralc{E}_{emb}$ represents energy savings taken back due to embodied energy effects.
Thus, Equation~\ref{eq:Re_takeback} can be employed to write embodied energy rebound as
%
\begin{equation} \label{eq:Re_emb}
  Re_{emb} = \Reembeqn{} \, .
\end{equation}

Embodied energy rebound can be either positive or negative, depending on 
the sign of the term
$(\alc{E}_{emb}/\blc{E}_{emb})(\blc{t}/\alc{t}) - 1$.
Typically, but not always,
rising in energy efficiency is associated with increased device complexity
and more embodied energy,
such that $\alc{E}_{emb} > \blc{E}_{emb}$ and $Re_{emb} > 0$.
However, if the upgraded device has longer life than the original device
($\alc{t} > \blc{t}$),
$\Delta \ralc{E}_{emb}$ can be negative,
meaning that the upgraded device has a lower embodied energy rate than the original device,
yielding $Re_{emb} < 0$.


%..............................
\paragraph{$Re_{\OMd}$} 
\label{sec:Re_OMd}
%..............................

In addition to embodied energy effects, 
rebound can be associated with energy demanded by operations, maintenance, and disposal~($\OMd$) expenditures.
Operations and maintenance expenditures are typically modeled as a per-year expense, a rate (e.g., $\rblc{C}_{\OM}$).
Disposal costs (e.g., $\blc{C}_d$) are one-time expenses incurred at the end of the useful life of the energy conversion device.
Like embodied energy, we spread disposal costs across the lifetime 
of the original and upgraded devices ($\blc{t}$ and $\alc{t}$, respectively)
to form cost rates such that $\rblc{C}_{\OMd} = \rblc{C}_{\OM} + \blc{C}_d/\blc{t}$
and
$\ralc{C}_{\OMd} = \ralc{C}_{\OM} + \alc{C}_d/\alc{t}$.

We assume, for simplicity, that $\OMd$ expenditures indicate energy consumption
elsewhere in the economy at its energy intensity~($I_E$).
Therefore, the change in energy consumption rate caused by a change in $\OMd$ expenditures
is given by $\Delta \ralc{C}_{\OMd} I_E$.
This term represents energy takeback, so operations, maintenance, and disposal rebound is given by
%
\begin{equation} \label{eq:Re_OMd_def}
  Re_{\OMd} = \frac{\Delta \ralc{C}_{\OMd} I_E}{\Sdot} \; ,
\end{equation}
%
as shown in Table~\ref{tab:lceffect}.
Slight rearrangement gives
%
\begin{equation} \label{eq:Re_OMd}
  Re_{\OMd} = \frac{\left( \frac{\ralc{C}_{\OMd}}{\rblc{C}_{\OMd}} - 1 \right) \rblc{C}_{\OMd} I_E}{\Sdot} \; .
\end{equation}

Rebound from operations, maintenance, and disposal can be positive or negative,
depending on the sign of the term $\ralc{C}_{\OMd}/\rblc{C}_{\OMd} - 1$.


%------------------------------
\subsubsection{\Subeffect{}} 
\label{sec:Re_sub}
%------------------------------

Two terms comprise substitution effect rebound, 
direct substitution rebound ($Re_{dsub}$) and 
indirect substitution rebound ($Re_{isub}$).
This section derives each in terms of energy service efficiencies ($\bdev{\eta}$ and $\aprod{\eta}$),
the service price elasticity of consumption ($\epsilon_{p_s}$), and 
utility function parameters ($\alpha$ and $\beta$).
We begin with derivation of two ratios that are helpful later, 
$\frac{\rasub{q}_s}{\rbsub{q}_s}$ and 
$\frac{\rasub{q}_o}{\rbsub{q}_o}$.


%..............................
\paragraph{Expressions for two ratios, $\frac{\rasub{q}_s}{\rbsub{q}_s}$ and $\frac{\rasub{q}_o}{\rbsub{q}_o}$} 
\label{sec:two_ratios}
%..............................

The EEU's energy efficiency increase
($\aprod{\eta} > \bdev{\eta}$) 
causes the price of the energy service provided by the device to fall
($\aprod{p}_s < \bdev{p}_s$). 
The substitution effect quantifies the amount by which 
the device owner, in response, 
increases the consumption rate of the energy service ($\rasub{q}_s > \rbsub{q}_s$) and 
decreases the consumption rate of other goods ($\rasub{q}_o < \rbsub{q}_o$).

The relationship between energy service price and energy service consumption rate
is given by the service price elasticity of consumption~($\epsilon_{p_s}$), 
such that
%
\begin{equation}
  \frac{\rasub{q}_s}{\rbsub{q}_s} = \left( \frac{\aprod{p}_s}{\bdev{p}_s} \right)^{\epsilon_{p_s}} \; .
\end{equation}
%
We assume that the service price elasticity of consumption~($\epsilon_{p_s}$) is observable 
from service prices only. 
Note that a negative value for the service price elasticity of consumption is expected ($\epsilon_{p_s} < 0$),
such that when the energy service price decreases ($\aprod{p}_s < \bdev{p}_s$), 
the rate of energy service consumption increases ($\rasub{q}_s > \rbsub{q}_s$).

Substituting the typical relationship of Eq.~\ref{eq:ps_pE_eta} in the form
$\bdev{p}_s = \frac{\bdev{p}_E}{\bdev{\eta}}$ and
$\aprod{p}_s = \frac{\bdev{p}_E}{\aprod{\eta}}$
gives 
%
\begin{equation} \label{eq:q_ratio_func_of_eps}
  \frac{\rasub{q}_s}{\rbsub{q}_s} = \left( \frac{\aprod{\eta}}{\bdev{\eta}} \right)^{-\epsilon_{p_s}} \; .
\end{equation}
%
Again, note that the energy service price elasticity of consumption
is negative ($\epsilon_{p_s} < 0$), so
as energy service efficiency increases ($\aprod{\eta} > \bdev{\eta}$),
the energy service consumption rate increases ($\rasub{q}_s > \rbsub{q}_s$).

To quantify the substitution effect on other purchases,
we introduce a utility ($\rate{u}$) constraint.
%
\begin{equation}
  \rbsub{u} = \rasub{u}
\end{equation}
%
We assume a utility function of the general form
$\rate{u} = A \left( \frac{\rate{q}_s}{\rate{q}_{s,0}} \right)^\alpha \left( \frac{\rate{q}_o}{\rate{q}_{o,0}} \right)^\beta$,
where both $\alpha$ and $\beta$ are positive.
Applying the utility constraint gives
%
\begin{equation}
  (\rbsub{q}_s)^\alpha (\rbsub{q}_o)^\beta = (\rasub{q}_s)^\alpha (\rasub{q}_o)^\beta \; .
\end{equation}
%
Rearrangement gives
%
\begin{equation}
  \frac{\rasub{q}_s}{\rbsub{q}_s} = \left( \frac{\rasub{q}_o}{\rbsub{q}_o} \right) ^ {- \beta / \alpha } \; .
\end{equation}
%
Equating to Eq.~\ref{eq:q_ratio_func_of_eps} and rearranging gives
%
\begin{equation} \label{eq:qohat_qostar}
  \frac{\rasub{q}_o}{\rbsub{q}_o} = \left( \frac{\aprod{\eta}}{\bdev{\eta}}\right) ^ {\epsilon_{p_s} (\alpha / \beta)} \; .
\end{equation}
%
Note that the service price elasticity of consumption ($\epsilon_{p_s}$)
is negative and utility function exponents ($\alpha$ and $\beta$) are positive, such that
when energy service efficiency increases ($\aprod{\eta} > \bdev{\eta}$),
the quantity of other goods demanded decreases ($\rasub{q}_o < \rbsub{q}_o$),
as expected for the substitution effect.
Note, too, that if the device owner derives relatively more utility 
from the energy service compared to other purchases ($\alpha > \beta$), 
the reduction in other purchases is enhanced, because $\frac{\alpha}{\beta} > 1$.

Next, we derive an expression for direct substitution rebound ($Re_{dsub}$).


%..............................
\paragraph{Expression for $Re_{dsub}$} 
\label{sec:Re_dsub}
%..............................

As shown in Table~\ref{tab:subeffect}, direct substitution rebound is defined as 
%
\begin{equation} \label{eq:Re_dsub_def}
  Re_{dsub} \equiv \frac{\Delta \rasub{E}_s}{\Sdot} \; .
\end{equation}
%
Expansion of the difference term and 
substitution of the typical relationship of Eq.~\ref{eq:typ_qs_eta_Edot} in the form 
$\rasub{E}_s = \frac{\rasub{q}_s}{\aprod{\eta}}$ 
and 
$\rbsub{E}_s = \frac{\rbsub{q}_s}{\bdev{\eta}}$ 
gives
%
\begin{equation}
   Re_{dsub} = \frac{\rasub{E}_s - \rbsub{E}_s}{\Sdot} \; ,
\end{equation}
%
and
%
\begin{equation}
     Re_{dsub} = \frac{\frac{\rasub{q}_s}{\aprod{\eta}} - \frac{\rbsub{q}_s}{\bdev{\eta}}}{\Sdot} \; .
\end{equation}
%
Substitution of Eq.~\ref{eq:Sdot} gives
%
\begin{equation}
  Re_{dsub} = \frac{\frac{\rasub{q}_s}{\aprod{\eta}} - \frac{\rbsub{q}_s}{\bdev{\eta}}}
              {\Sdoteqn} \; .
\end{equation}
%
Rearrangement of the numerator and canceling terms gives
%
\begin{equation}
  Re_{dsub} = \frac{\left( \frac{\rasub{q}_s}{\rbsub{q}_s} - 1 \right) \frac{\rbsub{q}_s}{\cancel{\aprod{\eta}}} }
              {\left( \frac{\aprod{\eta}}{\bdev{\eta}} - 1 \right)\!\frac{\bdev{\eta}}{\cancel{\aprod{\eta}}} \rbdev{E}_s} \; ,
\end{equation}
%
and
%
\begin{equation}
    Re_{dsub} = \frac{\frac{\rasub{q}_s}{\rbsub{q}_s} - 1}{\frac{\aprod{\eta}}{\bdev{\eta}} - 1} \; \; 
                \frac{\frac{\rbsub{q}_s}{\bdev{\eta}}}{\rbdev{E}_s} \; .
\end{equation}
%
Noting that $\frac{\rbsub{q}_s}{\rbdev{\eta}} = \frac{\rbdev{q}_s}{\rbdev{\eta}} = \rbdev{E}_s$,
canceling $\rbdev{E}_s$ terms,
and substituting Eq.~\ref{eq:q_ratio_func_of_eps} gives
%
\begin{equation} \label{eq:Re_dsub}
  Re_{dsub} = \Redsubeqn \; .
\end{equation}

Note that the service price elasticity of energy service consumption is
expected to be negative ($\epsilon_{p_s} < 0$).
When $\epsilon_{p_s} \in (-1, 0)$, as expected,
the direct substitution rebound will be positive but less than 1 ($0 < Re_{dsub} < 1$).
For example, when $\epsilon_{p_s} = -0.2$ and $\frac{\aprod{\eta}}{\bdev{\eta}} = 2$, 
$Re_{dsub} = 0.15$.

Finally, we derive an expression for indirect substitution rebound ($Re_{isub}$).


%..............................
\paragraph{Expression for $Re_{isub}$} 
\label{sec:Re_isub}
%..............................

The increase in consumption of the energy service after its price drop
substitutes for consumption of other goods in the economy, 
subject to a utility constraint. 
The reduction in spending on other good in the economy 
is captured by indirect substitution rebound~($Re_{isub}$).

To derive an expression for indirect substitution rebound, 
we begin with the definition of $Re_{isub}$ 
from Table~\ref{tab:subeffect}:
%
\begin{equation} \label{eq:Re_isub_dev}
  Re_{isub} \equiv \frac{\Delta \rasub{C}_o I_E}{\Sdot} \; .
\end{equation}
%
Expansion of the difference term to $\rasub{C}_o - \rbsub{C}_o$ and rearranging the numerator gives
%
\begin{equation} \label{eq:Re_isub_prelim}
  Re_{isub} = \frac{\left( \frac{\rasub{C}_o}{\rbsub{C}_o} - 1  \right) \rbsub{C}_o I_E} {\Sdot} \; .
\end{equation}
%
We assume a basket of other goods purchased in the economy, 
each ($i$) with its own price ($p_{o,i}$) and rate of consumption ($q_{o,i}$), 
such that the average price of all other good purchased in the economy~($p_o$) is given by
%
\begin{equation}
  p_o = \frac{\sum\limits_i \bdev{p}_{o,i} \rbdev{q}_{o,i}}{\sum\limits_i \rbdev{q}_{o,i}} \; .
\end{equation}
%
Then, the cost rate of other purchases in the economy can be given as
%
\begin{equation}
  \rbsub{C}_o = \bsub{p}_o \rbsub{q}_o \; ,
\end{equation}
%
and
%
\begin{equation}
  \rasub{C}_o = \asub{p}_o \rasub{q}_o \; .
\end{equation}
%
Assuming that the average price is unchanged across the substitution effect, 
such that $\asub{p}_o = \bsub{p}_o$, 
the preceding two equations can be set equal to find
%
\begin{equation}
  \frac{\rasub{C}_o}{\rbsub{C}_o} = \frac{\rasub{q}_o}{\rbsub{q}_o} \; .
\end{equation}
%
Substituting into Eq.~\ref{eq:Re_isub_prelim} gives
%
\begin{equation}
  Re_{isub} = \frac{\left( \frac{\rasub{q}_o}{\rbsub{q}_o} - 1  \right) \rbsub{C}_o I_E} {\Sdot} \; .
\end{equation}
% %
Substituting Eqs.~\ref{eq:qohat_qostar} and~\ref{eq:Sdot} gives
%
\begin{equation}
  Re_{isub} = \frac{\left[ \left(\frac{\aprod{\eta}}{\bdev{\eta}} \right)
                  ^{\epsilon_{p_s} (\alpha / \beta )} - 1  \right] \rbsub{C}_o I_E}
                  {\Sdoteqn} \; .
\end{equation}
%
Noting that $\rbsub{C}_o = \rbdev{C}_o$ and rearranging yields
%
\begin{equation} \label{eq:Re_isub}
  Re_{isub} = \Reisubeqn{} \; .
\end{equation}

Because the service price elasticity of consumption is negative ($\epsilon_{p_s} < 0$), 
the service efficiency ratio is greater than 1 ($\aprod{\eta} > \bdev{\eta}$), 
and
the ratio of utility exponents is positive ($\alpha / \beta > 0$),
indirect substitution rebound will be negative always ($Re_{isub} < 0$), 
as expected.
Negative rebound indicates that indirect substitution reduces the energy takeback by direct substitution.
Indirect substitution rebound becomes more negative 
when the ratio of utility exponents ($\alpha / \beta$) gets larger.


%------------------------------
\subsubsection{\Inceffect{}} 
\label{sec:Re_inc}
%------------------------------

Rebound from the income effect rebound quantifies the rate of additional energy demand 
that arises when the device owner spends net
income from the EEU.
Derivations of expressions for gross and net income ($\rate{G}$ and $\rate{N}$, respectively) from the 
\deveffect{} ($\rate{G}$), through the \lceffect{} ($\ralc{N}$), to the \subeffect{} ($\rasub{N}$)
are presented in Tables~\ref{tab:deveffect}--\ref{tab:subeffect}.
Gross income from the EEU is given by \ref{eq:G_dot}
as $\radev{G} = p_E \Sdot$. 
In combination, the life cycle effect and the substitution effect leave the device owner with
\emph{net} income ($\rasub{N}$) from the EEU,
as shown in Eq.~\ref{eq:N_dot_after_sub}.
Total income before the EEU is $\rbdev{M}$.
Total income after the substitution effect is $\rbdev{M} + \rasub{N}$.

In this model, all net income ($\rasub{N}$) is spent on either 
%
\begin{enumerate*}[label={(\alph*)}]
	
  \item additional energy service ($\rainc{q}_s > \rbinc{q}_s$) or
  
  \item additional other goods ($\rainc{q}_o > \rbinc{q}_o$).
    
\end{enumerate*}
%
The direct income elasticity of consumption ($\epsilon_{dinc}$) 
quantifies the amount of net income spent 
on more of the energy service ($\rainc{q}_s > \rbinc{q}_s$).
The budget constraint for the income effect (Eq.~\ref{eq:inc_budget_constraint}) 
means that leftover income is spent on other goods.

The purpose of this section is derivation of expressions for 
direct income rebound~($Re_{dinc}$) and indirect income rebound~($Re_{iinc}$).
But we first derive expressions for later use.


%..............................
\paragraph{Expression for $\frac{\rainc{q}_s}{\rbinc{q}_s}$}
\label{sec:qs_ratio}
%..............................

The ratio of rates of energy service consumed across the income effect is given by
%
\begin{equation}
  \frac{\rainc{q}_s}{\rbinc{q}_s} = \left( \frac{\rbinc{M} + \rbinc{N}}{\rbinc{M}} \right) ^ {\epsilon_{dinc}} \; .
\end{equation}
%
Recognizing that $\rbinc{M} = \rbsub{M} = \rblc{M} = \rbdev{M}$ and rearranging slightly gives
%
\begin{equation} \label{eq:q_ratio_across_inc}
  \frac{\rainc{q}_s}{\rbinc{q}_s} = \left( 1 + \frac{\rbinc{N}}{\rbdev{M}} \right) ^ {\epsilon_{dinc}} \; .
\end{equation}


%..............................
\paragraph{Expression for $\rbinc{E}_s$} 
\label{sec:E_dot_s_hat_expression}
%..............................

An expression for $\rbinc{E}_s$ that will be helpful later
begins with
%
\begin{equation}
  \rbinc{E}_s = \left( \frac{\rasub{E}_s}{\rbsub{E}_s} \right)
                \left( \frac{\ralc{E}_s}{\rblc{E}_s} \right)
                \left( \frac{\radev{E}_s}{\rbdev{E}_s} \right)
                \rbdev{E}_s \; .
\end{equation}
%
Substituting the typical relationship from Eq.~\ref{eq:typ_qs_eta_Edot} and noting efficiency ($\eta$)
equalities from Table~\ref{tab:analysis_assumptions} gives
%
\begin{equation}
  \rbinc{E}_s = \left( \frac{\rasub{q}_s / \cancel{\aprod{\eta}}}{\rbsub{q}_s / \cancel{\aprod{\eta}}} \right)
                \left( \frac{\ralc{q}_s / \cancel{\aprod{\eta}}}{\rblc{q}_s / \cancel{\aprod{\eta}}} \right)
                \left( \frac{\radev{q}_s / \aprod{\eta}}{\rbdev{q}_s / \bdev{\eta}} \right)
                \rbdev{E}_s \; .
\end{equation}
%
Canceling terms yields
%
\begin{equation}
  \rbinc{E}_s = \left( \frac{\rasub{q}_s}{\rbsub{q}_s} \right)
                \left( \cancel{\frac{\ralc{q}_s}{\rblc{q}_s}} \right)
                \left( \cancel{\frac{\radev{q}_s}{\rbdev{q}_s}} \right)
                \left( \frac{\bdev{\eta}}{\aprod{\eta}}  \right)
                \rbdev{E}_s \; .
\end{equation}
%
Noting energy service consumption rate ($\rate{q}_s$) equalities from Table~\ref{tab:analysis_assumptions} gives
%
\begin{equation} \label{eq:E_dot_s_hat}
  \rbinc{E}_s = \frac{\rasub{q}_s}{\rbsub{q}_s}
                \frac{\bdev{\eta}}{\aprod{\eta}}
                \rbdev{E}_s \; .
\end{equation}

%..............................
\paragraph{Expression for $\frac{\rbinc{N} I_E}{\Sdot}$}
\label{sec:N_dot_hat_I_E_over_Sdot}
%..............................

Another term of use later is $\frac{\rbinc{N} I_E}{\Sdot}$.
We begin by substituting Eq.~\ref{eq:N_dot_after_sub} for net income~($\rbinc{N}$).
%
\begin{equation}
  \frac{\rbinc{N} I_E}{\Sdot} = \frac{\radev{G} I_E}{\Sdot}
                                - \frac{\Delta \ralc{C}_{cap} I_E}{\Sdot}
                                - \frac{\Delta \ralc{C}_{\OMd} I_E}{\Sdot}
                                - \frac{p_E I_E \Delta \rasub{E}_s}{\Sdot}
                                - \frac{\Delta \rasub{C}_o I_E}{\Sdot}
\end{equation}
%
Substituting Eqs.~\ref{eq:G_dot}, \ref{eq:Re_OMd_def}, \ref{eq:Re_dsub_def}, \ref{eq:Re_isub_dev} gives
%
\begin{equation}
  \frac{\rbinc{N} I_E}{\Sdot} = \frac{p_E \cancel{\Sdot} I_E}{\cancel{\Sdot}}
                                - \frac{\Delta \ralc{C}_{cap} I_E}{\Sdot}
                                - Re_{\OMd}
                                - p_E I_E Re_{dsub}
                                - Re_{isub} \; .
\end{equation}
%
Canceling terms and defining $Re_{cap}$ as
%
\begin{equation} \label{eq:Re_cap}
  Re_{cap} \equiv \frac{\Delta \ralc{C}_{cap} I_E}{\Sdot} \; ,
\end{equation}
%
gives
%
\begin{equation} \label{eq:N_dot_I_E_Sdot}
  \frac{\rbinc{N} I_E}{\Sdot} = p_E I_E
                                - Re_{cap}
                                - Re_{\OMd}
                                - p_E I_E Re_{dsub}
                                - Re_{isub} \; .
\end{equation}

The next step is to develop an expression for $Re_{dinc}$
using the direct income elasticity of consumption.


%..............................
\paragraph{Expression for $Re_{dinc}$}
\label{sec:Re_dinc}
%..............................

As shown in Table~\ref{tab:inceffect}, direct income rebound is defined as
%
\begin{equation} \label{eq:Re_dinc_def}
  Re_{dinc} \equiv \frac{\Delta \rainc{E}_s}{\Sdot} \; .
\end{equation}
%
Expanding the difference and rearranging gives
%
\begin{equation}
  Re_{dinc} = \frac{\rainc{E}_s - \rbinc{E}_s}{\Sdot} \; , 
\end{equation}
%
and
%
\begin{equation}
  Re_{dinc} = \frac{\left( \frac{\rainc{E}_s}{\rbinc{E}_s} - 1  \right) \rbinc{E}_s}{\Sdot} \; .
\end{equation}
%
Substituting the typical relationship of Eq.~\ref{eq:typ_qs_eta_Edot} as
$\rainc{E}_s = \frac{\rainc{q}_s}{\aprod{\eta}}$ and  
$\rbinc{E}_s = \frac{\rbinc{q}_s}{\aprod{\eta}}$ gives
%
\begin{equation}
  Re_{dinc} = \frac{\left( \frac{\rainc{q}_s / \cancel{\aprod{\eta}}}{\rbinc{q}_s / \cancel{\aprod{\eta}}} - 1  \right) \rbinc{E}_s} 
              {\Sdot} \; .
\end{equation}
%
Canceling terms and substituting Eqs.~\ref{eq:Sdot} and~\ref{eq:q_ratio_across_inc} gives
%
\begin{equation}
  Re_{dinc} = \frac{\left[ \left( 1 + \frac{\rbinc{N}}{\rbdev{M}} \right) ^{\epsilon_{dinc}} - 1  \right] \rbinc{E}_s} 
              {\Sdoteqn} \; .
\end{equation}
%
Substituting Eq.~\ref{eq:E_dot_s_hat} gives
%
\begin{equation}
  Re_{dinc} = \frac{\left[ \left( 1 + \frac{\rbinc{N}}{\rbdev{M}} \right) ^{\epsilon_{dinc}} - 1  \right] 
                  \frac{\rasub{q}_s}{\rbsub{q}_s}
                \cancel{\frac{\bdev{\eta}}{\aprod{\eta}}}
                \cancel{\rbdev{E}_s}}
              {\left( \frac{\aprod{\eta}}{\bdev{\eta}} - 1 \right)\! \cancel{\frac{\bdev{\eta}}{\aprod{\eta}}} \cancel{\rbdev{E}_s}} \; .
\end{equation}
%
Canceling terms and substituting Eq.~\ref{eq:q_ratio_func_of_eps} gives
%
\begin{equation} \label{eq:Re_dinc}
  Re_{dinc} = \Redinceqn{} \; .
\end{equation}

If there is no net income ($\rasub{N} = 0$), 
direct income effect rebound is zero ($Re_{dinc} = 0$), as expected.
As either of the elasticities get stronger 
($\epsilon_{dinc}$ becomes more positive or $\epsilon_{p_s}$ gets more negative), 
direct income rebound~($Re_{dinc}$) grows.

The next step is to develop an expression for $Re_{iinc}$
using the budget constraint of Eq.~\ref{eq:inc_budget_constraint}.

%..............................
\paragraph{Expression for $Re_{iinc}$}
\label{sec:Re_iinc}
%..............................

In this framework,
any net income not spent on procuring more of the energy service
goes toward other goods in the economy.
Rebound from the indirect income effect involves 
the energy implications of spending net income ($\rbinc{N}$)
on those other goods in the economy.

As shown in Table~\ref{tab:inceffect}, indirect income rebound is defined as
%
\begin{equation}
  Re_{iinc} \equiv \frac{\Delta \rainc{C}_o I_E}{\Sdot} \; .
\end{equation}
%
The increased spending on other goods in the economy ($\Delta \rainc{C}_o$)
can be found by rearranging the budget constraint for the income effect 
(Eq.~\ref{eq:inc_budget_constraint}) to be 
$\Delta \rainc{C}_o = \rbinc{N} - p_E \Delta \rainc{E}_s$.
Substituting gives
%
\begin{equation}
  Re_{iinc} = \frac{\rbinc{N} I_E - p_E I_E \Delta \rainc{E}_s}{\Sdot} \; .
\end{equation}
%
Splitting the terms and substituting Eq.~\ref{eq:Re_dinc_def} gives
%
\begin{equation}
  Re_{iinc} = \frac{\rbinc{N} I_E}{\Sdot} - p_E I_E Re_{dinc} \; .
\end{equation}
%
Substituting Eq.~\ref{eq:N_dot_I_E_Sdot} gives 
%
\begin{equation} \label{eq:Re_iinc_in_terms_of_other_Re}
  Re_{iinc} = \Reiinceqn{} \; .
\end{equation}

This equation shows that a good first estimate for indirect income rebound
is $Re_{iinc} = p_E I_E$.
Positive rebound from capital expenditures~($Re_{cap}$), 
operations, maintenance, and disposal~($Re_{\OMd}$),
direct substitution~($Re_{dsub}$), 
indirect substitution~($Re_{isub}$), and 
direct income~($Re_{dinc}$)
reduce the indirect income effect rebound, 
because they reduce net income available to the device owner to spend on other goods in the economy.
Any negative rebound effects enhance the indirect income effect rebound, because 
they add to the net income available to the device owner to spend on other goods in the economy.


% In reality, some devices (typically, light bulbs) are paid in full at the time of acquisition,
% but others (automobiles) are paid off in installments.
% Since we are interested in energy consumption and not welfare effects, we abstract from discounting.
% By spreading device capital costs and operations, maintenance, and disposal costs across the life of the device,
% we treat all purchase plans equally.




%------------------------------
\subsubsection{\Prodeffect{}} 
\label{sec:Re_prod}
%------------------------------

Productivity rebound~($Re_{prod}$) is given by Eq.~\ref{eq:Re_prod_def}.
Substituting Eq.~\ref{eq:N_dot_I_E_Sdot} gives
%
\begin{equation} \label{eq:Re_prod_in_terms_of_other_Re}
  Re_{prod} = \Reprodeqn{} \; .
\end{equation}


%------------------------------
\subsubsection{Total rebound} 
\label{sec:total_rebound}
%------------------------------

Total rebound is the sum of all rebound effects.
%
\begin{equation}
  Re_{tot} = \cancelto{0}{Re_{dev}} + Re_{lc} + Re_{sub} + Re_{inc} + Re{prod}
\end{equation}
%
Substituting Eqs.~\ref{eq:Re_lc_def}, \ref{eq:Re_sub_def}, \ref{eq:Re_inc_def}
%
\begin{align}
  Re_{tot} = \; &Re_{emb} + Re_{\OMd}      & \mathrm{\lceffect}   \nonumber \\
                &+ Re_{dsub} + Re_{isub}   & \mathrm{\subeffect}  \nonumber \\
                &+ Re_{dinc} + Re_{iinc}   & \mathrm{\inceffect}  \nonumber \\
                &+ Re_{prod}               & \mathrm{\prodeffect}
\end{align}
%
Interestingly, 
indirect income effect rebound~($Re_{iinc}$, Eq.~\ref{eq:Re_iinc_in_terms_of_other_Re}) and
productivity effect rebound~($Re_{prod}$, Eq.~\ref{eq:Re_prod_in_terms_of_other_Re})
are expressed in terms of other rebound effects.
Substituting Eqs.~\ref{eq:Re_iinc_in_terms_of_other_Re} and~\ref{eq:Re_prod_in_terms_of_other_Re} gives
%
\begin{align}
  Re_{tot} = \; &Re_{emb} + Re_{\OMd}      & \mathrm{\lceffect}                             \nonumber \\
                &+ Re_{dsub} + Re_{isub}   & \mathrm{\subeffect}                            \nonumber \\
                &+ Re_{dinc} + p_E I_E - Re_{cap} - Re_{\OMd} - p_E I_E Re_{dsub} 
                             - Re_{isub} - p_E I_E Re_{dinc}   & \mathrm{\inceffect}        \nonumber \\
                &+ k p_E I_E - k Re_{cap} - k Re_{\OMd} - k p_E I_E Re_{dsub} - k Re_{isub}  & \mathrm{\prodeffect}
\end{align}
%
Rearranging distributes many indirect income effect and productivity effect terms 
to other life cycle, substitution, and income effect terms.
This last rearrangement gives the final expression for total rebound.
%
\begin{align} \label{eq:Re_tot}
  Re_{tot} = \; \Retoteqn{}
\end{align}

Eq.~\ref{eq:Re_tot} shows that determining six rebound values,
%
\begin{itemize}

  \item $Re_{emb}$ (Eq.~\ref{eq:Re_emb}), 

  \item $Re_{cap}$ (Eq.~\ref{eq:Re_cap}), 
  
  \item $Re_{\OMd}$ (Eq.~\ref{eq:Re_OMd}),
  
  \item $Re_{dsub}$ (Eq.~\ref{eq:Re_dsub}),
  
  \item $Re_{isub}$ (Eq.~\ref{eq:Re_isub}), and
  
  \item $Re_{dinc}$ (Eq.~\ref{eq:Re_dinc}),

\end{itemize}
%
is sufficient to calculate total rebound, 
provided that 
the productivity factor~($k$),
the price of energy~($p_E$), and
the energy intensity of the economy~($I_E$) 
are known.
