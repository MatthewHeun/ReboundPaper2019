% The next command tells RStudio to do "Compile PDF" on HSB.Rnw,
% instead of this file, thereby eliminating the need to switch back to HSB.Rnw 
% before building the paper.
%!TEX root = ../HSB.Rnw


**** 
We may want to delete (or comment out)
paragraphs of this appendix that are duplicated
in the body of the paper.---MKH
****

This appendix provides a detailed derivation of the comprehensive rebound framework,
beginning with relationships for each rebound effect.


%++++++++++++++++++++++++++++++
\subsection{Relationships for rebound effects}
\label{sec:relationships_for_stages}
%++++++++++++++++++++++++++++++

For each energy rebound effect in Fig.~\ref{fig:flowchart},
energy and financial analysis must be performed.
The purposes of the analyses are to determine for each effect
%
\begin{enumerate*}[label={(\alph*)}]

  \item a definition for energy rebound~($Re$) and

  \item an equation for net income~($\rate{N}$).

\end{enumerate*}

Analysis of each stage involves a set of assumptions and constraints
as shown in Table~\ref{tab:analysis_assumptions}.
In Table~\ref{tab:analysis_assumptions}, 
relationships for \empleffect{} 
embodied energy rates, 
capital expenditure rates, and 
maintenance and disposal expenditure rates
are typical, and
inequalities could switch direction for a specific EEU.
\Prodeffect{} relationships are given for a single device only.
If the EEU is deployed at scale across the economy, 
the energy service consumption rate~($\rate{q}_s$), 
device energy consumption rate~($\rate{E}_s$), 
embodied energy rate~($\rate{E}_{emb}$),
capital expenditure rate~($\rate{C}_{cap}$), and 
maintenance and disposal expenditure rate~($\rate{C}_{\OMd}$)
will all increase.

% The next command tells RStudio to do "Compile PDF" on HSB.Rnw,
% instead of this file, thereby eliminating the need to switch back to HSB.Rnw 
% before building the paper.
%!TEX root = ../HSB.Rnw

\begin{landscape}

\begin{table}
\centering
\caption{Assumptions and constraints for analysis of rebound effects.}
\label{tab:analysis_assumptions}

\begin{tabular}{r c c c c c}
\toprule
% Parameter & \DevEffect{} & \LcEffect & \SubEffect & \IncEffect & \MacroEffect \\
Parameter & \EmplEffect{} & \SubEffect & \IncEffect & \MacroEffect \\
\midrule
Energy price                     & $\bempl{p}_E  = \aempl{p}_E$         
                                 & $\bsub{p}_E   = \asub{p}_E$ 
                                 & $\binc{p}_E   = \ainc{p}_E$ 
                                 & $\bprod{p}_E  = \aprod{p}_E$ \\
%
Energy service efficiency        & $\bempl{\eta}  < \aempl{\eta}$         
                                 & $\bsub{\eta}   = \asub{\eta}$ 
                                 & $\binc{\eta}   = \ainc{\eta}$ 
                                 & $\bprod{\eta}  = \aprod{\eta}$ \\
%
Energy service price             & $\bempl{p}_s  > \aempl{p}_s$          
                                 & $\bsub{p}_s   = \asub{p}_s$ 
                                 & $\binc{p}_s   = \ainc{p}_s$  
                                 & $\bprod{p}_s  = \aprod{p}_s$ \\
%
Other goods price                & $\bempl{p}_o  = \aempl{p}_o$          
                                 & $\bsub{p}_o   = \asub{p}_o$ 
                                 & $\binc{p}_o   = \ainc{p}_o$  
                                 & $\bprod{p}_o  = \aprod{p}_o$ \\
%
Energy service consumption rate  & $\rbempl{q}_s  = \raempl{q}_s$         
                                 & $\rbsub{q}_s   < \rasub{q}_s$ 
                                 & $\rbinc{q}_s   < \rainc{q}_s$ 
                                 & $\rbprod{q}_s  = \raprod{q}_s$ \\
%
Other goods consumption rate     & $\rbempl{q}_o  = \raempl{q}_o$         
                                 & $\rbsub{q}_o   > \rasub{q}_o$ 
                                 & $\rbinc{q}_o   < \rainc{q}_o$ 
                                 & $\rbprod{q}_o  = \raprod{q}_o$ \\
%
Device energy consumption rate   & $\rbempl{E}_s  > \raempl{E}_s$
                                 & $\rbsub{E}_s   < \rasub{E}_s$ 
                                 & $\rbinc{E}_s   < \rainc{E}_s$ 
                                 & $\rbprod{E}_s  = \raprod{E}_s$ \\
%
Embodied energy rate             & $\rbempl{E}_{emb}  < \raempl{E}_{emb}$ 
                                 & $\rbsub{E}_{emb}   = \rasub{E}_{emb}$ 
                                 & $\rbinc{E}_{emb}   = \rainc{E}_{emb}$ 
                                 & $\rbprod{E}_{emb}  = \raprod{E}_{emb}$ \\
%
Capital expenditure rate         & $\rbempl{C}_{cap}  < \raempl{C}_{cap}$ 
                                 & $\rbsub{C}_{cap}   = \rasub{C}_{cap}$ 
                                 & $\rbinc{C}_{cap}   = \rainc{C}_{cap}$ 
                                 & $\rbprod{C}_{cap}  = \raprod{C}_{cap}$ \\
%
Maint.\ and disp.\ expenditure rate & $\rbempl{C}_{\md}  < \raempl{C}_{\md}$ 
                                 & $\rbsub{C}_{\md}      = \rasub{C}_{\md}$ 
                                 & $\rbinc{C}_{\md}      = \rainc{C}_{\md}$ 
                                 & $\rbprod{C}_{\md}     = \raprod{C}_{\md}$ \\
%
Energy service expenditure rate  & $\rbempl{C}_s  > \raempl{C}_s$
                                 & $\rbsub{C}_s   < \rasub{C}_s$ 
                                 & $\rbinc{C}_s   < \rainc{C}_s$ 
                                 & $\rbprod{C}_s  = \raprod{C}_s$ \\
%
Other goods expenditure rate     & $\rbempl{C}_o  = \raempl{C}_o$         
                                 & $\rbsub{C}_o   > \rasub{C}_o$ 
                                 & $\rbinc{C}_o   < \rainc{C}_o$ 
                                 & $\rbprod{C}_o  = \raprod{C}_o$ \\
%
Income                           & $\rbempl{M} = \raempl{M}$         
                                 & $\rbsub{M}  = \rasub{M}$ 
                                 & $\rbinc{M}  = \rainc{M}$ 
                                 & $\rbprod{M} = \raprod{M}$  \\
%
Net income (freed cash)          & 0 = $\rbempl{N} <   \raempl{N}$         
                                 & $\rbsub{N}      \ne \rasub{N}$ 
                                 & $\rbinc{N}      >   \rainc{N} = 0$ 
                                 & $\rbprod{N}     =   \raprod{N} = 0$  \\
\bottomrule
\end{tabular}


\end{table}

\end{landscape}



%++++++++++++++++++++++++++++++
\subsection{Derivations}
\label{sec:derivations}
%++++++++++++++++++++++++++++++

Derivations for rebound definitions and net income equations
are presented in Tables~\ref{tab:empleffect}--\ref{tab:prodeffect},
one for each stage in Fig.~\ref{fig:flowchart}.
Energy and financial analyses are shown side by side, because
each informs the other.

% Derivation tables

% The next command tells RStudio to do "Compile PDF" on HSB.Rnw,
% instead of this file, thereby eliminating the need to switch back to HSB.Rnw 
% before building the paper.
%!TEX root = ../HSB.Rnw

% This file contains the derivation for emplacement effect rebound terms and net income.

\begin{landscape}

\linespread{1}

%%%%%%%%%%%%%%%%%%%%%%%%%%%%%%%%%%%%%%%%
%%%%%%%%%% Emplacement Effect %%%%%%%%%%
%%%%%%%%%%%%%%%%%%%%%%%%%%%%%%%%%%%%%%%%
\derivheader{\refstepcounter{table} Table~\thetable \label{tab:empleffect}. \bf{\EmplEffect}}

\sectionsep{}

%%%%%%%%%% Before Emplacement Effect %%%%%%%%%%
\derivsection{before}
{
% Original energy
\begin{equation} \label{eq:E_acct_orig}
  \Eacctorig{}
\end{equation}
}
{
% Original financial
\begin{equation} \label{eq:M_acct_orig}
  \Macctorig{}
\end{equation}
}

\sectionsep{}

%%%%%%%%%% After Emplacement Effect %%%%%%%%%%
\derivsection{after ($\aempl{ }$)}
{
% After device energy
\begin{equation} \label{eq:E_acct_aemp}
  \Eacctaempl{}
\end{equation}
}
{
% After device financial
\begin{equation} \label{eq:M_acct_aemp}
  \Macctaempl{}
\end{equation}
}

\sectionsep{}

%%%%%%%%%% Derivations for Emplacement Effect %%%%%%%%%%
\derivsection{}
% Emplacement effect: energy differences
{
%%%%%%%%%% Energy Emplacement Effect %%%%%%%%%%
~
  
Take differences to obtain the change in energy consumption, $\Delta \raempl{E} \equiv \raempl{E} - \rbempl{E}$.
%
\begin{equation}
  \Delta \raempl{E} = \Delta \raempl{E}_s
                      + \Delta \raempl{E}_{emb}
                      + (\Delta \raempl{C}_{\OMd}
                      + \cancelto{0}{\Delta \raempl{C}_o}) I_E
\end{equation}
%
Thus, 
%
\begin{equation}
\Delta \raempl{E} = \Delta \raempl{E}_s + \Delta \raempl{E}_{emb} + \Delta \raempl{C}_{\OMd} I_E \; .
\end{equation}
%
Use Eq.~(\ref{eq:Re_def}) and define
%
\begin{equation} \label{eq:Sdot_def}
\Sdot \equiv -\Delta \raempl{E}_s
\end{equation}
%
to obtain
%
\begin{equation}
Re_{empl} = 1 - \frac{-\Delta \raempl{E}}{\Sdot} 
          = 1 - \frac{-\Delta \raempl{E}_s}{\Sdot} 
              - \frac{-\Delta \raempl{E}_{emb}}{\Sdot}
              - \frac{-\Delta \raempl{C}_{\OMd} I_E}{\Sdot} \; .
\end{equation}
%
Define $Re_{dempl} \equiv 1 - \frac{-\Delta \raempl{E}_s}{\Sdot} = 0$, 
$Re_{iempl} \equiv Re_{emb} + Re_{\OMd}$, 
$Re_{emb} \equiv \frac{\Delta \raempl{E}_{emb}}{\Sdot}$, and
$Re_{\OMd} \equiv \frac{\Delta \raempl{C}_{\OMd} I_E}{\Sdot}$, 
such that
%
\begin{equation} \label{eq:Re_empl_def}
Re_{empl} = Re_{dempl} + Re_{iempl} \; .
\end{equation}
}
{
%%%%%%%%%% Financial Emplacement Effect %%%%%%%%%%
~
    
Use the monetary constraint ($\rbempl{M} = \raempl{M}$)
and constant spending on other items ($\rbempl{C}_o = \raempl{C}_o$) to cancel terms to obtain
%
\begin{align}
  p_E \rbempl{E}_s &+ \rbempl{C}_{cap} + \rbempl{C}_{\OMd} + \cancel{\rbempl{C}_o} + \cancelto{0}{\rbempl{N}} \nonumber \\
                   &= p_E \raempl{E}_s + \raempl{C}_{cap} + \raempl{C}_{\OMd} + \cancel{\raempl{C}_o}  + \raempl{N} \; .
\end{align}
%
Solving for $\Delta \raempl{N} \equiv \raempl{N} - \cancelto{0}{\rbempl{N}}$ gives 
%
\begin{equation}
  \Delta \raempl{N} = p_E(\rbempl{E}_s - \raempl{E}_s) 
                      + \rbempl{C}_{cap} - \raempl{C}_{cap}
                      + \rbempl{C}_{\OMd} - \raempl{C}_{\OMd} \; .
\end{equation}
%
Rewriting with $\Delta$ terms gives
%
\begin{equation}
  \Delta \raempl{N} = - p_E \Delta \raempl{E}_s - \Delta \raempl{C}_{cap} - \Delta \raempl{C}_{\OMd} \; .
\end{equation}
%
Substituting Eq.~(\ref{eq:Sdot_def}) gives
%
\begin{equation}
  \Delta \raempl{N} = \raempl{N} = p_E \Sdot - \Delta \raempl{C}_{cap} - \Delta \raempl{C}_{\OMd} \; .
\end{equation}
%
Gross monetary savings ($\rate{G}$) resulting from the EEU, 
before any energy takeback, is given by 
%
\begin{equation} \label{eq:G_dot}
  \rate{G} = p_E \Sdot \; .
\end{equation}

Note that Eq.~(\ref{eq:M_acct_orig}) and $\rbempl{N} = 0$ can be used to calculate $\rbempl{C}_o$ as
%
\begin{equation} \label{eq:C_dot_o}
  \rbempl{C}_o = \rate{M} - p_E \rbempl{E}_s - \rbempl{C}_{cap} - \rbempl{C}_{\OMd} \; .
\end{equation}
%

}
\end{landscape}



% The next command tells RStudio to do "Compile PDF" on HSB.Rnw,
% instead of this file, thereby eliminating the need to switch back to HSB.Rnw 
% before building the paper.
%!TEX root = ../HSB.Rnw

% This file contains the derivation for substitution effect rebound and net income.

\begin{landscape}

\linespread{1}

%%%%%%%%%%%%%%%%%%%%%%%%%%%%%%%%%%%%%%%%%
%%%%%%%%%% Substitution Effect %%%%%%%%%%
%%%%%%%%%%%%%%%%%%%%%%%%%%%%%%%%%%%%%%%%%
\derivheader{\refstepcounter{table} Table~\thetable \label{tab:subeffect}. \bf{\SubEffect}}

\sectionsep{}

%%%%%%%%%% Before Substitution Effect %%%%%%%%%%
\derivsection{before ($\bsub{})$}
{
% Before substitution energy
\begin{equation}
  \Eacctbsub{}
\end{equation}
}
{
% Before substitution financial
\begin{equation}
  \Macctbsub{}
\end{equation}
}

\sectionsep{}


%%%%%%%%%% After Substitution Effect %%%%%%%%%%
\derivsection{after ($\asub{~})$}
{
% After substitution energy
\begin{equation}
  \Eacctasub{}
\end{equation}
}
{
% After substitution financial
\begin{equation}
  \Macctasub{}
\end{equation}
}

\sectionsep{}

%%%%%%%%%% Derivations for Substitution Effect %%%%%%%%%%
\derivsection{}
% Substitution effect: energy differences
{
%%%%%%%%%% Energy Substitution Effect %%%%%%%%%%
~
  
Take differences to obtain the change in energy consumption, $\Delta \rasub{E} \equiv \rasub{E} - \rbsub{E}$.
%
\begin{equation}
  \Delta \rasub{E} = \Delta \rasub{E}_s 
                      + \cancelto{0}{\Delta \rasub{E}_{emb}} 
                      + (\cancelto{0}{\Delta \rasub{C}_{\OMd}} + \Delta \rasub{C}_o) I_E
\end{equation}
%
Thus, 
%
\begin{equation}
  \Delta \rasub{E} = \Delta \rasub{E}_s + \Delta \rasub{C}_o I_E \; .
\end{equation}
%
All terms are energy takeback terms.
Divide by $\Sdot$
to create rebound terms.
%
\begin{equation}
    \frac{\Delta \rasub{E}}{\Sdot} = \frac{\Delta \rasub{E}_s}{\Sdot} + \frac{\Delta \rasub{C}_o I_E}{\Sdot}
\end{equation}
%
Define 
$Re_{sub} \equiv \frac{\Delta \rasub{E}}{\Sdot}$, 
$Re_{dsub} \equiv \frac{\Delta \rasub{E}_s}{\Sdot}$, and
$Re_{isub} \equiv \frac{\Delta \rasub{C}_o I_E}{\Sdot}$,
such that
%
\begin{equation} \label{eq:Re_sub_def}
  Re_{sub} = Re_{dsub} + Re_{isub} \; .
\end{equation}

}
{
%%%%%%%%%% Financial Substitution Effect %%%%%%%%%%
~
  
Use the monetary constraint ($\rbsub{M} = \rasub{M}$) to obtain
%
\begin{align}
  p_E \raempl{E}_s &+ \cancel{\raempl{C}_{cap}} + \cancel{\raempl{C}_{\OMd}} + \raempl{C}_o + \raempl{N} \nonumber \\
                   &= p_E \rasub{E}_s + \cancel{\rasub{C}_{cap}} + \cancel{\rasub{C}_{\OMd}} + \rasub{C}_o + \rasub{N} \; .
\end{align}
%
For the substitution effect, there is no change in capital or operations, maintenance, and disposal costs
($\rasub{C}_{cap} = \rbsub{C}_{cap}$ and $\rasub{C}_{\OMd} = \rbsub{C}_{\OMd}$).
Solving for $\Delta \rasub{N} \equiv \rasub{N} - \rbsub{N}$ gives
%
\begin{equation}
  \Delta \rasub{N} = - p_E \Delta \rasub{E}_s - \Delta \rasub{C}_o \; .
\end{equation}
%
With $\rasub{N} = \rbsub{N} + \Delta \rasub{N}$, 
we find that
%
\begin{equation} \label{eq:N_dot_after_sub}
  \rasub{N} = \rate{G} - \Delta \rbsub{C}_{cap} - \Delta \rbsub{C}_{\OMd} - p_E \Delta \rasub{E}_s - \Delta \rasub{C}_o \; .
\end{equation}
%
}

\end{landscape}



% The next command tells RStudio to do "Compile PDF" on HSB.Rnw,
% instead of this file, thereby eliminating the need to switch back to HSB.Rnw 
% before building the paper.
%!TEX root = ../HSB.Rnw

% This file contains the derivation for income effect rebound and net income.

\begin{landscape}

\linespread{1}

%%%%%%%%%%%%%%%%%%%%%%%%%%%%%%%%%%%
%%%%%%%%%% Income Effect %%%%%%%%%%
%%%%%%%%%%%%%%%%%%%%%%%%%%%%%%%%%%%
\derivheader{\refstepcounter{table} Table~\thetable \label{tab:inceffect}. \bf{\IncEffect}}

\sectionsep{}

%%%%%%%%%% Before Income Effect %%%%%%%%%%
\derivsection{before ($\binc{~})$}
{
% Before income energy
\begin{equation}
\Eacctbinc{}
\end{equation}
}
{
% Before income financial
\begin{equation}
\Macctbinc{}
\end{equation}
}

\sectionsep{}

%%%%%%%%%% After Income Effect %%%%%%%%%%
\derivsection{after ($\ainc{~})$}
{
% After income energy
\begin{equation}
\Eacctainc{}
\end{equation}
}
{
% After income financial
\begin{equation}
\Macctainc{}
\end{equation}
}

\sectionsep{}

%%%%%%%%%%% Derivations for Income Effect %%%%%%%%%%
\derivsection{}
% Income effect: energy differences
{
%%%%%%%%%% Energy Income Effect %%%%%%%%%%
~

Take differences to obtain the change in energy consumption, $\Delta \rainc{E} \equiv \rainc{E} - \rbinc{E}$.
%
\begin{equation}
  \Delta \rainc{E} = \Delta \rainc{E}_s 
                     + \cancelto{0}{\Delta \rainc{E}_{emb}}
                     + (\cancelto{0}{\Delta \rainc{C}_{\OMd}} + \Delta \rainc{C}_o) I_E
\end{equation}
%
Thus, 
%
\begin{equation}
  \Delta \rainc{E} = \Delta \rainc{E}_s + \Delta \rainc{C}_o I_E
\end{equation}
%
All terms are energy takeback terms.
Divide by $\Sdot$
to create rebound terms.
%
\begin{equation}
  \frac{\Delta \rainc{E}}{\Sdot} = \frac{\Delta \rainc{E}_s}{\Sdot} + \frac{\Delta \rainc{C}_o I_E}{\Sdot}
\end{equation}
%
Define 
$Re_{inc} \equiv \frac{\Delta \rainc{E}}{\Sdot}$, 
$Re_{dinc} \equiv \frac{\Delta \rainc{E}_s}{\Sdot}$, and 
$Re_{iinc} \equiv \frac{\Delta \rainc{C}_o I_E}{\Sdot}$,
such that
%
\begin{equation} \label{eq:Re_inc_def}
  Re_{inc} = Re_{dinc} + Re_{iinc} \; .
\end{equation}
%
}
{
%%%%%%%%%% Financial Income Effect %%%%%%%%%%
~

Use the monetary constraint ($\rbinc{M} = \rainc{M}$) to obtain
%
\begin{align}
  p_E \rasub{E}_s &+ \cancel{\rasub{C}_{cap}} + \cancel{\rasub{C}_{\OMd}} + \rasub{C}_o + \rasub{N} \nonumber \\
                  &= p_E \rainc{E}_s + \cancel{\rainc{C}_{cap}} + \cancel{\rainc{C}_{\OMd}} + \rainc{C}_o + \cancelto{0}{\rainc{N}} \; .
\end{align}
%
For the income effect, there is no change in capital or maintainance and disposal costs
($\rasub{C}_{cap} = \rbsub{C}_{cap}$ and $\rasub{C}_{\OMd} = \rbsub{C}_{\OMd}$).
Notably, $\rainc{N} = 0$,
because it is assumed that all net monetary savings ($\rbinc{N}$) are spent on
more energy service ($\rainc{E}_s > \rbinc{E}_s$)
and
additional purchases in the economy ($\rainc{C}_o > \rbinc{C}_o$).
Solving for $\rbinc{N}$ gives 
%
\begin{equation} \label{eq:inc_budget_constraint}
  \rbinc{N} = p_E \Delta \rainc{E}_s + \Delta \rainc{C}_o \; ,
\end{equation}
%
the budget constraint for the income effect.
Eq.~(\ref{eq:inc_budget_constraint}) ensures
spending of net income~($\rbinc{N}$) on
%
\begin{enumerate*}[label={(\alph*)}]
	
  \item additional energy services~($\Delta \rainc{E}_s$) and
  
  \item additional purchases of other goods in the economy~($\Delta \rainc{C}_o$) only.
    
\end{enumerate*}
}
\end{landscape}



% The next command tells RStudio to do "Compile PDF" on book.Rnw,
% instead of this chapter, thereby eliminating the need to switch back to book.Rnw 
% before making the book.
%!TEX root = ../HSB.Rnw

% This file contains the derivation for income effect rebound and net income.

\begin{landscape}

\linespread{1}

%%%%%%%%%%%%%%%%%%%%%%%%%%%%%%%%%%%%%%%%%
%%%%%%%%%% Productivity Effect %%%%%%%%%%
%%%%%%%%%%%%%%%%%%%%%%%%%%%%%%%%%%%%%%%%%
\derivheader{\refstepcounter{table} Table~\thetable \label{tab:prodeffect}. \bf{\ProdEffect}}

\sectionsep{}

%%%%%%%%%% Before Productivity Effect %%%%%%%%%%
\derivsection{before ($\bprod{~})$}
{
% Before productivity energy
\begin{equation}
  \rbprod{E}
\end{equation}
}
{
% Before productivity financial
}

\sectionsep{}

%%%%%%%%%% After Productivity Effect %%%%%%%%%%
\derivsection{after ($\aprod{~})$}
{
% After productivity energy
\begin{equation}
\raprod{E}
\end{equation}
}
{
% After productivity financial
}

\sectionsep{}

%%%%%%%%%%% Derivations for productivty Effect %%%%%%%%%%
\derivsection{}
% Productivity effect: energy differences
{
%%%%%%%%%% Energy Productivity Effect %%%%%%%%%%
~

Take differences to obtain the change in energy consumption,
%
\begin{equation}
  \Delta \raprod{E} \equiv \raprod{E} - \rbprod{E} \; .
\end{equation}
%
The energy change due to the productivity effect ($\Delta \raprod{E}$) 
is a scalar multiple ($k$) of net income ($\rbinc{N}$), 
assumed to be spent at the energy intensity of the economy ($I_E$).
%
\begin{equation}
  \Delta \raprod{E} = k \rbinc{N} I_E
\end{equation}
%
All terms are energy takeback terms.
Divide by $\Sdot$
to create rebound terms.
%
\begin{equation}
  \frac{\Delta \raprod{E}}{\Sdot} = \frac{k \rbinc{N} I_E}{\Sdot}
\end{equation}
%
Define 
$Re_{prod} \equiv \frac{\Delta \raprod{E}}{\Sdot}$, 
such that
%
\begin{equation}
  Re_{prod} = \frac{k \rbinc{N} I_E}{\Sdot} \; . \tag{\ref{eq:Re_prod_def}}
\end{equation}
%
}
{
%%%%%%%%%% Financial Productivity Effect %%%%%%%%%%
~
\centering

N/A
}
\end{landscape}



%++++++++++++++++++++++++++++++
\subsection{Rebound expressions}
\label{sec:rebound_expressions}
%++++++++++++++++++++++++++++++

All that remains is to determine expressions for each rebound term, 
beginning with the expected energy savings rate~($\Sdot$), which
appears in the denominator of all rebound expressions.


%------------------------------
\subsubsection{Expected energy savings ($\Sdot$)} 
\label{sec:Sdot}
%------------------------------

$\Sdot$ is the reduction of energy consumption rate
by the device due to the EEU.
No other effects are considered.

\begin{equation}
  \Sdot \equiv \rbempl{E}_s - \raempl{E}_s  \tag{\ref{eq:S_dot_def}}
\end{equation}
%
The final energy consumption rates ($\rbempl{E}_s$ and $\raempl{E}_s$) 
can be written as Eq.~(\ref{eq:typ_qs_eta_Edot}) in the forms
$\rbempl{E}_s = \frac{\rbempl{q}_s}{\bempl{\eta}}$ and 
$\raempl{E}_s = \frac{\raempl{q}_s}{\aempl{\eta}}$. 

\begin{equation}
  \Sdot = \frac{\rbempl{q}_s}{\bempl{\eta}} - \frac{\raempl{q}_s}{\aempl{\eta}}
\end{equation}
%
With reference to Table~\ref{tab:analysis_assumptions}, 
we use $\raempl{q}_s = \rbempl{q}_s$ and $\aempl{\eta} = \aprod{\eta}$ to obtain

\begin{equation}
  \Sdot = \frac{\rbempl{q}_s}{\bempl{\eta}} - \frac{\rbempl{q}_s}{\aprod{\eta}} \; .
\end{equation}
%
When the EEU increases efficiency such that $\aprod{\eta} > \bempl{\eta}$,
expected energy savings grows ($\Sdot > 0$)
as the rate of final energy consumption declines,
as expected.
As $\aprod{\eta} \rightarrow \infty$,
all final energy consumption is eliminated ($\raempl{E}_s \rightarrow 0$), and
$\Sdot = \frac{\rbempl{q}_s}{\bempl{\eta}} = \rbempl{E}_s$.
(Of course, $\aprod{\eta} \rightarrow \infty$ is impossible. 
See \citet{Paoli:2020aa} for a recent discussion of upper limits to device efficiencies.)

After rearrangement and using $\rbempl{E}_s = \frac{\rbempl{q}_s}{\bempl{\eta}}$, 
we obtain a convenient form

\begin{equation}
  \Sdot = \Sdoteqn \; .  \tag{\ref{eq:Sdot}}
\end{equation}


%------------------------------
\subsubsection{\Empleffect{}}
\label{sec:Re_emp}
%------------------------------

The emplacement effect accounts for performance of the EEU only.
No behavior changes occur.
The direct emplacement effect of the EEU is device energy savings and energy cost savings.
The indirect emplacement effects of the EEU produce changes in the embodied energy rate and
the maintenance and disposal cost rates.
By definition, the direct emplacement effect has no rebound. 
However, indirect emplacement effects may cause energy rebound.
Both direct and indirect emplacement effects are discussed below.


%..............................
\paragraph{$Re_{dempl}$}
\label{sec:Re_dempl}
%..............................

As shown in Table~\ref{tab:empleffect},
the direct rebound from the emplacement effect is
$Re_{dempl} = 0$.
This result is expected, 
because, in the absence of behavior changes,
there is no takeback of energy savings
at the upgraded device.


%..............................
\paragraph{$Re_{iempl}$} 
\label{sec:Re_iempl}
%..............................

Indirect emplacement rebound effects 
can occur at any point in the life cycle of an energy conversion device,
from manufacturing and distribution 
to the use phase (maintenance),
and finally to disposal.
For simplicity, we group maintenance with disposal to form
two distinct indirect emplacement rebound effects:
%
\begin{enumerate*}[label={(\alph*)}]
	
  \item an embodied energy effect ($Re_{emb}$) and 
  
  \item a maintenance and disposal effect ($Re_{\OMd}$).
    
\end{enumerate*}


%..............................
\paragraph{$Re_{emb}$}
\label{sec:Re_emb}
%..............................

The first component of indirect emplacement effect rebound
involves embodied energy.
We define embodied energy consistent with the energy/exergy analysis literature
to be the sum of all final energy consumed
in the production of the energy conversion device.
The EEU
causes the embodied final energy of the device to change
from $\rbempl{E}_{emb}$ to $\raempl{E}_{emb}$.

Energy is embodied in the device within manufacturing and distribution supply chains
prior to consumer acquisition of the device.
No energy is embodied in the device while in service.
However, for simplicity, we spread all embodied energy
over the lifetime of the device,
an equal amount assigned to each period.
We later take the same approach to capital costs and
maintenance and disposal costs.
A justification for spreading embodied energy purchase costs comes from considering
staggered device replacements by many consumers across several years.
In the aggregate, staggered replacements
work out to about the same embodied energy in every period.

Thus, we allocate embodied energy over the life of the original and upgraded devices
($\bempl{t}$ and $\aempl{t}$, respectively)
to obtain embodied energy rates, such that
$\rbempl{E}_{emb} = \bempl{E}_{emb} / \bempl{t}$
and 
$\raempl{E}_{emb} = \aempl{E}_{emb} / \aempl{t}$.
The change in embodied final energy due to the EEU (expressed as a rate) is given by
$\raempl{E}_{emb} - \rbempl{E}_{emb}$.
After substitution and algebraic rearrangement,
the change in embodied energy rate due to the EEU can be expressed as
$\left( \frac{\aempl{E}_{emb}}{\bempl{E}_{emb}}
  \frac{\bempl{t}}{\aempl{t}} - 1 \right) \rbempl{E}_{emb}$, 
a term that represents energy savings taken back due to embodied energy effects.
Thus, Eq.~(\ref{eq:Re_takeback}) can be employed to write embodied energy rebound as
%
\begin{equation} 
  Re_{emb} = \Reembeqn{} \, . \tag{\ref{eq:Re_emb}}
\end{equation}

Embodied energy rebound can be either positive or negative, depending on 
the sign of the term
$(\aempl{E}_{emb}/\bempl{E}_{emb})(\bempl{t}/\aempl{t}) - 1$.
Typically, but not always,
rising energy efficiency is associated with increased device complexity
and more embodied energy,
such that $\aempl{E}_{emb} > \bempl{E}_{emb}$ and $Re_{emb} > 0$.
However, if the upgraded device has longer life than the original device
($\aempl{t} > \bempl{t}$),
$\raempl{E}_{emb} - \rbempl{E}_{emb}$ can be negative,
meaning that the upgraded device has a lower embodied energy rate than the original device.


%..............................
\paragraph{$Re_{\OMd}$} 
\label{sec:Re_OMd}
%..............................

In addition to embodied energy effects, 
indirect emplacement rebound 
can be associated with energy demanded by maintenance and disposal~($\OMd$) expenditures.
Maintenance expenditures are typically modeled as a per-year expense, 
a rate (e.g., $\rbempl{C}_m$).
Disposal costs (e.g., $\bempl{C}_d$) are one-time expenses incurred at the end of the useful life of the energy conversion device.
Like embodied energy, we spread disposal costs across the lifetime 
of the original and upgraded devices ($\bempl{t}$ and $\aempl{t}$, respectively)
to form cost rates such that $\rbempl{C}_{\OMd} = \rbempl{C}_{\OM} + \bempl{C}_d/\bempl{t}$
and
$\raempl{C}_{\OMd} = \raempl{C}_{\OM} + \aempl{C}_d/\aempl{t}$.

We assume, for simplicity, that $\OMd$ expenditures indicate energy consumption
elsewhere in the economy at its energy intensity~($I_E$).
Therefore, the change in energy consumption rate caused by a change in $\OMd$ expenditures
is given by $\Delta \raempl{C}_{\OMd} I_E$.
This term represents energy takeback, so maintenance and disposal rebound is given by

\begin{equation} \label{eq:Re_OMd_def}
  Re_{\OMd} = \frac{\Delta \raempl{C}_{\OMd} I_E}{\Sdot} \; ,
\end{equation}
%
as shown in Table~\ref{tab:empleffect}.
Slight rearrangement gives

\begin{equation}
  Re_{\OMd} = \ReOMdeqn{} \; . \tag{\ref{eq:Re_OMd}}
\end{equation}

Rebound from maintenance and disposal can be positive or negative,
depending on the sign of the term $\raempl{C}_{\OMd}/\rbempl{C}_{\OMd} - 1$.


%------------------------------
\subsubsection{\Subeffect{}} 
\label{sec:Re_sub}
%------------------------------

Two terms comprise substitution effect rebound,
direct substitution rebound ($Re_{dsub}$) and
indirect substitution rebound ($Re_{isub}$).
This section derives each in terms of 
energy service efficiencies ($\bempl{\eta}$ and $\aprod{\eta}$) and
the service price elasticities 
of energy service consumption ($\eqsps$) and
other goods consumption ($\eqops$).
We begin with derivation of two ratios that are helpful later,
$\frac{\rasub{q}_s}{\rbsub{q}_s}$ and
$\frac{\rasub{q}_o}{\rbsub{q}_o}$.


%..............................
\paragraph{Expressions for two ratios, $\frac{\rasub{q}_s}{\rbsub{q}_s}$ and $\frac{\rasub{q}_o}{\rbsub{q}_o}$}
\label{sec:two_ratios}
%..............................

The EEU's energy efficiency increase
($\aprod{\eta} > \bempl{\eta}$)
causes the price of the energy service provided by the device to fall
($\aprod{p}_s < \bempl{p}_s$).
The substitution effect quantifies the amount by which
the device owner, in response,
increases the consumption rate of the energy service ($\rasub{q}_s > \rbsub{q}_s$) and
decreases the consumption rate of other goods ($\rasub{q}_o < \rbsub{q}_o$).
(See Appendix~\ref{sec:elasticities} for information about elasticities.)

The relationship between energy service price and energy service consumption rate
is given by the service price elasticity of energy service consumption~($\eqsps$),
such that

\begin{equation}
  \frac{\rasub{q}_s}{\rbsub{q}_s} = \left( \frac{\aprod{p}_s}{\bempl{p}_s} \right)^{\eqsps} \; .
\end{equation}
%
Note that a negative value for the service price elasticity of energy service consumption
is expected ($\eqsps < 0$),
such that when the energy service price decreases ($\aprod{p}_s < \bempl{p}_s$),
the rate of energy service consumption increases ($\rasub{q}_s > \rbsub{q}_s$).

Substituting Eq.~(\ref{eq:ps_pE_eta}) in the form
$\bempl{p}_s = \frac{\bempl{p}_E}{\bempl{\eta}}$ and
$\aprod{p}_s = \frac{\bempl{p}_E}{\aprod{\eta}}$
gives

\begin{equation}
  \frac{\rasub{q}_s}{\rbsub{q}_s} = \left( \frac{\aprod{\eta}}{\bempl{\eta}} \right)^{-\eqsps} \; .
                                                                        \tag{\ref{eq:q_ratio_func_of_eps}}
\end{equation}
%
Again, note that the energy service price elasticity of energy service consumption
is negative ($\eqsps < 0$), so that
as energy service efficiency increases ($\aprod{\eta} > \bempl{\eta}$),
the energy service consumption rate increases ($\rasub{q}_s > \rbsub{q}_s$).

To quantify the substitution effect on other purchases,
we introduce a second elasticity, 
the service price elasticity of other goods consumption~($\eqops$).
See Appendix~\ref{sec:elasticities} for constraints on the relationship between the
energy service price elasticity of energy service consumption~($\eqsps$) and the
energy service price elasticity of other goods consumption~($\eqops$).

The ratio of other goods consumption on both sides of the substitution effect is given by

\begin{equation}
  \frac{\rasub{q}_o}{\rbsub{q}_o} = \left( \frac{\aprod{p}_s}{\bempl{p}_s} \right) ^ {\eqops} \; . 
\end{equation}
%
Note that the service price elasticity of other good consumption
is expected to be positive ($\eqops > 0$).
As the energy service price decreases~($\aprod{p}_s < \bempl{p_s}$), 
consumption of other goods is expected to decrease ($\rasub{q}_o < \rbsub{q}_o$), 
because constant utility is assumed across the substitution effect.
(See Appendix~\ref{sec:elasticities}.)
Substituting Eq.~(\ref{eq:ps_pE_eta}) in the form
$\bempl{p}_s = \frac{\bempl{p}_E}{\bempl{\eta}}$ and
$\aprod{p}_s = \frac{\bempl{p}_E}{\aprod{\eta}}$
gives

\begin{equation}
  \frac{\rasub{q}_o}{\rbsub{q}_o} = \left( \frac{\aprod{\eta}}{\bempl{\eta}} \right) ^ {-\eqops} \; . 
                                                                                              \tag{\ref{eq:qohat_qostar}}
\end{equation}

Next, we derive an expression for direct substitution rebound ($Re_{dsub}$).


%..............................
\paragraph{Expression for $Re_{dsub}$}
\label{sec:Re_dsub}
%..............................

As shown in Table~\ref{tab:subeffect}, direct substitution rebound is defined as

\begin{equation}
  Re_{dsub} \equiv \frac{\Delta \rasub{E}_s}{\Sdot} \; . \tag{\ref{eq:Re_dsub_def}}
\end{equation}
%
Expansion of the difference term and
substitution of Eq.~(\ref{eq:typ_qs_eta_Edot}) in the form
$\rasub{E}_s = \frac{\rasub{q}_s}{\aprod{\eta}}$
and
$\rbsub{E}_s = \frac{\rbsub{q}_s}{\bempl{\eta}}$
gives

\begin{equation}
   Re_{dsub} = \frac{\rasub{E}_s - \rbsub{E}_s}{\Sdot} \; ,
\end{equation}
%
and

\begin{equation}
     Re_{dsub} = \frac{\frac{\rasub{q}_s}{\aprod{\eta}} - \frac{\rbsub{q}_s}{\bempl{\eta}}}{\Sdot} \; .
\end{equation}
%
Substitution of Eq.~(\ref{eq:Sdot}) gives

\begin{equation}
  Re_{dsub} = \frac{\frac{\rasub{q}_s}{\aprod{\eta}} - \frac{\rbsub{q}_s}{\bempl{\eta}}}
              {\Sdoteqn} \; .
\end{equation}
%
Rearrangement of the numerator and canceling terms gives

\begin{equation}
  Re_{dsub} = \frac{\left( \frac{\rasub{q}_s}{\rbsub{q}_s} - 1 \right) \frac{\rbsub{q}_s}{\cancel{\aprod{\eta}}} }
              {\left( \frac{\aprod{\eta}}{\bempl{\eta}} - 1 \right)\!\frac{\bempl{\eta}}{\cancel{\aprod{\eta}}} \rbempl{E}_s} \; ,
\end{equation}
%
and

\begin{equation}
    Re_{dsub} = \frac{\frac{\rasub{q}_s}{\rbsub{q}_s} - 1}{\frac{\aprod{\eta}}{\bempl{\eta}} - 1} \; \;
                \frac{\cancelto{\rbempl{E}_s}{\frac{\rbsub{q}_s}{\bempl{\eta}}}}{\rbempl{E}_s} \; .
\end{equation}
%
Noting that $\frac{\rbsub{q}_s}{\bempl{\eta}} = \frac{\rbempl{q}_s}{\bempl{\eta}} = \rbempl{E}_s$,
canceling $\rbempl{E}_s$ terms,
and substituting Eq.~(\ref{eq:q_ratio_func_of_eps}) gives

\begin{equation}
  Re_{dsub} = \Redsubeqn \; . \tag{\ref{eq:Re_dsub}}
\end{equation}

Note that the service price elasticity of energy service consumption is
expected to be negative ($\eqsps < 0$).
For example, when $\eqsps = -0.2$ and $\frac{\aprod{\eta}}{\bempl{\eta}} = 2$,
$Re_{dsub} = 0.15$.

To find the values for $Re_{dsub}$ in the limits where $\etaratioinline{} \to 1$ or 
$\etaratioinline{} \to \infty$, we take derivatives of numerator and denominator
of Eq.~(\ref{eq:Re_dsub}) with respect to $\etaratioinline{}$ and invoke L'H\^{o}pital's rule.

\begin{equation}
  \frac{\dbydetaeta{}\left[ \left( \etaratiostacked{} - 1 \right)^{-\eqsps}  \right]}
           {\dbydetaeta{}\left[ \etaratiostacked{} - 1\right]}
    = -\eqsps \left( \etaratiostacked{} \right)^{-(\eqsps + 1)}
\end{equation}
%
Substituting 1 and $\infty$ for $\etaratioinline$ gives

\begin{equation}
  \lim_{\aprod{\eta}/\bempl{\eta} \to 1^{\!+}} Re_{dsub} 
    = -\eqsps ( 1 )^{-(\eqsps + 1)}
    = -\eqsps \; ,                               \tag{\ref{eq:lim_Redsub_1}}
\end{equation}
%
and

\begin{equation}
  \lim_{\aprod{\eta}/\bempl{\eta} \to \infty} Re_{dsub} 
    = -\eqsps ( \infty )^{-(\eqsps + 1)}
    = 0 \; ,                                              \tag{\ref{eq:lim_Redsub_infty}}
\end{equation}
%
because $\eqsps \in [-1, 0)$ and, therefore, $(\eqsps + 1) > 0$.

With $\eqsps \in (-1, 0)$ expected,
direct substitution rebound will never be larger than 1.
The direct substitution effect alone
can never cause backfire. 

Finally, we derive an expression for indirect substitution rebound ($Re_{isub}$).


%..............................
\paragraph{Expression for $Re_{isub}$}
\label{sec:Re_isub}
%..............................

The increase in consumption of the energy service after its price drop
substitutes for consumption of other goods in the economy,
subject to a utility constraint.
The reduction in spending on other goods in the economy
is captured by indirect substitution rebound~($Re_{isub}$).

To derive an expression for indirect substitution rebound,
we begin with the definition of $Re_{isub}$
from Table~\ref{tab:subeffect}:

\begin{equation}
  Re_{isub} \equiv \frac{\Delta \rasub{C}_o I_E}{\Sdot} \; .  \tag{\ref{eq:Re_isub_dev}}
\end{equation}
%
Expansion of the difference term to $\rasub{C}_o - \rbsub{C}_o$ and rearranging the numerator gives

\begin{equation} \label{eq:Re_isub_prelim}
  Re_{isub} = \frac{\left( \frac{\rasub{C}_o}{\rbsub{C}_o} - 1  \right) \rbsub{C}_o I_E} {\Sdot} \; .
\end{equation}
%
We assume a basket of other goods purchased in the economy,
each ($i$) with its own price ($p_{o,i}$) and rate of consumption ($\rate{q}_{o,i}$),
such that the average price of all other goods purchased in the economy~($p_o$) is given by

\begin{equation}
  p_o = \frac{\sum\limits_i \bempl{p}_{o,i} \rbempl{q}_{o,i}}{\sum\limits_i \rbempl{q}_{o,i}} \; .
\end{equation}
%
Then, the cost rate of other purchases in the economy can be given as

\begin{equation}
  \rbsub{C}_o = \bsub{p}_o \rbsub{q}_o \; ,
\end{equation}
%
and

\begin{equation}
  \rasub{C}_o = \asub{p}_o \rasub{q}_o \; .
\end{equation}
%
Assuming that the average price is unchanged across the substitution effect,
such that $\asub{p}_o = \bsub{p}_o$,
the preceding two equations can be set equal and Eq.~(\ref{eq:qohat_qostar}) can be added to find

\begin{equation} \label{eq:Cdot_o_ratio}
  \frac{\rasub{C}_o}{\rbsub{C}_o} 
      = \frac{\rasub{q}_o}{\rbsub{q}_o} 
      = \left( \frac{\aprod{\eta}}{\bempl{\eta}} \right)^{-\eqops}  \; .
\end{equation}
%
Note that Eq.~(\ref{eq:Cdot_o_ratio}) 
(along with $\rbsub{C}_o = \rbempl{C}_o$)
can be used to determine the rate of expenditures 
on other goods in the economy~($\rasub{C}_o$) by

\begin{equation}
  \rasub{C}_o = \rbempl{C}_o \left( \frac{\aprod{\eta}}{\bempl{\eta}} \right)^{-\eqops} \; .
\end{equation}

Substituting Eq.~\ref{eq:Cdot_o_ratio} into Eq.~(\ref{eq:Re_isub_prelim}) gives

\begin{equation}
  Re_{isub} = \frac{\left[ \left( \frac{\aprod{\eta}}{\bempl{\eta}} \right)^{-\eqops} - 1  \right] \rbsub{C}_o I_E} {\Sdot} \; .
\end{equation}
%
Substituting Eq.~\ref{eq:Sdot} gives

\begin{equation}
  Re_{isub} = \frac{\left[ \left(\frac{\aprod{\eta}}{\bempl{\eta}} \right)
                  ^{-\eqops} - 1  \right] \rbsub{C}_o I_E}
                  {\Sdoteqn} \; .
\end{equation}
%
Noting that $\rbsub{C}_o = \rbempl{C}_o$ and rearranging yields

\begin{equation}
  Re_{isub} = \Reisubeqn{} \; . \tag{\ref{eq:Re_isub}}
\end{equation}

Because the service price elasticity of other goods consumption is positive ($\eqops > 0$) and
the energy service efficiency ratio is greater than 1 ($\aprod{\eta} > \bempl{\eta}$),
indirect substitution rebound will be negative always ($Re_{isub} < 0$),
as expected.
Negative rebound indicates that indirect substitution reduces the energy takeback by direct substitution.

We can take derivatives of numerator and denominator and
invoke L'H\^{o}pital's rule
to find limits for $Re_{isub}$ when
$\etaratioinline{} \to 1^{\! +}$ and when
$\etaratioinline{} \to \infty$.
First, rearranging Eq.~(\ref{eq:Re_isub}) gives

\begin{equation}
  Re_{isub} = \frac{\left( \etaratiostacked{} \right)^m - \etaratiostacked{}}{\left( \etaratiostacked{} \right) - 1} \frac{\rbempl{C}_o I_E}{\rbempl{E}_s} \; ,
\end{equation}
%
where $m \equiv 1 - \eqops$.
Taking the derivatives of numerator and denominator of Eq.~(\ref{eq:Re_isub})
with respect to $\etaratioinline{}$ gives

\begin{equation}
  \frac{\dbydetaeta{} \left[ \left( \etaratiostacked{} \right)^m - \etaratiostacked \right]}
            {\dbydetaeta{}\left[ \left( \etaratiostacked{} \right)  - 1 \right] } \frac{\rbempl{C}_o I_E}{\rbempl{E}_s} 
       = \left[ m \left( \etaratiostacked{} \right)^{m-1} - 1 \right] \frac{\rbempl{C}_o I_E}{\rbempl{E}_s} \; .
\end{equation}
%
Substituting $\etaratioinline = 1$ and invoking L'H\^{o}pital's rule gives

\begin{equation}
  \lim_{\etaratioinline \to 1^{+}} Re_{isub} 
        = \left[ m ( 1 )^{m-1} - 1 \right] \frac{\rbempl{C}_o I_E}{\rbempl{E}_s}
        = [m - 1] \frac{\rbempl{C}_o I_E}{\rbempl{E}_s} \; .
\end{equation}
%
Substituting for $m$ gives

\begin{equation}
  \lim_{\etaratioinline \to 1^{+}} Re_{isub} 
        = -\eqops \frac{\rbempl{C}_o I_E}{\rbempl{E}_s} \; . \tag{\ref{eq:lim_Reisub_1}}
\end{equation}

Similarly, 
substituting $\etaratioinline = \infty$ and invoking L'H\^{o}pital's rule gives

\begin{equation}
  \lim_{\etaratioinline \to \infty} Re_{isub} 
        = \left[ m ( \infty )^{m-1} - 1 \right] \frac{\rbempl{C}_o I_E}{\rbempl{E}_s} \; .
\end{equation}
%
Because $\eqops \in (0, 1)$ is expected **** Check with Gregor.  Is this true? ****,
$0 < m < 1$ is expected, $m - 1 < 0$ is expected, and

\begin{equation}
  \lim_{\aprod{\eta}/\bempl{\eta} \to \infty} Re_{isub} = -\frac{\rbempl{C}_0 I_E}{\rbempl{E}_s} \; .
\end{equation}


%------------------------------
\subsubsection{\Inceffect{}} 
\label{sec:Re_inc}
%------------------------------

Rebound from the income effect rebound quantifies the rate of additional energy demand 
that arises when the energy conversion device owner spends net
income from the EEU.
Derivations of expressions for gross and net income ($\rate{G}$ and $\rate{N}$, respectively) from the 
\empleffect{} ($\rate{G}$) and the \subeffect{} ($\rasub{N}$)
are presented in Tables~\ref{tab:empleffect} and~\ref{tab:subeffect}.
Gross income from the EEU is given by Eq.~(\ref{eq:G_dot})
as $\rate{G} = p_E \Sdot$. 
In combination, the \empleffect{} and the \subeffect{} leave the device owner with
\emph{net} income ($\rasub{N}$) from the EEU,
as shown in Eq.~(\ref{eq:N_dot_after_sub}).
Total income before the EEU is $\rbempl{M}$.
Total income after the substitution effect is $\rbempl{M} + \rasub{N}$.

In this framework, all net income ($\rasub{N}$) is spent either 
%
\begin{enumerate*}[label={(\alph*)}]
	
  \item on additional energy service ($\rainc{q}_s > \rbinc{q}_s$) or
  
  \item on additional other goods ($\rainc{q}_o > \rbinc{q}_o$).
    
\end{enumerate*}
%
The direct income elasticity of consumption ($\epsilon_{dinc}$) 
quantifies the amount of net income spent 
on more of the energy service ($\rainc{q}_s > \rbinc{q}_s$).
The budget constraint for the income effect (Eq.~(\ref{eq:inc_budget_constraint})) 
means that leftover income is spent on other goods.

The purpose of this section is derivation of expressions for 
direct income rebound~($Re_{dinc}$) and indirect income rebound~($Re_{iinc}$).
But we first derive expressions for later use.


%..............................
\paragraph{Expression for $\frac{\rainc{q}_s}{\rbinc{q}_s}$}
\label{sec:qs_ratio}
%..............................

The ratio of rates of energy service consumed across the income effect is given by

\begin{equation}
  \frac{\rainc{q}_s}{\rbinc{q}_s} = \left( \frac{\rbinc{M} + \rbinc{N}}{\rbinc{M}} \right) ^ {\eqsM} \; ,
\end{equation}
%
where $\eqsM$ is the income elasticity of energy service consumption.
Recognizing that $\rbinc{M} = \rbsub{M} = \rbempl{M}$ and rearranging slightly gives

\begin{equation}
  \frac{\rainc{q}_s}{\rbinc{q}_s} = \left( 1 + \frac{\rbinc{N}}{\rbempl{M}} \right) ^ {\eqsM} \; . 
                                                                                \tag{\ref{eq:q_ratio_across_inc}}
\end{equation}


%..............................
\paragraph{Expression for $\rbinc{E}_s$} 
\label{sec:E_dot_s_hat_expression}
%..............................

An expression for $\rbinc{E}_s$ that will be helpful later
begins with

\begin{equation}
  \rbinc{E}_s = \left( \frac{\rasub{E}_s}{\rbsub{E}_s} \right)
                \left( \frac{\raempl{E}_s}{\rbempl{E}_s} \right)
                \rbempl{E}_s \; .
\end{equation}
%
Substituting Eq.~(\ref{eq:typ_qs_eta_Edot}) and noting efficiency ($\eta$)
equalities from Table~\ref{tab:analysis_assumptions} gives

\begin{equation}
  \rbinc{E}_s = \left( \frac{\rasub{q}_s / \cancel{\aprod{\eta}}}{\rbsub{q}_s / \cancel{\aprod{\eta}}} \right)
                \left( \frac{\raempl{q}_s / \aprod{\eta}}{\rbempl{q}_s / \bempl{\eta}} \right)
                \rbempl{E}_s \; .
\end{equation}
%
Canceling terms yields

\begin{equation}
  \rbinc{E}_s = \left( \frac{\rasub{q}_s}{\rbsub{q}_s} \right)
                \left( \cancel{\frac{\raempl{q}_s}{\rbempl{q}_s}} \right)
                \left( \frac{\bempl{\eta}}{\aprod{\eta}}  \right)
                \rbempl{E}_s \; .
\end{equation}
%
Noting energy service consumption rate equalities from Table~\ref{tab:analysis_assumptions} 
($\raempl{q}_s = \rbempl{q}_s$) gives

\begin{equation} \label{eq:E_dot_s_hat}
  \rbinc{E}_s = \frac{\rasub{q}_s}{\rbsub{q}_s}
                \frac{\bempl{\eta}}{\aprod{\eta}}
                \rbempl{E}_s \; .
\end{equation}


%..............................
\paragraph{Expression for $\frac{\rbinc{N} I_E}{\Sdot}$}
\label{sec:N_dot_hat_I_E_over_Sdot}
%..............................

Another term of use later is $\frac{\rbinc{N} I_E}{\Sdot}$.
We begin by substituting Eq.~(\ref{eq:N_dot_after_sub}) for net income~($\rbinc{N}$).

\begin{equation}
  \frac{\rbinc{N} I_E}{\Sdot} = \frac{\rate{G} I_E}{\Sdot}
                                - \frac{\Delta \raempl{C}_{cap} I_E}{\Sdot}
                                - \frac{\Delta \raempl{C}_{\OMd} I_E}{\Sdot}
                                - \frac{p_E I_E \Delta \rasub{E}_s}{\Sdot}
                                - \frac{\Delta \rasub{C}_o I_E}{\Sdot}
\end{equation}
%
Substituting Eqs.~\ref{eq:G_dot}, \ref{eq:Re_OMd_def}, \ref{eq:Re_dsub_def}, \ref{eq:Re_isub_dev} gives

\begin{equation}
  \frac{\rbinc{N} I_E}{\Sdot} = \frac{p_E \cancel{\Sdot} I_E}{\cancel{\Sdot}}
                                - \frac{\Delta \raempl{C}_{cap} I_E}{\Sdot}
                                - Re_{\OMd}
                                - p_E I_E Re_{dsub}
                                - Re_{isub} \; .
\end{equation}
%
Canceling terms and defining $Re_{cap}$ as

\begin{equation}
  Re_{cap} \equiv \Recapeqn{} \; , \tag{\ref{eq:Re_cap}}
\end{equation}
%
gives

\begin{equation} \label{eq:N_dot_I_E_Sdot}
  \frac{\rbinc{N} I_E}{\Sdot} = p_E I_E
                                - Re_{cap}
                                - Re_{\OMd}
                                - p_E I_E Re_{dsub}
                                - Re_{isub} \; .
\end{equation}

The next step is to develop an expression for $Re_{dinc}$
using the direct income elasticity of consumption.


%..............................
\paragraph{Expression for $Re_{dinc}$}
\label{sec:Re_dinc}
%..............................

As shown in Table~\ref{tab:inceffect}, direct income rebound is defined as

\begin{equation}
  Re_{dinc} \equiv \frac{\Delta \rainc{E}_s}{\Sdot} \; . \tag{\ref{eq:Re_dinc_def}}
\end{equation}
%
Expanding the difference and rearranging gives

\begin{equation}
  Re_{dinc} = \frac{\rainc{E}_s - \rbinc{E}_s}{\Sdot} \; , 
\end{equation}
%
and

\begin{equation}
  Re_{dinc} = \frac{\left( \frac{\rainc{E}_s}{\rbinc{E}_s} - 1  \right) \rbinc{E}_s}{\Sdot} \; .
\end{equation}
%
Substituting the Eq.~(\ref{eq:typ_qs_eta_Edot}) as
$\rainc{E}_s = \frac{\rainc{q}_s}{\aprod{\eta}}$ and  
$\rbinc{E}_s = \frac{\rbinc{q}_s}{\aprod{\eta}}$ gives

\begin{equation}
  Re_{dinc} = \frac{\left( \frac{\rainc{q}_s / \cancel{\aprod{\eta}}}{\rbinc{q}_s / \cancel{\aprod{\eta}}} - 1  \right) \rbinc{E}_s} 
              {\Sdot} \; .
\end{equation}
%
Canceling terms and substituting Eqs.~(\ref{eq:Sdot}) and~(\ref{eq:q_ratio_across_inc}) gives

\begin{equation}
  Re_{dinc} = \frac{\left[ \left( 1 + \frac{\rbinc{N}}{\rbempl{M}} \right) ^{\eqsM} - 1  \right] \rbinc{E}_s} 
              {\Sdoteqn} \; .
\end{equation}
%
Substituting Eq.~(\ref{eq:E_dot_s_hat}) gives

\begin{equation}
  Re_{dinc} = \frac{\left[ \left( 1 + \frac{\rbinc{N}}{\rbempl{M}} \right) ^{\eqsM} - 1  \right] 
                  \frac{\rasub{q}_s}{\rbsub{q}_s}
                \cancel{\etaratiostacked}
                \cancel{\rbempl{E}_s}}
              {\left( \etaratiostacked - 1 \right)\! \cancel{\etaratiostacked} \cancel{\rbempl{E}_s}} \; .
\end{equation}
%
Canceling terms and substituting Eq.~(\ref{eq:q_ratio_func_of_eps}) gives

\begin{equation} 
  Re_{dinc} = \Redinceqn{} \; . \tag{\ref{eq:Re_dinc}}
\end{equation}

If there is no net income ($\rasub{N} = 0$), 
direct income effect rebound is zero ($Re_{dinc} = 0$), as expected.
As either of the elasticities get stronger 
($\eqsM$ becomes more positive or $\eqsps$ becomes more negative), 
direct income rebound~($Re_{dinc}$) grows.

The next step is to develop an expression for $Re_{iinc}$
using the budget constraint of Eq.~(\ref{eq:inc_budget_constraint}).

%..............................
\paragraph{Expression for $Re_{iinc}$}
\label{sec:Re_iinc}
%..............................

In this framework,
any net income not spent on procuring more of the energy service
goes toward consumption of other goods in the economy.
Rebound from the indirect income effect involves 
the energy implications of spending net income ($\rbinc{N}$)
on those other goods in the economy.

As shown in Table~\ref{tab:inceffect}, indirect income rebound is defined as

\begin{equation}
  Re_{iinc} \equiv \frac{\Delta \rainc{C}_o I_E}{\Sdot} \; . \tag{\ref{eq:Re_iinc_def}}
\end{equation}
%
The increased spending on other goods in the economy ($\Delta \rainc{C}_o$)
can be found by rearranging the budget constraint for the income effect 
(Eq.~(\ref{eq:inc_budget_constraint})) to be 
$\Delta \rainc{C}_o = \rbinc{N} - p_E \Delta \rainc{E}_s$.
Substituting gives

\begin{equation}
  Re_{iinc} = \frac{\rbinc{N} I_E - p_E I_E \Delta \rainc{E}_s}{\Sdot} \; .
\end{equation}
%
Splitting the terms and substituting Eq.~(\ref{eq:Re_dinc_def}) gives

\begin{equation}
  Re_{iinc} = \frac{\rbinc{N} I_E}{\Sdot} - p_E I_E Re_{dinc} \; .
\end{equation}
%
Substituting Eq.~(\ref{eq:N_dot_I_E_Sdot}) gives 

\begin{equation}
  Re_{iinc} = \Reiinceqn{} \; . \tag{\ref{eq:Re_iinc}}
\end{equation}

This equation shows that a good first estimate for indirect income rebound
is $p_E I_E$.
Positive rebound from capital expenditures~($Re_{cap}$), 
maintenance and disposal~($Re_{\OMd}$),
direct substitution~($Re_{dsub}$), 
indirect substitution~($Re_{isub}$), and 
direct income~($Re_{dinc}$)
reduce the indirect income effect rebound, 
because each reduces net income available to the device owner to spend on other goods in the economy.
Any negative rebound effects enhance the indirect income effect rebound, because 
they add to the net income available to the device owner to spend on other goods in the economy.


%------------------------------
\subsubsection{Expression for total direct rebound ($Re_d$)} 
\label{sec:Re_d}
%------------------------------

Total direct rebound is the sum of 
direct substitution rebound ($Re_{dsub}$, Eq.~(\ref{eq:Re_dsub})) and
direct income rebound ($Re_{dinc}$, Eq.~(\ref{eq:Re_dinc})).

\begin{align} 
  Re_d &= Re_{dsub} + Re_{dinc} \nonumber \\
       &= \Redsubeqn + \Redinceqn
\end{align}
%
Simplification gives

\begin{equation} \label{eq:Re_d}
  Re_d = \frac{ \left( \etaratiostacked \right)^{-\eqsps}
             \left( 1 + \frac{\rasub{N}}{\rbempl{M}} \right)^{\eqsM}   - 1}
         {\etaratiostacked - 1} \; .
\end{equation}

It is instructive to compare Eq.~(\ref{eq:Re_dsub}) and Eq.~(\ref{eq:Re_d}).
Because 
$\frac{\rasub{N}}{\rbempl{M}} > 0$ and 
$\eqsM > 0$,
total direct rebound~($Re_d$) is greater than 
direct substitution rebound~($Re_{dsub}$) by a factor
that is associated with direct income rebound, namely
$\left( 1 + \frac{\rasub{N}}{\rbempl{M}} \right)^{\eqsM}$.


%------------------------------
\subsubsection{\Prodeffect{}} 
\label{sec:Re_prod}
%------------------------------

Productivity rebound~($Re_{prod}$) is given by Eq.~(\ref{eq:Re_prod_def}).
Substituting Eq.~(\ref{eq:N_dot_I_E_Sdot}) gives

\begin{equation}
  Re_{prod} = \Reprodeqn{} \; . \tag{\ref{eq:Re_prod}}
\end{equation}


%------------------------------
\subsubsection{Rebound sum} 
\label{sec:total_rebound}
%------------------------------

The sum of all rebound effects is 

\begin{equation}
  Re_{sum} = Re_{empl} + Re_{sub} + Re_{inc} + Re{prod} \; .
\end{equation}
%
Substituting Eqs.~(\ref{eq:Re_empl_def}), (\ref{eq:Re_sub_def}), and~(\ref{eq:Re_inc_def}) gives

\begin{align}
  Re_{sum} = \; &Re_{emb} + Re_{\OMd}      & \mathrm{\empleffect} \nonumber \\
                &+ Re_{dsub} + Re_{isub}   & \mathrm{\subeffect}  \nonumber \\
                &+ Re_{dinc} + Re_{iinc}   & \mathrm{\inceffect}  \nonumber \\
                &+ Re_{prod}               & \mathrm{\prodeffect}
\end{align}
%
Interestingly, 
indirect income effect rebound~($Re_{iinc}$, Eq.~(\ref{eq:Re_iinc})) and
productivity effect rebound~($Re_{prod}$, Eq.~(\ref{eq:Re_prod}))
can be expressed in terms of other rebound effects.
Substituting Eqs.~(\ref{eq:Re_iinc}) and~(\ref{eq:Re_prod}) gives

\begin{align}
  Re_{sum} = \; &Re_{emb} + Re_{\OMd}      & \mathrm{\empleffect}                           \nonumber \\
                &+ Re_{dsub} + Re_{isub}   & \mathrm{\subeffect}                            \nonumber \\
                &+ Re_{dinc} + p_E I_E - Re_{cap} - Re_{\OMd} - p_E I_E Re_{dsub} 
                             - Re_{isub} - p_E I_E Re_{dinc}   & \mathrm{\inceffect}        \nonumber \\
                &+ k p_E I_E - k Re_{cap} - k Re_{\OMd} - k p_E I_E Re_{dsub} - k Re_{isub} \; .  & \mathrm{\prodeffect}
\end{align}
%
Rearranging distributes many indirect income effect and productivity effect terms 
to other life cycle, substitution, and income effect terms.
This last rearrangement gives the final expression for total rebound.

\begin{align}
  Re_{sum} = \; \Retoteqn{} \tag{\ref{eq:Re_tot}}
\end{align}

Eq.~(\ref{eq:Re_tot}) shows that determining six rebound values,

\begin{itemize}

  \item $Re_{emb}$ (Eq.~(\ref{eq:Re_emb})), 

  \item $Re_{cap}$ (Eq.~(\ref{eq:Re_cap})), 
  
  \item $Re_{\OMd}$ (Eq.~(\ref{eq:Re_OMd})),
  
  \item $Re_{dsub}$ (Eq.~(\ref{eq:Re_dsub})),
  
  \item $Re_{isub}$ (Eq.~(\ref{eq:Re_isub})), and
  
  \item $Re_{dinc}$ (Eq.~(\ref{eq:Re_dinc})),

\end{itemize}
%
is sufficient to calculate total rebound, 
provided that 
the productivity factor~($k$),
the price of energy~($p_E$), and
the energy intensity of the economy~($I_E$) 
are known.
