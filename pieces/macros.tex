% The next command tells RStudio to do "Compile PDF" on HSB.Rnw,
% instead of this file, thereby eliminating the need to switch back to HSB.Rnw 
% before building the paper.
%!TEX root = ../HSB.Rnw


%%%%%%%%%%%%%%%%%%%%%%%%%%%%%%%%%%%%%%%%%%%%%%%%%%%%%%%%%%%%%%
% This file contains macros for
% Heun, Semieniuk, Brockway, 
% A Comprehensive Analytical Framework for Energy Rebound. 
% It is incorporated into the main file by the command
% % The next command tells RStudio to do "Compile PDF" on HSB.Rnw,
% instead of this file, thereby eliminating the need to switch back to HSB.Rnw 
% before building the paper.
%!TEX root = ../HSB.Rnw


%%%%%%%%%%%%%%%%%%%%%%%%%%%%%%%%%%%%%%%%%%%%%%%%%%%%%%%%%%%%%%
% This file contains macros for
% Heun, Semieniuk, Brockway, 
% A Comprehensive Analytical Framework for Energy Rebound. 
% It is incorporated into the main file by the command
% % The next command tells RStudio to do "Compile PDF" on HSB.Rnw,
% instead of this file, thereby eliminating the need to switch back to HSB.Rnw 
% before building the paper.
%!TEX root = ../HSB.Rnw


%%%%%%%%%%%%%%%%%%%%%%%%%%%%%%%%%%%%%%%%%%%%%%%%%%%%%%%%%%%%%%
% This file contains macros for
% Heun, Semieniuk, Brockway, 
% A Comprehensive Analytical Framework for Energy Rebound. 
% It is incorporated into the main file by the command
% % The next command tells RStudio to do "Compile PDF" on HSB.Rnw,
% instead of this file, thereby eliminating the need to switch back to HSB.Rnw 
% before building the paper.
%!TEX root = ../HSB.Rnw


%%%%%%%%%%%%%%%%%%%%%%%%%%%%%%%%%%%%%%%%%%%%%%%%%%%%%%%%%%%%%%
% This file contains macros for
% Heun, Semieniuk, Brockway, 
% A Comprehensive Analytical Framework for Energy Rebound. 
% It is incorporated into the main file by the command
% \input{macros.tex}.
%%%%%%%%%%%%%%%%%%%%%%%%%%%%%%%%%%%%%%%%%%%%%%%%%%%%%%%%%%%%%%

%%%%% Decorations for symbols

\newcommand{\rate}[1]{\dot{#1}}                    % Rate of a quantity

% Create "after" commands
\newcommand{\orig}[1]{#1}
\newcommand{\aempl}[1]{{#1}^*}
\newcommand{\asub}[1]{\hat{#1}}
\newcommand{\ainc}[1]{\bar{#1}}
\newcommand{\aprod}[1]{\tilde{#1}}

% Create the "before" commands
\newcommand{\bempl}[1]{\orig{#1}}
\newcommand{\bsub}[1]{\aempl{#1}}
\newcommand{\binc}[1]{\asub{#1}}
\newcommand{\bprod}[1]{\ainc{#1}}

% Decoration combinations
% Rates after
\newcommand{\rorig}[1]{\orig{\rate{#1}}}
\newcommand{\raempl}[1]{\aempl{\rate{#1}}}
\newcommand{\rasub}[1]{\asub{\rate{#1}}}
\newcommand{\rainc}[1]{\ainc{\rate{#1}}}
\newcommand{\raprod}[1]{\aprod{\rate{#1}}}

% Rates before
\newcommand{\rbempl}[1]{\rorig{#1}}
\newcommand{\rbsub}[1]{\raempl{#1}}
\newcommand{\rbinc}[1]{\rasub{#1}}
\newcommand{\rbprod}[1]{\rainc{#1}}

%%%%% Subscript kerning

\newcommand{\OM}{O\!M}
% \newcommand{\OMd}{O\!M\!d}
\newcommand{\OMd}{md}
\newcommand{\productivity}{pr\!od}


%%%%% Convenient symbols

\newcommand{\Sdot}{\rate{S}_{dev}}
\newcommand{\Mdothatprime}{\rbinc{M}^\prime}


%%%%% Elasticities and income shares

\newcommand{\eqsps}{\epsilon_{\rate{q}_s\!,p_s}}
\newcommand{\eqops}{\epsilon_{\rate{q}_o\!,p_s}}
\newcommand{\eqsM}{\epsilon_{\rate{q}_s\!,\rate{M}}}
\newcommand{\eqoM}{\epsilon_{\rate{q}_o\!,\rate{M}}}
\newcommand{\eqspsUC}{\epsilon_{\rate{q}_s\!,p_s\!,U\!C}}
\newcommand{\eqopsUC}{\epsilon_{\rate{q}_o\!,p_s\!,U\!C}}
\newcommand{\fqs}{f_{\rbempl{q}_s}}

%%%%% Colors

\newcommand{\emplcolor}{red!25!white}
\newcommand{\subcolor}{orange!25!white}
\newcommand{\inccolor}{green!25!white}
\newcommand{\prodcolor}{blue!25!white}


%%%%% Coloration of text background

%
% Inline color box around text
% Arguments:
%   [#1]: background color for the box
%   {#2}: text inside the box
%
\newtcbox{\inlinebox}[1][]{on line, 
colback=#1,
colframe=#1,
before upper={\rule[-2pt]{0pt}{10pt}},
boxrule=1pt,
boxsep=0pt,
left=3pt,
right=3pt,
top=2pt,
bottom=2pt}


%%%%% Colored phrases

% Emplacement effect
\newcommand{\empleffect}{\inlinebox[\emplcolor]{emplacement effect}}
\newcommand{\empleffectadj}{\inlinebox[\emplcolor]{emplacement-effect}}
\newcommand{\Empleffect}{\inlinebox[\emplcolor]{Emplacement effect}}
\newcommand{\EmplEffect}{\inlinebox[\emplcolor]{Emplacement Effect}}

% Substitution effect
\newcommand{\subeffect}{\inlinebox[\subcolor]{substitution effect}}
\newcommand{\subeffectadj}{\inlinebox[\subcolor]{substitution-effect}}
\newcommand{\Subeffect}{\inlinebox[\subcolor]{Substitution effect}}
\newcommand{\SubEffect}{\inlinebox[\subcolor]{Substitution Effect}}

% Income effect
\newcommand{\inceffect}{\inlinebox[\inccolor]{income effect}}
\newcommand{\inceffectadj}{\inlinebox[\inccolor]{income-effect}}
\newcommand{\Inceffect}{\inlinebox[\inccolor]{Income effect}}
\newcommand{\IncEffect}{\inlinebox[\inccolor]{Income Effect}}

% Productivity effect
\newcommand{\prodeffect}{\inlinebox[\prodcolor]{productivity effect}}
\newcommand{\prodeffectadj}{\inlinebox[\prodcolor]{productivity-effect}}
\newcommand{\Prodeffect}{\inlinebox[\prodcolor]{Productivity effect}}
\newcommand{\ProdEffect}{\inlinebox[\prodcolor]{Productivity Effect}}



%%%%% minipage for assumptions and constraints tables
% Arguments:
%   #1: Width (multiple of \linewidth
%   #2: text inside the minipage
%
\newcommand{\mptable}[2]{\begin{minipage}{#1\linewidth} \useshortskip{} \begin{equation} #2 \end{equation} \end{minipage}}


%%%%% Oft-used references

\newcommand{\Ba}[1]{\citeauthor[#1]{Borenstein:2015aa}}
\newcommand{\Bapp}[1]{\citeauthor[#1]{Borenstein:2015aa}'s \citeyearpar{Borenstein:2015aa}}
\newcommand{\Bp}[1]{\citep[#1]{Borenstein:2015aa}}
\newcommand{\Bt}[1]{\citet[#1]{Borenstein:2015aa}}

\newcommand{\Ta}[1]{\citeauthor[#1]{Thomas:2013aa}}
\newcommand{\Tapp}[1]{\citeauthor[#1]{Thomas:2013aa}'s \citeyearpar{Thomas:2013aa}}
\newcommand{\Tp}[1]{\citep[#1]{Thomas:2013aa, Thomas:2013ab}}
\newcommand{\Tpone}[1]{\citep[#1]{Thomas:2013aa}}
\newcommand{\Tptwo}[1]{\citep[#1]{Thomas:2013ab}}
\newcommand{\Tt}[1]{\citet[#1]{Thomas:2013aa, Thomas:2013ab}}
\newcommand{\Ttone}[1]{\citet[#1]{Thomas:2013aa}}
\newcommand{\Tttwo}[1]{\citet[#1]{Thomas:2013ab}}


%%%%% Derivation pages

% Column widths
\newcommand{\derivtextsize}{\footnotesize}
\newcommand{\derivpageleftcolwidth}{0.11\textwidth}
\newcommand{\derivpageenergycolwidth}{0.7\textwidth}
\newcommand{\derivpagefinancialcolwidth}{0.7\textwidth}

% Horizontal rule between sections of derivations

\newcommand{\sectionsep}{\noindent\rule{1.6\textwidth}{0.4pt}}


%
% Derivation section
% Arguments:
%   #1: accounting stage (original, prime, etc.)
%   #2: energy column
%   #3: financial column
%
\newcommand{\derivsection}[3]{%

\derivtextsize{}

\begin{minipage}[t]{\derivpageleftcolwidth}
~\\#1
\end{minipage}
%
%
%
\begin{minipage}[t]{\derivpageenergycolwidth}
#2
\end{minipage}
%
~
%
\begin{minipage}[t]{\derivpagefinancialcolwidth}
#3
\end{minipage}

\normalsize{}

}


%
% Derivation page header
% Arguments:
%   #1: Effect type header text (e.g., Emplacement Effect)
%
\newcommand{\derivheader}[1]{

\begin{center}
  #1
\end{center}

\derivsection{}
{\begin{center}\emph{Energy analysis}\end{center}}
{\begin{center}\emph{Financial analysis}\end{center}}

}


% Equations

% Efficiency ratios
\newcommand{\etaratioinline}{\aprod{\eta}/\bempl{\eta}}
\newcommand{\etaratiostacked}{\frac{\aprod{\eta}}{\bempl{\eta}}}

% Derivative with respect to efficiency ratio
\newcommand{\dbydetaeta}{\frac{\mathrm{d}}{\mathrm{d}(\etaratioinline{})}}


% Original
\newcommand{\Eacctorig}{\rbempl{E} = \rbempl{E}_s + \rbempl{E}_{emb} + (\rbempl{C}_{\OMd} + \rbempl{C}_o) I_E}
\newcommand{\Macctorig}{\rbempl{M} = p_E \rbempl{E}_s + \rbempl{C}_{cap} + \rbempl{C}_{\OMd} + \rbempl{C}_o + \rbempl{N}}

% Before emplacement effect (same as original)
\newcommand{\Eacctbempl}{\Eacctorig}      
\newcommand{\Macctbempl}{\Macctorig}      

% After emplacement effect
\newcommand{\Eacctaempl}{\raempl{E} = \raempl{E}_s + \raempl{E}_{emb} + (\raempl{C}_{\OMd} + \raempl{C}_o) I_E}                  
\newcommand{\Macctaempl}{\raempl{M} = p_E \raempl{E}_s + \raempl{C}_{cap} + \raempl{C}_{\OMd} + \raempl{C}_o + \raempl{N}}         

% Before substitution effect (same as after emplacement effect)
\newcommand{\Eacctbsub}{\Eacctaempl}
\newcommand{\Macctbsub}{\Macctaempl}

% After substitution effect
\newcommand{\Eacctasub}{\rasub{E} = \rasub{E}_s + \rasub{E}_{emb} + (\rasub{C}_{\OMd} + \rasub{C}_o) I_E}
\newcommand{\Macctasub}{\rasub{M} = p_E \rasub{E}_s + \rasub{C}_{cap} + \rasub{C}_{\OMd} + \rasub{C}_o + \rasub{N}}

% Before income effect (same as after substitution effect)
\newcommand{\Eacctbinc}{\Eacctasub}
\newcommand{\Macctbinc}{\Macctasub}

% After income effect
\newcommand{\Eacctainc}{\rainc{E} = \rainc{E}_s + \rainc{E}_{emb} + (\rainc{C}_{\OMd} + \rainc{C}_o) I_E}
\newcommand{\Macctainc}{\rainc{M} = p_E \rainc{E}_s + \rainc{C}_{cap} + \rainc{C}_{\OMd} + \rainc{C}_o + \rainc{N}}

% Embodied energy rebound
\newcommand{\Reembeqn}{\frac{\left( \frac{\aempl{E}_{emb}}{\bempl{E}_{emb}}
  \frac{\bempl{t}}{\aempl{t}} - 1 \right) \rbempl{E}_{emb}}{\Sdot}}
  
% Ops, Maintenance, and disposal energy rebound
\newcommand{\ReOMdeqn}{\frac{\left( \frac{\raempl{C}_{\OMd}}{\rbempl{C}_{\OMd}} - 1 \right) \rbempl{C}_{\OMd} I_E}{\Sdot}}

% Equation for S_dot_dev
\newcommand{\Sdoteqn}{\left( \etaratiostacked - 1 \right)\!\etaratiostacked \rbempl{E}_s}

% Equation for Re_dsub
\newcommand{\Redsubeqn}{\frac{\left( \etaratiostacked \right)^{-\eqsps} - 1}
                        {\etaratiostacked - 1}}
                        
% Equation for Re_isub
\newcommand{\Reisubeqn}{\frac{{\left( \etaratiostacked  \right)}
                          ^{-\eqops} - 1}{\etaratiostacked - 1} \; 
                          \etaratiostacked \; 
                          \frac{\rbempl{C}_o I_E}{\rbempl{E}_s}}

% Equation for Re_dinc
\newcommand{\Redinceqn}{\frac{\left[ \left( 1 + \frac{\rbinc{N}}{\Mdothatprime} \right) ^{\eqsM} - 1  \right]}
              { \etaratiostacked - 1 } \left( \etaratiostacked \right)^{-\eqsps}}

% Equation for Re_cap
\newcommand{\Recapeqn}{\frac{\Delta \raempl{C}_{cap} I_E}{\Sdot}}

% Equation for Re_iinc
\newcommand{\Reiinceqn}{\frac{\left( 1 + \frac{\rbinc{N}}{\Mdothatprime} \right)^{\eqoM} - 1}{\etaratiostacked - 1} 
              \left( \etaratiostacked \right)^{-(1 + \eqops)}
              \frac{\rbempl{C}_o I_E}{\rbempl{E}_s}}

% Equation for Re_d (total direct rebound)
\newcommand{\Redeqn}{\frac{ \left( \etaratiostacked \right)^{-\eqsps}
             \left( 1 + \frac{\rasub{N}}{\rbempl{M}} \right)^{\eqsM}   - 1}
         {\etaratiostacked - 1}}

% Equation for Re_prod
\newcommand{\Reprodeqn}{k p_E I_E - k Re_{cap} - k Re_{\OMd} - k p_E I_E Re_{dsub} - k Re_{isub}}

% Equation for Re_tot
\newcommand{\Retoteqn}{&Re_{emb} - k Re_{cap} + (1-k) Re_{\OMd}         \nonumber \\
                       &+ (1 - k p_E I_E) Re_{dsub} + (1 - k) Re_{isub}   \nonumber \\
                       &+ Re_{dinc} + Re_{iinc} +  k p_E I_E}
                 

%%%% Income preference equations

% Equation for energy service income preferences
\newcommand{\incprefseqn}{\frac{\rainc{q}_s}{\rbinc{q}_s} = \left( 1 + \frac{\rbinc{N}}{\Mdothatprime}  \right) ^{\eqsM}}

% Equation for other goods income preferences
\newcommand{\incprefoeqn}{\frac{\rainc{q}_o}{\rbinc{q}_o} = \left( 1 + \frac{\rbinc{N}}{\Mdothatprime}  \right) ^{\eqoM}}

% Equation for effective income
\newcommand{\effinceqn}{\Mdothatprime \equiv \rbempl{M} - \rbempl{C}_{cap} - \rbempl{C}_{md} 
                        - \rate{G} + p_E \Delta \rbinc{E}_s + \Delta \rbinc{C}_o}.
%%%%%%%%%%%%%%%%%%%%%%%%%%%%%%%%%%%%%%%%%%%%%%%%%%%%%%%%%%%%%%

%%%%% Decorations for symbols

\newcommand{\rate}[1]{\dot{#1}}                    % Rate of a quantity

% Create "after" commands
\newcommand{\orig}[1]{#1}
\newcommand{\aempl}[1]{{#1}^*}
\newcommand{\asub}[1]{\hat{#1}}
\newcommand{\ainc}[1]{\bar{#1}}
\newcommand{\aprod}[1]{\tilde{#1}}

% Create the "before" commands
\newcommand{\bempl}[1]{\orig{#1}}
\newcommand{\bsub}[1]{\aempl{#1}}
\newcommand{\binc}[1]{\asub{#1}}
\newcommand{\bprod}[1]{\ainc{#1}}

% Decoration combinations
% Rates after
\newcommand{\rorig}[1]{\orig{\rate{#1}}}
\newcommand{\raempl}[1]{\aempl{\rate{#1}}}
\newcommand{\rasub}[1]{\asub{\rate{#1}}}
\newcommand{\rainc}[1]{\ainc{\rate{#1}}}
\newcommand{\raprod}[1]{\aprod{\rate{#1}}}

% Rates before
\newcommand{\rbempl}[1]{\rorig{#1}}
\newcommand{\rbsub}[1]{\raempl{#1}}
\newcommand{\rbinc}[1]{\rasub{#1}}
\newcommand{\rbprod}[1]{\rainc{#1}}

%%%%% Subscript kerning

\newcommand{\OM}{O\!M}
% \newcommand{\OMd}{O\!M\!d}
\newcommand{\OMd}{md}
\newcommand{\productivity}{pr\!od}


%%%%% Convenient symbols

\newcommand{\Sdot}{\rate{S}_{dev}}
\newcommand{\Mdothatprime}{\rbinc{M}^\prime}


%%%%% Elasticities and income shares

\newcommand{\eqsps}{\epsilon_{\rate{q}_s\!,p_s}}
\newcommand{\eqops}{\epsilon_{\rate{q}_o\!,p_s}}
\newcommand{\eqsM}{\epsilon_{\rate{q}_s\!,\rate{M}}}
\newcommand{\eqoM}{\epsilon_{\rate{q}_o\!,\rate{M}}}
\newcommand{\eqspsUC}{\epsilon_{\rate{q}_s\!,p_s\!,U\!C}}
\newcommand{\eqopsUC}{\epsilon_{\rate{q}_o\!,p_s\!,U\!C}}
\newcommand{\fqs}{f_{\rbempl{q}_s}}

%%%%% Colors

\newcommand{\emplcolor}{red!25!white}
\newcommand{\subcolor}{orange!25!white}
\newcommand{\inccolor}{green!25!white}
\newcommand{\prodcolor}{blue!25!white}


%%%%% Coloration of text background

%
% Inline color box around text
% Arguments:
%   [#1]: background color for the box
%   {#2}: text inside the box
%
\newtcbox{\inlinebox}[1][]{on line, 
colback=#1,
colframe=#1,
before upper={\rule[-2pt]{0pt}{10pt}},
boxrule=1pt,
boxsep=0pt,
left=3pt,
right=3pt,
top=2pt,
bottom=2pt}


%%%%% Colored phrases

% Emplacement effect
\newcommand{\empleffect}{\inlinebox[\emplcolor]{emplacement effect}}
\newcommand{\empleffectadj}{\inlinebox[\emplcolor]{emplacement-effect}}
\newcommand{\Empleffect}{\inlinebox[\emplcolor]{Emplacement effect}}
\newcommand{\EmplEffect}{\inlinebox[\emplcolor]{Emplacement Effect}}

% Substitution effect
\newcommand{\subeffect}{\inlinebox[\subcolor]{substitution effect}}
\newcommand{\subeffectadj}{\inlinebox[\subcolor]{substitution-effect}}
\newcommand{\Subeffect}{\inlinebox[\subcolor]{Substitution effect}}
\newcommand{\SubEffect}{\inlinebox[\subcolor]{Substitution Effect}}

% Income effect
\newcommand{\inceffect}{\inlinebox[\inccolor]{income effect}}
\newcommand{\inceffectadj}{\inlinebox[\inccolor]{income-effect}}
\newcommand{\Inceffect}{\inlinebox[\inccolor]{Income effect}}
\newcommand{\IncEffect}{\inlinebox[\inccolor]{Income Effect}}

% Productivity effect
\newcommand{\prodeffect}{\inlinebox[\prodcolor]{productivity effect}}
\newcommand{\prodeffectadj}{\inlinebox[\prodcolor]{productivity-effect}}
\newcommand{\Prodeffect}{\inlinebox[\prodcolor]{Productivity effect}}
\newcommand{\ProdEffect}{\inlinebox[\prodcolor]{Productivity Effect}}



%%%%% minipage for assumptions and constraints tables
% Arguments:
%   #1: Width (multiple of \linewidth
%   #2: text inside the minipage
%
\newcommand{\mptable}[2]{\begin{minipage}{#1\linewidth} \useshortskip{} \begin{equation} #2 \end{equation} \end{minipage}}


%%%%% Oft-used references

\newcommand{\Ba}[1]{\citeauthor[#1]{Borenstein:2015aa}}
\newcommand{\Bapp}[1]{\citeauthor[#1]{Borenstein:2015aa}'s \citeyearpar{Borenstein:2015aa}}
\newcommand{\Bp}[1]{\citep[#1]{Borenstein:2015aa}}
\newcommand{\Bt}[1]{\citet[#1]{Borenstein:2015aa}}

\newcommand{\Ta}[1]{\citeauthor[#1]{Thomas:2013aa}}
\newcommand{\Tapp}[1]{\citeauthor[#1]{Thomas:2013aa}'s \citeyearpar{Thomas:2013aa}}
\newcommand{\Tp}[1]{\citep[#1]{Thomas:2013aa, Thomas:2013ab}}
\newcommand{\Tpone}[1]{\citep[#1]{Thomas:2013aa}}
\newcommand{\Tptwo}[1]{\citep[#1]{Thomas:2013ab}}
\newcommand{\Tt}[1]{\citet[#1]{Thomas:2013aa, Thomas:2013ab}}
\newcommand{\Ttone}[1]{\citet[#1]{Thomas:2013aa}}
\newcommand{\Tttwo}[1]{\citet[#1]{Thomas:2013ab}}


%%%%% Derivation pages

% Column widths
\newcommand{\derivtextsize}{\footnotesize}
\newcommand{\derivpageleftcolwidth}{0.11\textwidth}
\newcommand{\derivpageenergycolwidth}{0.7\textwidth}
\newcommand{\derivpagefinancialcolwidth}{0.7\textwidth}

% Horizontal rule between sections of derivations

\newcommand{\sectionsep}{\noindent\rule{1.6\textwidth}{0.4pt}}


%
% Derivation section
% Arguments:
%   #1: accounting stage (original, prime, etc.)
%   #2: energy column
%   #3: financial column
%
\newcommand{\derivsection}[3]{%

\derivtextsize{}

\begin{minipage}[t]{\derivpageleftcolwidth}
~\\#1
\end{minipage}
%
%
%
\begin{minipage}[t]{\derivpageenergycolwidth}
#2
\end{minipage}
%
~
%
\begin{minipage}[t]{\derivpagefinancialcolwidth}
#3
\end{minipage}

\normalsize{}

}


%
% Derivation page header
% Arguments:
%   #1: Effect type header text (e.g., Emplacement Effect)
%
\newcommand{\derivheader}[1]{

\begin{center}
  #1
\end{center}

\derivsection{}
{\begin{center}\emph{Energy analysis}\end{center}}
{\begin{center}\emph{Financial analysis}\end{center}}

}


% Equations

% Efficiency ratios
\newcommand{\etaratioinline}{\aprod{\eta}/\bempl{\eta}}
\newcommand{\etaratiostacked}{\frac{\aprod{\eta}}{\bempl{\eta}}}

% Derivative with respect to efficiency ratio
\newcommand{\dbydetaeta}{\frac{\mathrm{d}}{\mathrm{d}(\etaratioinline{})}}


% Original
\newcommand{\Eacctorig}{\rbempl{E} = \rbempl{E}_s + \rbempl{E}_{emb} + (\rbempl{C}_{\OMd} + \rbempl{C}_o) I_E}
\newcommand{\Macctorig}{\rbempl{M} = p_E \rbempl{E}_s + \rbempl{C}_{cap} + \rbempl{C}_{\OMd} + \rbempl{C}_o + \rbempl{N}}

% Before emplacement effect (same as original)
\newcommand{\Eacctbempl}{\Eacctorig}      
\newcommand{\Macctbempl}{\Macctorig}      

% After emplacement effect
\newcommand{\Eacctaempl}{\raempl{E} = \raempl{E}_s + \raempl{E}_{emb} + (\raempl{C}_{\OMd} + \raempl{C}_o) I_E}                  
\newcommand{\Macctaempl}{\raempl{M} = p_E \raempl{E}_s + \raempl{C}_{cap} + \raempl{C}_{\OMd} + \raempl{C}_o + \raempl{N}}         

% Before substitution effect (same as after emplacement effect)
\newcommand{\Eacctbsub}{\Eacctaempl}
\newcommand{\Macctbsub}{\Macctaempl}

% After substitution effect
\newcommand{\Eacctasub}{\rasub{E} = \rasub{E}_s + \rasub{E}_{emb} + (\rasub{C}_{\OMd} + \rasub{C}_o) I_E}
\newcommand{\Macctasub}{\rasub{M} = p_E \rasub{E}_s + \rasub{C}_{cap} + \rasub{C}_{\OMd} + \rasub{C}_o + \rasub{N}}

% Before income effect (same as after substitution effect)
\newcommand{\Eacctbinc}{\Eacctasub}
\newcommand{\Macctbinc}{\Macctasub}

% After income effect
\newcommand{\Eacctainc}{\rainc{E} = \rainc{E}_s + \rainc{E}_{emb} + (\rainc{C}_{\OMd} + \rainc{C}_o) I_E}
\newcommand{\Macctainc}{\rainc{M} = p_E \rainc{E}_s + \rainc{C}_{cap} + \rainc{C}_{\OMd} + \rainc{C}_o + \rainc{N}}

% Embodied energy rebound
\newcommand{\Reembeqn}{\frac{\left( \frac{\aempl{E}_{emb}}{\bempl{E}_{emb}}
  \frac{\bempl{t}}{\aempl{t}} - 1 \right) \rbempl{E}_{emb}}{\Sdot}}
  
% Ops, Maintenance, and disposal energy rebound
\newcommand{\ReOMdeqn}{\frac{\left( \frac{\raempl{C}_{\OMd}}{\rbempl{C}_{\OMd}} - 1 \right) \rbempl{C}_{\OMd} I_E}{\Sdot}}

% Equation for S_dot_dev
\newcommand{\Sdoteqn}{\left( \etaratiostacked - 1 \right)\!\etaratiostacked \rbempl{E}_s}

% Equation for Re_dsub
\newcommand{\Redsubeqn}{\frac{\left( \etaratiostacked \right)^{-\eqsps} - 1}
                        {\etaratiostacked - 1}}
                        
% Equation for Re_isub
\newcommand{\Reisubeqn}{\frac{{\left( \etaratiostacked  \right)}
                          ^{-\eqops} - 1}{\etaratiostacked - 1} \; 
                          \etaratiostacked \; 
                          \frac{\rbempl{C}_o I_E}{\rbempl{E}_s}}

% Equation for Re_dinc
\newcommand{\Redinceqn}{\frac{\left[ \left( 1 + \frac{\rbinc{N}}{\Mdothatprime} \right) ^{\eqsM} - 1  \right]}
              { \etaratiostacked - 1 } \left( \etaratiostacked \right)^{-\eqsps}}

% Equation for Re_cap
\newcommand{\Recapeqn}{\frac{\Delta \raempl{C}_{cap} I_E}{\Sdot}}

% Equation for Re_iinc
\newcommand{\Reiinceqn}{\frac{\left( 1 + \frac{\rbinc{N}}{\Mdothatprime} \right)^{\eqoM} - 1}{\etaratiostacked - 1} 
              \left( \etaratiostacked \right)^{-(1 + \eqops)}
              \frac{\rbempl{C}_o I_E}{\rbempl{E}_s}}

% Equation for Re_d (total direct rebound)
\newcommand{\Redeqn}{\frac{ \left( \etaratiostacked \right)^{-\eqsps}
             \left( 1 + \frac{\rasub{N}}{\rbempl{M}} \right)^{\eqsM}   - 1}
         {\etaratiostacked - 1}}

% Equation for Re_prod
\newcommand{\Reprodeqn}{k p_E I_E - k Re_{cap} - k Re_{\OMd} - k p_E I_E Re_{dsub} - k Re_{isub}}

% Equation for Re_tot
\newcommand{\Retoteqn}{&Re_{emb} - k Re_{cap} + (1-k) Re_{\OMd}         \nonumber \\
                       &+ (1 - k p_E I_E) Re_{dsub} + (1 - k) Re_{isub}   \nonumber \\
                       &+ Re_{dinc} + Re_{iinc} +  k p_E I_E}
                 

%%%% Income preference equations

% Equation for energy service income preferences
\newcommand{\incprefseqn}{\frac{\rainc{q}_s}{\rbinc{q}_s} = \left( 1 + \frac{\rbinc{N}}{\Mdothatprime}  \right) ^{\eqsM}}

% Equation for other goods income preferences
\newcommand{\incprefoeqn}{\frac{\rainc{q}_o}{\rbinc{q}_o} = \left( 1 + \frac{\rbinc{N}}{\Mdothatprime}  \right) ^{\eqoM}}

% Equation for effective income
\newcommand{\effinceqn}{\Mdothatprime \equiv \rbempl{M} - \rbempl{C}_{cap} - \rbempl{C}_{md} 
                        - \rate{G} + p_E \Delta \rbinc{E}_s + \Delta \rbinc{C}_o}.
%%%%%%%%%%%%%%%%%%%%%%%%%%%%%%%%%%%%%%%%%%%%%%%%%%%%%%%%%%%%%%

%%%%% Decorations for symbols

\newcommand{\rate}[1]{\dot{#1}}                    % Rate of a quantity

% Create "after" commands
\newcommand{\orig}[1]{#1}
\newcommand{\aempl}[1]{{#1}^*}
\newcommand{\asub}[1]{\hat{#1}}
\newcommand{\ainc}[1]{\bar{#1}}
\newcommand{\aprod}[1]{\tilde{#1}}

% Create the "before" commands
\newcommand{\bempl}[1]{\orig{#1}}
\newcommand{\bsub}[1]{\aempl{#1}}
\newcommand{\binc}[1]{\asub{#1}}
\newcommand{\bprod}[1]{\ainc{#1}}

% Decoration combinations
% Rates after
\newcommand{\rorig}[1]{\orig{\rate{#1}}}
\newcommand{\raempl}[1]{\aempl{\rate{#1}}}
\newcommand{\rasub}[1]{\asub{\rate{#1}}}
\newcommand{\rainc}[1]{\ainc{\rate{#1}}}
\newcommand{\raprod}[1]{\aprod{\rate{#1}}}

% Rates before
\newcommand{\rbempl}[1]{\rorig{#1}}
\newcommand{\rbsub}[1]{\raempl{#1}}
\newcommand{\rbinc}[1]{\rasub{#1}}
\newcommand{\rbprod}[1]{\rainc{#1}}

%%%%% Subscript kerning

\newcommand{\OM}{O\!M}
% \newcommand{\OMd}{O\!M\!d}
\newcommand{\OMd}{md}
\newcommand{\productivity}{pr\!od}


%%%%% Convenient symbols

\newcommand{\Sdot}{\rate{S}_{dev}}
\newcommand{\Mdothatprime}{\rbinc{M}^\prime}


%%%%% Elasticities and income shares

\newcommand{\eqsps}{\epsilon_{\rate{q}_s\!,p_s}}
\newcommand{\eqops}{\epsilon_{\rate{q}_o\!,p_s}}
\newcommand{\eqsM}{\epsilon_{\rate{q}_s\!,\rate{M}}}
\newcommand{\eqoM}{\epsilon_{\rate{q}_o\!,\rate{M}}}
\newcommand{\eqspsUC}{\epsilon_{\rate{q}_s\!,p_s\!,U\!C}}
\newcommand{\eqopsUC}{\epsilon_{\rate{q}_o\!,p_s\!,U\!C}}
\newcommand{\fqs}{f_{\rbempl{q}_s}}

%%%%% Colors

\newcommand{\emplcolor}{red!25!white}
\newcommand{\subcolor}{orange!25!white}
\newcommand{\inccolor}{green!25!white}
\newcommand{\prodcolor}{blue!25!white}


%%%%% Coloration of text background

%
% Inline color box around text
% Arguments:
%   [#1]: background color for the box
%   {#2}: text inside the box
%
\newtcbox{\inlinebox}[1][]{on line, 
colback=#1,
colframe=#1,
before upper={\rule[-2pt]{0pt}{10pt}},
boxrule=1pt,
boxsep=0pt,
left=3pt,
right=3pt,
top=2pt,
bottom=2pt}


%%%%% Colored phrases

% Emplacement effect
\newcommand{\empleffect}{\inlinebox[\emplcolor]{emplacement effect}}
\newcommand{\empleffectadj}{\inlinebox[\emplcolor]{emplacement-effect}}
\newcommand{\Empleffect}{\inlinebox[\emplcolor]{Emplacement effect}}
\newcommand{\EmplEffect}{\inlinebox[\emplcolor]{Emplacement Effect}}

% Substitution effect
\newcommand{\subeffect}{\inlinebox[\subcolor]{substitution effect}}
\newcommand{\subeffectadj}{\inlinebox[\subcolor]{substitution-effect}}
\newcommand{\Subeffect}{\inlinebox[\subcolor]{Substitution effect}}
\newcommand{\SubEffect}{\inlinebox[\subcolor]{Substitution Effect}}

% Income effect
\newcommand{\inceffect}{\inlinebox[\inccolor]{income effect}}
\newcommand{\inceffectadj}{\inlinebox[\inccolor]{income-effect}}
\newcommand{\Inceffect}{\inlinebox[\inccolor]{Income effect}}
\newcommand{\IncEffect}{\inlinebox[\inccolor]{Income Effect}}

% Productivity effect
\newcommand{\prodeffect}{\inlinebox[\prodcolor]{productivity effect}}
\newcommand{\prodeffectadj}{\inlinebox[\prodcolor]{productivity-effect}}
\newcommand{\Prodeffect}{\inlinebox[\prodcolor]{Productivity effect}}
\newcommand{\ProdEffect}{\inlinebox[\prodcolor]{Productivity Effect}}



%%%%% minipage for assumptions and constraints tables
% Arguments:
%   #1: Width (multiple of \linewidth
%   #2: text inside the minipage
%
\newcommand{\mptable}[2]{\begin{minipage}{#1\linewidth} \useshortskip{} \begin{equation} #2 \end{equation} \end{minipage}}


%%%%% Oft-used references

\newcommand{\Ba}[1]{\citeauthor[#1]{Borenstein:2015aa}}
\newcommand{\Bapp}[1]{\citeauthor[#1]{Borenstein:2015aa}'s \citeyearpar{Borenstein:2015aa}}
\newcommand{\Bp}[1]{\citep[#1]{Borenstein:2015aa}}
\newcommand{\Bt}[1]{\citet[#1]{Borenstein:2015aa}}

\newcommand{\Ta}[1]{\citeauthor[#1]{Thomas:2013aa}}
\newcommand{\Tapp}[1]{\citeauthor[#1]{Thomas:2013aa}'s \citeyearpar{Thomas:2013aa}}
\newcommand{\Tp}[1]{\citep[#1]{Thomas:2013aa, Thomas:2013ab}}
\newcommand{\Tpone}[1]{\citep[#1]{Thomas:2013aa}}
\newcommand{\Tptwo}[1]{\citep[#1]{Thomas:2013ab}}
\newcommand{\Tt}[1]{\citet[#1]{Thomas:2013aa, Thomas:2013ab}}
\newcommand{\Ttone}[1]{\citet[#1]{Thomas:2013aa}}
\newcommand{\Tttwo}[1]{\citet[#1]{Thomas:2013ab}}


%%%%% Derivation pages

% Column widths
\newcommand{\derivtextsize}{\footnotesize}
\newcommand{\derivpageleftcolwidth}{0.11\textwidth}
\newcommand{\derivpageenergycolwidth}{0.7\textwidth}
\newcommand{\derivpagefinancialcolwidth}{0.7\textwidth}

% Horizontal rule between sections of derivations

\newcommand{\sectionsep}{\noindent\rule{1.6\textwidth}{0.4pt}}


%
% Derivation section
% Arguments:
%   #1: accounting stage (original, prime, etc.)
%   #2: energy column
%   #3: financial column
%
\newcommand{\derivsection}[3]{%

\derivtextsize{}

\begin{minipage}[t]{\derivpageleftcolwidth}
~\\#1
\end{minipage}
%
%
%
\begin{minipage}[t]{\derivpageenergycolwidth}
#2
\end{minipage}
%
~
%
\begin{minipage}[t]{\derivpagefinancialcolwidth}
#3
\end{minipage}

\normalsize{}

}


%
% Derivation page header
% Arguments:
%   #1: Effect type header text (e.g., Emplacement Effect)
%
\newcommand{\derivheader}[1]{

\begin{center}
  #1
\end{center}

\derivsection{}
{\begin{center}\emph{Energy analysis}\end{center}}
{\begin{center}\emph{Financial analysis}\end{center}}

}


% Equations

% Efficiency ratios
\newcommand{\etaratioinline}{\aprod{\eta}/\bempl{\eta}}
\newcommand{\etaratiostacked}{\frac{\aprod{\eta}}{\bempl{\eta}}}

% Derivative with respect to efficiency ratio
\newcommand{\dbydetaeta}{\frac{\mathrm{d}}{\mathrm{d}(\etaratioinline{})}}


% Original
\newcommand{\Eacctorig}{\rbempl{E} = \rbempl{E}_s + \rbempl{E}_{emb} + (\rbempl{C}_{\OMd} + \rbempl{C}_o) I_E}
\newcommand{\Macctorig}{\rbempl{M} = p_E \rbempl{E}_s + \rbempl{C}_{cap} + \rbempl{C}_{\OMd} + \rbempl{C}_o + \rbempl{N}}

% Before emplacement effect (same as original)
\newcommand{\Eacctbempl}{\Eacctorig}      
\newcommand{\Macctbempl}{\Macctorig}      

% After emplacement effect
\newcommand{\Eacctaempl}{\raempl{E} = \raempl{E}_s + \raempl{E}_{emb} + (\raempl{C}_{\OMd} + \raempl{C}_o) I_E}                  
\newcommand{\Macctaempl}{\raempl{M} = p_E \raempl{E}_s + \raempl{C}_{cap} + \raempl{C}_{\OMd} + \raempl{C}_o + \raempl{N}}         

% Before substitution effect (same as after emplacement effect)
\newcommand{\Eacctbsub}{\Eacctaempl}
\newcommand{\Macctbsub}{\Macctaempl}

% After substitution effect
\newcommand{\Eacctasub}{\rasub{E} = \rasub{E}_s + \rasub{E}_{emb} + (\rasub{C}_{\OMd} + \rasub{C}_o) I_E}
\newcommand{\Macctasub}{\rasub{M} = p_E \rasub{E}_s + \rasub{C}_{cap} + \rasub{C}_{\OMd} + \rasub{C}_o + \rasub{N}}

% Before income effect (same as after substitution effect)
\newcommand{\Eacctbinc}{\Eacctasub}
\newcommand{\Macctbinc}{\Macctasub}

% After income effect
\newcommand{\Eacctainc}{\rainc{E} = \rainc{E}_s + \rainc{E}_{emb} + (\rainc{C}_{\OMd} + \rainc{C}_o) I_E}
\newcommand{\Macctainc}{\rainc{M} = p_E \rainc{E}_s + \rainc{C}_{cap} + \rainc{C}_{\OMd} + \rainc{C}_o + \rainc{N}}

% Embodied energy rebound
\newcommand{\Reembeqn}{\frac{\left( \frac{\aempl{E}_{emb}}{\bempl{E}_{emb}}
  \frac{\bempl{t}}{\aempl{t}} - 1 \right) \rbempl{E}_{emb}}{\Sdot}}
  
% Ops, Maintenance, and disposal energy rebound
\newcommand{\ReOMdeqn}{\frac{\left( \frac{\raempl{C}_{\OMd}}{\rbempl{C}_{\OMd}} - 1 \right) \rbempl{C}_{\OMd} I_E}{\Sdot}}

% Equation for S_dot_dev
\newcommand{\Sdoteqn}{\left( \etaratiostacked - 1 \right)\!\etaratiostacked \rbempl{E}_s}

% Equation for Re_dsub
\newcommand{\Redsubeqn}{\frac{\left( \etaratiostacked \right)^{-\eqsps} - 1}
                        {\etaratiostacked - 1}}
                        
% Equation for Re_isub
\newcommand{\Reisubeqn}{\frac{{\left( \etaratiostacked  \right)}
                          ^{-\eqops} - 1}{\etaratiostacked - 1} \; 
                          \etaratiostacked \; 
                          \frac{\rbempl{C}_o I_E}{\rbempl{E}_s}}

% Equation for Re_dinc
\newcommand{\Redinceqn}{\frac{\left[ \left( 1 + \frac{\rbinc{N}}{\Mdothatprime} \right) ^{\eqsM} - 1  \right]}
              { \etaratiostacked - 1 } \left( \etaratiostacked \right)^{-\eqsps}}

% Equation for Re_cap
\newcommand{\Recapeqn}{\frac{\Delta \raempl{C}_{cap} I_E}{\Sdot}}

% Equation for Re_iinc
\newcommand{\Reiinceqn}{\frac{\left( 1 + \frac{\rbinc{N}}{\Mdothatprime} \right)^{\eqoM} - 1}{\etaratiostacked - 1} 
              \left( \etaratiostacked \right)^{-(1 + \eqops)}
              \frac{\rbempl{C}_o I_E}{\rbempl{E}_s}}

% Equation for Re_d (total direct rebound)
\newcommand{\Redeqn}{\frac{ \left( \etaratiostacked \right)^{-\eqsps}
             \left( 1 + \frac{\rasub{N}}{\rbempl{M}} \right)^{\eqsM}   - 1}
         {\etaratiostacked - 1}}

% Equation for Re_prod
\newcommand{\Reprodeqn}{k p_E I_E - k Re_{cap} - k Re_{\OMd} - k p_E I_E Re_{dsub} - k Re_{isub}}

% Equation for Re_tot
\newcommand{\Retoteqn}{&Re_{emb} - k Re_{cap} + (1-k) Re_{\OMd}         \nonumber \\
                       &+ (1 - k p_E I_E) Re_{dsub} + (1 - k) Re_{isub}   \nonumber \\
                       &+ Re_{dinc} + Re_{iinc} +  k p_E I_E}
                 

%%%% Income preference equations

% Equation for energy service income preferences
\newcommand{\incprefseqn}{\frac{\rainc{q}_s}{\rbinc{q}_s} = \left( 1 + \frac{\rbinc{N}}{\Mdothatprime}  \right) ^{\eqsM}}

% Equation for other goods income preferences
\newcommand{\incprefoeqn}{\frac{\rainc{q}_o}{\rbinc{q}_o} = \left( 1 + \frac{\rbinc{N}}{\Mdothatprime}  \right) ^{\eqoM}}

% Equation for effective income
\newcommand{\effinceqn}{\Mdothatprime \equiv \rbempl{M} - \rbempl{C}_{cap} - \rbempl{C}_{md} 
                        - \rate{G} + p_E \Delta \rbinc{E}_s + \Delta \rbinc{C}_o}.
%%%%%%%%%%%%%%%%%%%%%%%%%%%%%%%%%%%%%%%%%%%%%%%%%%%%%%%%%%%%%%

%%%%% Decorations for symbols

\newcommand{\rate}[1]{\dot{#1}}                    % Rate of a quantity

% Create "after" commands
\newcommand{\orig}[1]{{}{#1}^{\scriptscriptstyle \circ}}
\newcommand{\aempl}[1]{{#1}^*}
\newcommand{\asub}[1]{\hat{#1}}
\newcommand{\ainc}[1]{\bar{#1}}
\newcommand{\aprod}[1]{\tilde{#1}}

% Create the "before" commands
\newcommand{\bempl}[1]{\orig{#1}}
\newcommand{\bsub}[1]{\aempl{#1}}
\newcommand{\binc}[1]{\asub{#1}}
\newcommand{\bprod}[1]{\ainc{#1}}

% Decoration combinations
% Rates after
\newcommand{\rorig}[1]{\orig{\rate{#1}}}
\newcommand{\raempl}[1]{\aempl{\rate{#1}}}
\newcommand{\rasub}[1]{\asub{\rate{#1}}}
\newcommand{\rainc}[1]{\ainc{\rate{#1}}}
\newcommand{\raprod}[1]{\aprod{\rate{#1}}}

% Rates before
\newcommand{\rbempl}[1]{\rorig{#1}}
\newcommand{\rbsub}[1]{\raempl{#1}}
\newcommand{\rbinc}[1]{\rasub{#1}}
\newcommand{\rbprod}[1]{\rainc{#1}}

%%%%% Subscript kerning

% \newcommand{\OM}{O\!M}
% \newcommand{\md}{O\!M\!d}
\newcommand{\md}{md}
\newcommand{\productivity}{pr\!od}
\newcommand{\life}{li\!f\!e}


%%%%% Convenient symbols

\newcommand{\Sdot}{\rate{S}_{dev}}
\newcommand{\Mdothatprime}{\rbinc{M}^\prime}


%%%%% Elasticities and income shares

\newcommand{\eqsps}{\epsilon_{\rate{q}_s\!,p_s}}
\newcommand{\eqops}{\epsilon_{\rate{q}_o\!,p_s}}
\newcommand{\eqsM}{\epsilon_{\rate{q}_s\!,\rate{M}}}
\newcommand{\eqoM}{\epsilon_{\rate{q}_o\!,\rate{M}}}
\newcommand{\eqspsUC}{\epsilon_{\rate{q}_s\!,p_s\!,U\!C}}
\newcommand{\eqopsUC}{\epsilon_{\rate{q}_o\!,p_s\!,U\!C}}
\newcommand{\fCs}{\bempl{f}_{\rate{C}_s}}

%%%%% Colors

% Original spectrum colours
% \colorlet{emplcolor}{red!25!white}
% \colorlet{subcolor}{orange!25!white}
% \colorlet{inccolor}{green!25!white}
% \colorlet{prodcolor}{blue!25!white}

% New Viridis "plasma" colours
\definecolor{emplcolor}{HTML}{150789}
\definecolor{subcolor}{HTML}{99149F}
\definecolor{inccolor}{HTML}{E76F5A}
\definecolor{prodcolor}{HTML}{F7E225}



%%%%% Coloration of text background

%
% Inline color box around text
% Arguments:
%   [#1]: background color for the box
%   {#2}: text inside the box
%
\newtcbox{\inlinebox}[1][]{on line, 
colback=#1,
colframe=#1,
before upper={\rule[-2pt]{0pt}{10pt}},
boxrule=1pt,
boxsep=0pt,
left=3pt,
right=3pt,
top=2pt,
bottom=2pt}


%%%%% Colored phrases

% Emplacement effect
\newcommand{\empleffect}{\inlinebox[emplcolor]{\textcolor{white}{emplacement effect}}}
\newcommand{\empleffectadj}{\inlinebox[emplcolor]{\textcolor{white}{emplacement-effect}}}
\newcommand{\Empleffect}{\inlinebox[emplcolor]{\textcolor{white}{Emplacement effect}}}
\newcommand{\EmplEffect}{\inlinebox[emplcolor]{\textcolor{white}{Emplacement Effect}}}

% Substitution effect
\newcommand{\subeffect}{\inlinebox[subcolor]{\textcolor{white}{substitution effect}}}
\newcommand{\subeffectadj}{\inlinebox[subcolor]{\textcolor{white}{substitution-effect}}}
\newcommand{\Subeffect}{\inlinebox[subcolor]{\textcolor{white}{Substitution effect}}}
\newcommand{\SubEffect}{\inlinebox[subcolor]{\textcolor{white}{Substitution Effect}}}

% Income effect
\newcommand{\inceffect}{\inlinebox[inccolor]{\textcolor{black}{income effect}}}
\newcommand{\inceffectadj}{\inlinebox[inccolor]{\textcolor{black}{income-effect}}}
\newcommand{\Inceffect}{\inlinebox[inccolor]{\textcolor{black}{Income effect}}}
\newcommand{\IncEffect}{\inlinebox[inccolor]{\textcolor{black}{Income Effect}}}

% Productivity effect
\newcommand{\prodeffect}{\inlinebox[prodcolor]{\textcolor{black}{productivity effect}}}
\newcommand{\prodeffectadj}{\inlinebox[prodcolor]{\textcolor{black}{productivity-effect}}}
\newcommand{\Prodeffect}{\inlinebox[prodcolor]{\textcolor{black}{Productivity effect}}}
\newcommand{\ProdEffect}{\inlinebox[prodcolor]{\textcolor{black}{Productivity Effect}}}


%%%%% minipage for assumptions and constraints tables
% Arguments:
%   #1: Width (multiple of \linewidth
%   #2: text inside the minipage
%
\newcommand{\mptable}[2]{\begin{minipage}{#1\linewidth} \useshortskip{} \begin{equation} #2 \end{equation} \end{minipage}}


%%%%% Oft-used references

\newcommand{\Ba}[1]{\citeauthor[#1]{Borenstein:2015aa}}
\newcommand{\Bapp}[1]{\citeauthor[#1]{Borenstein:2015aa}'s \citeyearpar{Borenstein:2015aa}}
\newcommand{\Bp}[1]{\citep[#1]{Borenstein:2015aa}}
\newcommand{\Bt}[1]{\citet[#1]{Borenstein:2015aa}}

\newcommand{\Ta}[1]{\citeauthor[#1]{Thomas:2013aa}}
\newcommand{\Tapp}[1]{\citeauthor[#1]{Thomas:2013aa}'s \citeyearpar{Thomas:2013aa}}
\newcommand{\Tp}[1]{\citep[#1]{Thomas:2013aa, Thomas:2013ab}}
\newcommand{\Tpone}[1]{\citep[#1]{Thomas:2013aa}}
\newcommand{\Tptwo}[1]{\citep[#1]{Thomas:2013ab}}
\newcommand{\Tt}[1]{\citet[#1]{Thomas:2013aa, Thomas:2013ab}}
\newcommand{\Ttone}[1]{\citet[#1]{Thomas:2013aa}}
\newcommand{\Tttwo}[1]{\citet[#1]{Thomas:2013ab}}


%%%%% Derivation pages

% Column widths
\newcommand{\derivtextsize}{\footnotesize}
\newcommand{\derivpageleftcolwidth}{0.11\textwidth}
\newcommand{\derivpageenergycolwidth}{0.7\textwidth}
\newcommand{\derivpagefinancialcolwidth}{0.7\textwidth}

% Horizontal rule between sections of derivations

\newcommand{\sectionsep}{\noindent\rule{1.6\textwidth}{0.4pt}}


%
% Derivation section
% Arguments:
%   #1: accounting stage (original, prime, etc.)
%   #2: energy column
%   #3: financial column
%
\newcommand{\derivsection}[3]{%

\derivtextsize{}

\begin{minipage}[t]{\derivpageleftcolwidth}
~\\#1
\end{minipage}
%
%
%
\begin{minipage}[t]{\derivpageenergycolwidth}
#2
\end{minipage}
%
~
%
\begin{minipage}[t]{\derivpagefinancialcolwidth}
#3
\end{minipage}

\normalsize{}

}


%
% Derivation page header
% Arguments:
%   #1: Effect type header text (e.g., Emplacement Effect)
%
\newcommand{\derivheader}[1]{

\begin{center}
  #1
\end{center}

\derivsection{}
{\begin{center}\emph{Energy analysis}\end{center}}
{\begin{center}\emph{Financial analysis}\end{center}}

}


% Equations

% Efficiency ratios
\newcommand{\etaratioinline}{\aprod{\eta}/\bempl{\eta}}
\newcommand{\etaratiostacked}{\frac{\aprod{\eta}}{\bempl{\eta}}}

% Derivative with respect to efficiency ratio
\newcommand{\dbydetaeta}{\frac{\mathrm{d}}{\mathrm{d}(\etaratioinline{})}}


% Original
\newcommand{\Eacctorig}{\rbempl{E} = \rbempl{E}_s + \rbempl{E}_{emb} + (\rbempl{C}_{\md} + \rbempl{C}_o) I_E}
\newcommand{\Macctorig}{\rbempl{M} = p_E \rbempl{E}_s + \rbempl{C}_{cap} + \rbempl{C}_{\md} + \rbempl{C}_o + \rbempl{N}}

% Before emplacement effect (same as original)
\newcommand{\Eacctbempl}{\Eacctorig}      
\newcommand{\Macctbempl}{\Macctorig}      

% After emplacement effect
\newcommand{\Eacctaempl}{\raempl{E} = \raempl{E}_s + \raempl{E}_{emb} + (\raempl{C}_{\md} + \raempl{C}_o) I_E}                  
\newcommand{\Macctaempl}{\raempl{M} = p_E \raempl{E}_s + \raempl{C}_{cap} + \raempl{C}_{\md} + \raempl{C}_o + \raempl{N}}         

% Before substitution effect (same as after emplacement effect)
\newcommand{\Eacctbsub}{\Eacctaempl}
\newcommand{\Macctbsub}{\Macctaempl}

% After substitution effect
\newcommand{\Eacctasub}{\rasub{E} = \rasub{E}_s + \rasub{E}_{emb} + (\rasub{C}_{\md} + \rasub{C}_o) I_E}
\newcommand{\Macctasub}{\rasub{M} = p_E \rasub{E}_s + \rasub{C}_{cap} + \rasub{C}_{\md} + \rasub{C}_o + \rasub{N}}

% Before income effect (same as after substitution effect)
\newcommand{\Eacctbinc}{\Eacctasub}
\newcommand{\Macctbinc}{\Macctasub}

% After income effect
\newcommand{\Eacctainc}{\rainc{E} = \rainc{E}_s + \rainc{E}_{emb} + (\rainc{C}_{\md} + \rainc{C}_o) I_E}
\newcommand{\Macctainc}{\rainc{M} = p_E \rainc{E}_s + \rainc{C}_{cap} + \rainc{C}_{\md} + \rainc{C}_o + \rainc{N}}

% Embodied energy rebound
\newcommand{\Reembeqn}{\frac{\left( \frac{\aempl{E}_{emb}}{\bempl{E}_{emb}}
  \frac{\bempl{t}_{\life}}{\aempl{t}_{\life}} - 1 \right) \rbempl{E}_{emb}}{\Sdot}}
  
% Ops, Maintenance, and disposal energy rebound
\newcommand{\ReOMdeqn}{\frac{\left( \frac{\raempl{C}_{\md}}{\rbempl{C}_{\md}} - 1 \right) \rbempl{C}_{\md} I_E}{\Sdot}}

% Equation for S_dot_dev
% \newcommand{\Sdoteqn}{\left( \etaratiostacked - 1 \right)\!\etaratiostacked \rbempl{E}_s}
\newcommand{\Sdoteqn}{\left( \etaratiostacked - 1 \right)\! 
                            \frac{\bempl{\eta}}{\aprod{\eta}} \rbempl{E}_s}

% Equation for Re_dsub
\newcommand{\Redsubeqn}{\frac{\left( \etaratiostacked \right)^{-\eqsps} - 1}
                        {\etaratiostacked - 1}}
                        
% Equation for Re_isub
\newcommand{\Reisubeqn}{\frac{{\left( \etaratiostacked  \right)}
                          ^{-\eqops} - 1}{\etaratiostacked - 1} \; 
                          \etaratiostacked \; 
                          \frac{\rbempl{C}_o I_E}{\rbempl{E}_s}}

% Equation for Re_dinc
\newcommand{\Redinceqn}{\frac{ \left( 1 + \frac{\rbinc{N}}{\Mdothatprime} \right) ^{\eqsM} - 1}
              { \etaratiostacked - 1 } \left( \etaratiostacked \right)^{-\eqsps}}

% Equation for Re_cap
\newcommand{\Recapeqn}{\frac{\Delta \raempl{C}_{cap} I_E}{\Sdot}}

% Equation for Re_iinc
\newcommand{\Reiinceqn}{\frac{\left( 1 + \frac{\rbinc{N}}{\Mdothatprime} \right)^{\eqoM} - 1}{\etaratiostacked - 1} 
              \left( \etaratiostacked \right)^{1 - \eqops}
              \frac{\rbempl{C}_o I_E}{\rbempl{E}_s}}

% Equation for Re_d (total direct rebound)
\newcommand{\Redeqn}{\frac{ \left( \etaratiostacked \right)^{-\eqsps}
             \left( 1 + \frac{\rasub{N}}{\rbempl{M}} \right)^{\eqsM}   - 1}
         {\etaratiostacked - 1}}

% Equation for Re_prod
\newcommand{\Reprodeqn}{k p_E I_E - k Re_{cap} - k Re_{\md} - k p_E I_E Re_{dsub} - k Re_{isub}}

% Equation for Re_tot
\newcommand{\Retoteqn}{&Re_{emb} - k Re_{cap} + (1-k) Re_{\md}         \nonumber \\
                       &+ (1 - k p_E I_E) Re_{dsub} + (1 - k) Re_{isub}   \nonumber \\
                       &+ Re_{dinc} + Re_{iinc} +  k p_E I_E}
                 

%%%% Income preference equations

% Equation for energy service income preferences
\newcommand{\incprefseqn}{\frac{\rainc{q}_s}{\rbinc{q}_s} = \left( 1 + \frac{\rbinc{N}}{\Mdothatprime}  \right) ^{\eqsM}}

% Equation for other goods income preferences
\newcommand{\incprefoeqn}{\frac{\rainc{q}_o}{\rbinc{q}_o} = \left( 1 + \frac{\rbinc{N}}{\Mdothatprime}  \right) ^{\eqoM}}

% Equation for effective income
\newcommand{\effinceqn}{\Mdothatprime \equiv \rbempl{M} - \rbempl{C}_{cap} - \rbempl{C}_{md} 
                        - \rate{G} + p_E \Delta \rbinc{E}_s + \Delta \rbinc{C}_o}
                        
% CES utility equation
\newcommand{\cesutility}{  \frac{\rate{u}}{\rbempl{u}} = 
  \left[ \fCs \left( \frac{\rate{q}_s}{\rbempl{q}_s} \right)^\rho 
        + (1-\fCs) \left( \frac{\rate{C}_o}{\rbempl{C}_o} \right)^\rho  \right]^{(1/\rho)}}


% Segments and lines
% Arguments:
%   #1: left character
%   #2: line color
%   #3: line thickness (e.g., 0.1 mm)
%   #4: right character
% Note that \raisebox{0.9 mm} moves the line up from the baseline.
% Also, \line(1,0){12} gives a horizontal line with length "12" (unknown units!)
\newcommand{\seg}[4]{#1\linethickness{#3}\raisebox{0.8 mm}{\textcolor{#2}{\line(1,0){12}}}#4}

% Construction lines
\newcommand{\iicirc}{\seg{$\bempl{\text{i}}$}{black}{0.3 mm}{$\bempl{\text{i}}$}}
\newcommand{\rr}{\seg{r}{black}{0.1 mm}{r}}
\newcommand{\circcirc}{\seg{$\circ$}{black}{0.1 mm}{$\circ$}}
\newcommand{\starstar}{\seg{$*$}{black}{0.1 mm}{$*$}}
\newcommand{\hathat}{\seg{$\wedge$}{black}{0.1 mm}{$\wedge$}}

% Line segments
\newcommand{\circa}{\seg{$\circ$}{emplcolor}{0.6 mm}{a}}
\newcommand{\ab}{\seg{a}{emplcolor}{0.6 mm}{b}}
\newcommand{\bstar}{\seg{b}{emplcolor}{0.6 mm}{$*$}}
\newcommand{\starc}{\seg{$*$}{subcolor}{0.6 mm}{$\,$c}}
\newcommand{\chat}{\seg{c$\,$}{subcolor}{0.6 mm}{$\wedge$}}








