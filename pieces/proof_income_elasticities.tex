% The next command tells RStudio to do "Compile PDF" on HSB.Rnw,
% instead of this file, thereby eliminating the need to switch back to HSB.Rnw 
% before building the paper.
%!TEX root = ../HSB.Rnw


After the substitution effect, 
a rate of net income is available ($\rasub{N}$), 
all of which is spent on
additional energy service ($\Delta \rainc{q}_s$, $\Delta \rainc{C}_s = p_E \Delta \rainc{E}_s$) or 
additional other goods ($\Delta \rainc{q}_o$, $\Delta \rainc{C}_o$).
The income effect must satisfy the budget constraint
such that net income is zero afterward ($\rainc{N} = 0$).
The budget constraint across the income effect 
is represented by Eq.~(\ref{eq:inc_budget_constraint}): 

\begin{equation}
  \rbinc{N} = p_E \Delta \rainc{E}_s + \Delta \rainc{C}_o \; . \tag{\ref{eq:inc_budget_constraint}}
\end{equation}

The additional spending due to the income effect is given by income preference equations

\begin{equation}
  \frac{\rainc{q}_s}{\rbinc{q}_s} = \left( 1 + \frac{\rbinc{N}}{\Mdothatprime}  \right) ^{\eqsM} 
                                                                \tag{\ref{eq:qsrat_eqsM}}
\end{equation}
%
and

\begin{equation}
  \frac{\rainc{q}_o}{\rbinc{q}_o} = \left( 1 + \frac{\rbinc{N}}{\Mdothatprime}  \right) ^{\eqoM} \; ,
                                                                \tag{\ref{eq:qorat_eqoM}}
\end{equation}
%
where

\begin{equation}
  \Mdothatprime \equiv \rbempl{M} - \rbempl{C}_{cap} - \rbempl{C}_{md} 
                       - \rate{G} + p_E \Delta \rbinc{E}_s + \Delta \rbinc{C}_o \; .
                                                                \tag{\ref{eq:effective_income}}
\end{equation}
%
This appendix proves that the income preference equations 
(Eqs.~(\ref{eq:qsrat_eqsM}) and~(\ref{eq:qorat_eqoM}))
satisfy the budget constraint (Eq.~(\ref{eq:inc_budget_constraint})).

The first step in the proof is to convert 
the income preference equations
to $\rbempl{C}_s$ and $\rbempl{C}_o$ ratios.
For the energy service income preference equation (Eq.~(\ref{eq:qsrat_eqsM})), 
multiply numerator and denominator of the left-hand side by $\aprod{p}_s = p_E / \aprod{\eta}$
(Eq.~(\ref{eq:ps_pE_eta}))
to obtain $\rainc{C}_s / \rbinc{C}_s$.
For the other goods income preference equation (Eq.~(\ref{eq:qorat_eqoM})), 
multiply numerator and denominator of the left-hand side by $p_o$
to obtain $\rainc{C}_o / \rbinc{C}_o$.
Then invoke homotheticity to set $\eqsM = 1$ and $\eqoM = 1$ to obtain

\begin{equation}
    \frac{\rainc{C}_s}{\rbinc{C}_s} = 1 + \frac{\rbinc{N}}{\Mdothatprime}
\end{equation}
%
and

\begin{equation}
  \frac{\rainc{C}_o}{\rbinc{C}_o} = 1 + \frac{\rbinc{N}}{\Mdothatprime} \; .
\end{equation}
%

The second step in the proof is to obtain expressions 
for $\Delta \rainc{C}_s$ and $\Delta \rainc{C}_o$.
Multiply the income preference equations above
by $\Delta \rbinc{C}_s$ and $\Delta \rbinc{C}_o$, respectively.
Then subtract $\Delta \rbinc{C}_s$ and $\Delta \rbinc{C}_o$, respectively, 
to obtain

\begin{equation}
  \Delta \rainc{C}_s = \frac{\rbinc{C}_s}{\Mdothatprime} \rbinc{N}
\end{equation}
%
and

\begin{equation}
  \Delta \rainc{C}_o = \frac{\rbinc{C}_o}{\Mdothatprime} \rbinc{N} \; .
\end{equation}

The above versions of the income preference equations 
can be substituted into the budget constraint
(Eq.~(\ref{eq:inc_budget_constraint})) to obtain

\begin{equation} \label{eq:inc_elasticity_proof_setup}
  \rbinc{N} \stackrel{?}{=} \frac{\rbinc{C}_s}{\Mdothatprime} \rbinc{N} 
                            + \frac{\rbinc{C}_o}{\Mdothatprime}  \rbinc{N} \; .
\end{equation}
%
If equality is demonstrated, 
the income preference equations satisfy the budget constraint.
The remainder of the proof shows the equality
of Eq.~(\ref{eq:inc_elasticity_proof_setup}).

Dividing by $\rbinc{N}$ and multiplying by $\Mdothatprime$ gives

\begin{equation}
  \rbinc{C}_s + \rbinc{C}_o \stackrel{?}{=} \Mdothatprime \; .
\end{equation}
%
Substituting Eq.~(\ref{eq:effective_income}) for $\Mdothatprime$ gives

\begin{equation}
  \rbinc{C}_s + \rbinc{C}_o \stackrel{?}{=} \rbempl{M} - \rbempl{C}_{cap} - \rbempl{C}_{md} 
                       - \rate{G} + p_E \Delta \rbinc{E}_s + \Delta \rbinc{C}_o \; .
\end{equation}
%
Expanding differences yields

\begin{equation}
  \rbinc{C}_s + \rbinc{C}_o \stackrel{?}{=} \rbempl{M} - \rbempl{C}_{cap} - \rbempl{C}_{\md} - \rate{G} 
            + p_E \rasub{E}_s - p_E \rbsub{E}_s + \rasub{C}_o - \rbsub{C}_o \; .
\end{equation}
%
Recognizing that $\raempl{M} = \rbempl{M}$ and 
substituting Eq.~(\ref{eq:before_substitution_financial}) for $\rbsub{M}$
leads to 

\begin{align}
  \rbinc{C}_s + \rbinc{C}_o \stackrel{?}{=} \; & \cancel{p_E \rbsub{E}_s} + \rbsub{C}_{cap} + \rbsub{C}_{md}
                                              + \cancel{\rbsub{C}_o} + \rbsub{N} \nonumber \\
                                            & - \rbempl{C}_{cap} - \rbempl{C}_{\md} - \rate{G}
                                              + p_E \rasub{E}_s - \cancel{p_E \rbsub{E}_s}
                                              + \rasub{C}_o - \cancel{\rbsub{C}_o} \; .
\end{align}
%
Eliminating terms, substituting Eq.~(\ref{eq:G_dot}) for $\rate{G}$, and 
forming difference terms $\Delta \rbsub{C}_{cap}$ and $\Delta \rbsub{C}_{\md}$ gives

\begin{equation}
  \rbinc{C}_s + \rbinc{C}_o \stackrel{?}{=} p_E \rasub{E}_s + \rasub{C}_o
                                            + \Delta \rbsub{C}_{cap} + \Delta \rbsub{C}_{\md}
                                            - p_E \Sdot + \rbsub{N} \; .
\end{equation}
%
Substituting Eq.~(\ref{eq:N_dot_star_empl}) for $\raempl{N}$ gives 

\begin{align}
  \rbinc{C}_s + \rbinc{C}_o \stackrel{?}{=} p_E \rasub{E}_s + \rasub{C}_o
        & + \cancel{\Delta \rbsub{C}_{cap}} + \cancel{\Delta \rbsub{C}_{\md}} - \cancel{p_E \Sdot} \nonumber \\
        & - \cancel{\Delta \rbsub{C}_{cap}} - \cancel{\Delta \rbsub{C}_{\md}} + \cancel{p_E \Sdot} \; .
\end{align}
%
Finally, eliminating terms and recognizing that $p_E \rbinc{E}_s = \rbinc{C}_s$ leads to

\begin{equation}
  \rbinc{C}_s + \rbinc{C}_o \stackrel{\checkmark}{=} \rbinc{C}_s + \rbinc{C}_o \; ,
\end{equation}
%
thereby completing the proof that the income preference equations
(Eqs.~(\ref{eq:qsrat_eqsM}) and~(\ref{eq:qorat_eqoM}))
satisfy the budget constraint
(Eq.~(\ref{eq:inc_budget_constraint})).
