% The next command tells RStudio to do "Compile PDF" on HSB.Rnw,
% instead of this file, thereby eliminating the need to switch back to HSB.Rnw 
% before building the paper.
%!TEX root = ../HSB.Rnw


This rebound framework utilizes
a partial equilibrium analysis
in which we account for the change of the energy service price
due to the EEU ($\orig{p}_s \neq \aempl{p}_s$), but 
we do not track the effect of the EEU on prices of other goods.
These assumptions have important implications for the relationship between 
the rate of consumption of other goods ($\rate{q}_o$) and 
the rate of expenditure on other goods ($\rate{C}_o$).

We assume a basket of other goods (besides the energy service) 
purchased in the economy,
each ($i$) with its own price ($p_{o,i}$) and rate of consumption ($\rate{q}_{o,i}$),
such that the average price of all other goods purchased in the economy
prior to the EEU~($\orig{p}_o$) is given by

\begin{equation}
  \orig{p}_o = \frac{\sum\limits_i \orig{p}_{o,i} \orig{q}_{o,i}}{\sum\limits_i \orig{q}_{o,i}} \; .
\end{equation}
%
Then, the expenditure rate of other purchases in the economy can be given as

\begin{equation}
  \rorig{C}_o = \orig{p}_o \rorig{q}_o
\end{equation}
%
before the EEU and

\begin{equation}
  \rasub{C}_o = \asub{p}_o \rasub{q}_o \; 
\end{equation}
%
after the substitution effect, for example.

We assume that any effects (emplacement, substitution, or income)
for a single device 
are not so large that they 
cause a measurable change in prices of other goods. 
Thus, 

\begin{equation}
  \orig{p}_o = \aempl{p}_o = \asub{p}_o = \ainc{p}_o = \amacro{p}_o \; .
\end{equation}

In the partial equilibrium analysis, 
two other goods prices can be equated 
across any rebound effect
to obtain
(for the example of the original conditions ($\circ$) 
and the post-substitution state ($\wedge$))

\begin{equation} \label{eq:qo_Co_equality}
  \frac{\rasub{C}_o}{\rorig{C}_o} 
      = \frac{\rasub{q}_o}{\rorig{q}_o} \; .
\end{equation}

Thus, a ratio of other goods expenditure rates
is always equal to a ratio of other goods consumption rates.









