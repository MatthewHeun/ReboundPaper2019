% The next command tells RStudio to do "Compile PDF" on HSB.Rnw,
% instead of this file, thereby eliminating the need to switch back to HSB.Rnw 
% before building the paper.
%!TEX root = ../HSB.Rnw

Rebound path graphs show the impact of direct and indirect rebound effects
in energy space, expenditure space, and preference space.
Notional rebound path graphs can be found in 
Figs.~\ref{fig:ExampleEnergyPathGraph}--\ref{fig:ExamplePrefsPathGraph}.
Rebound path graphs for the car example can be found in 
Figs.~\ref{fig:CarEnergyGraph}--\ref{fig:CarPrefsGraph}.
Graphs for the lamp example can be found in
Figs.~\ref{fig:LampEnergyGraph}--\ref{fig:LampPrefsGraph}.

This appendix shows the mathematical details of rebound path graphs,
specifically derivations of equations for lines and curves 
shown in Table~\ref{tab:lines_and_curves}.
The lines and curves enable construction of numerically accurate
rebound path graphs 
as shown in Figs.~\ref{fig:CarEnergyGraph}--\ref{fig:LampPrefsGraph}.

\begin{table}
\centering
\caption{Lines and curves for rebound path graphs.}
\label{tab:lines_and_curves}
\begin{tabular}{rl}
\toprule
Rebound path graph           & Lines and curves                        \\ 
\midrule
\multirow{2}{*}{Energy}      & Constant total energy consumption lines \\
                             & 0\% and 100\% rebound lines             \\
\midrule
Expenditure                  & Constant expenditure lines              \\
\midrule
\multirow{3}{*}{Preferences} & Constant expenditure lines              \\
                             & Rays from origin to $\wedge$ point      \\
                             & Indifference curves                     \\
\bottomrule
\end{tabular}
\end{table}


%++++++++++++++++++++++++++++++
\subsection{Energy path graphs}
\label{sec:energy_path_graph_details}
%++++++++++++++++++++++++++++++

Energy path graphs show direct (on the abscissa) and indirect (on the ordinate)
energy consumption associated with the energy conversion device 
and the device owner.
Lines of constant total energy consumption comprise a 
scale for total rebound.
E.g., the 0\% and 100\% rebound lines are constant total energy consumption
lines which pass through the original point ($\circ$) and
the post-direct-emplacement-effect point ($a$) 
on an energy path graph.

The equation of a constant total energy consumption line is derived from 

\begin{equation}
  \rate{E}_{tot} = \rate{E}_{dir} + \rate{E}_{indir}
\end{equation}
%
at any rebound stage. (See Fig.~\ref{fig:flowchart}.)
Direct energy consumption is energy consumed by the energy conversion device
($\rate{E}_s$), and 
indirect energy consumption is the sum of embodied energy, 
energy associated with maintenenace and disposal, and energy associated 
with expenditures on other goods
($\rate{E}_{emb} + (\rate{C}_{\md} + \rate{C}_o) I_E$).

For the energy path graph, 
indirect energy consumption is placed on the ordinate
and direct energy consumption is placed on the abscissa.
To derive the equation of a constant energy consumption line, 
we first rearrange to put the $y$ coordinate on the left of the equation:

\begin{equation}
  \rate{E}_{indir} = - \rate{E}_{dir} + \rate{E}_{tot} \; .
\end{equation}
%
Next, we substitute $y$ for $\rate{E}_{indir}$,
$x$ for $\rate{E}_{dir}$, and 
$\rate{E}_s + \rate{E}_{emb} + (\rate{C}_{\md} + \rate{C}_o) I_E$ for $\rate{E}_{tot}$
to obtain

\begin{equation}
  y = -x + \rate{E}_s + \rate{E}_{emb} + (\rate{C}_{\md} + \rate{C}_o) I_E \; ,
\end{equation}
%
where all of $\rate{E}_s$, $\rate{E}_{emb}$, $\rate{C}_{\md}$, and $\rate{C}_o$
apply at the same rebound stage of Fig.~\ref{fig:flowchart}.

The constant total energy consumption line 
that passes through the original point ($\circ$)
shows 100\% rebound:

\begin{equation}
  y = -x + \rbempl{E}_s + \rbempl{E}_{emb} + (\rbempl{C}_{\md} + \rbempl{C}_o) I_E \; .
\end{equation}

The 0\% rebound line is the constant total energy consumption line 
that accounts for expected energy savings ($\Sdot$) only:

\begin{equation}
  y = -x + (\rbempl{E}_s - \Sdot)
          + \rbempl{E}_{emb} + (\rbempl{C}_{\md} + \rbempl{C}_o) I_E \; .
\end{equation}
%
The above line passes through the $a$ point on an energy path graph.


%++++++++++++++++++++++++++++++
\subsection{Expenditure path graphs}
\label{sec:expenditure_path_graph_details}
%++++++++++++++++++++++++++++++

Expenditure path graphs show direct (on the abscissa) and indirect (on the ordinate)
expenses associated with the energy conversion device 
and the device owner.
Lines of constant expenditure are important, 
because they provide budget constraints for the device owner.

The equation of a constant total expenditure line is derived from 
the budget constraint

\begin{equation}
  \rate{C}_{tot} = \rate{C}_{dir} + \rate{C}_{indir}
\end{equation}
%
at any rebound stage.
For the expenditure path graph,
indirect expenditures are placed on the ordinate
and direct expenditures on energy for the energy conversion device are place on the abscissa.
Direct expenditure is the cost of energy consumed by the energy conversion device
($\rate{C}_s = p_E \rate{E}_s$), and 
indirect expenses are the sum of capital costs, 
maintenenace and disposal costs, and 
expenditures on other goods
($\rate{C}_{cap} + \rate{C}_{\md} + \rate{C}_o$).
Rearranging to put the $y$-axis variable on the left side of the equation gives

\begin{equation}
  \rate{C}_{indir} = - \rate{C}_{dir} + \rate{C}_{tot} \; .
\end{equation}

Substituting $y$ for $\rate{C}_{indir}$, 
$x$ for $\rate{C}_{dir}$, and 
$\rate{C}_s + \rate{C}_{cap} + \rate{C}_{\md} + \rate{C}_o$ for $\rate{C}_{tot}$
gives

\begin{equation}
  y = -x + \rate{C}_s + \rate{C}_{cap} + \rate{C}_{\md} + \rate{C}_o \; ,
\end{equation}
%
where all of $\rate{C}_s$, $\rate{C}_{cap}$, $\rate{C}_{\md}$, and $\rate{C}_o$
apply at the same rebound stage of Fig.~\ref{fig:flowchart}.

The constant total expenditure line 
that passes through the original point ($\circ$)
shows the budget constraint for the device owner:

\begin{equation}
  y = -x + \rbempl{C}_s + \rbempl{C}_{cap} + \rbempl{C}_{\md} + \rbempl{C}_o \; ,
\end{equation}
%
or

\begin{equation}
  y = -x + \rbempl{M} \; .
\end{equation}

The constant total expenditure line 
that accounts for expected energy savings ($\Sdot$) 
and gross costs savings ($\rate{G} = p_E \Sdot$) only 
is given by:

\begin{equation}
  y = -x + (\rbempl{C}_s - \rate{G}) + \rbempl{C}_{cap} + \rbempl{C}_{\md} + \rbempl{C}_o \; ,
\end{equation}
%
or

\begin{equation}
  y = -x + \rbempl{M} - \rate{G}\; .
\end{equation}
%
The line given by the above equation
passes through the $a$ point on an expenditure path graph.


%++++++++++++++++++++++++++++++
\subsection{Preference path graphs}
\label{sec:prefs_path_graph_details}
%++++++++++++++++++++++++++++++

Preferences path graphs show expenditures in 
$\rate{C}_o/\rbempl{C}_o$ vs.\ $\rate{q}_s/\rbempl{q}_s$ space
to accord with the utilty model.
(See Appendix~\ref{sec:utility_model}.)
Preference path graphs include 
%
\begin{enumerate*}[label={(\alph*)}]
	
  \item constant expenditure lines,
  
  \item a ray from the origin through the $\wedge$ point, and 
  
  \item indifference curves.
    
\end{enumerate*}
%
Derivations for each are shown in the following subsections.


%------------------------------
\subsubsection{Constant expenditure lines} 
\label{sec:pref_graph_constant_expenditure_lines}
%------------------------------

Like the expenditure path graph, 
lines of constant expenditure on a preferences path graph 
are derived from the budget constraint of the device owner.

Before the EEU, the budget constraint is

\begin{equation}
  \rbempl{M} = \bempl{p}_s \rate{q}_s + \rbempl{C}_{cap} + \rbempl{C}_{\md} + \rate{C}_o \; .
\end{equation}
%
Note that the cost rate of other goods and the rate of energy service consumption
can vary with respect to one another so are written as
$\rate{C}_o$ and $\rate{q}_s$
instead of $\rbempl{C}_o$ and $\rbempl{q}_s$.
To derive the equation of the line representing the original budget constraint 
in $\rate{C}_o/\rbempl{C}_o$ vs.\ $\rate{q}_s/\rbempl{q}_s$ space
(the \circcirc{} line through the $\circ$ point
in preference path graphs), 
we solve for $\rate{C}_o$ to obtain

\begin{equation}
  \rate{C}_o = - \bempl{p}_s \rate{q}_s + \rbempl{M} - \rbempl{C}_{cap} - \rbempl{C}_{\md} \; .
\end{equation}
%
Multiplying judiciously by $\rbempl{C}_o/\rbempl{C}_o$ and $\rbempl{q}_s/\rbempl{q}_s$ gives

\begin{equation}
  \frac{\rate{C}_o}{\rbempl{C}_o} \rbempl{C}_o
       = - \bempl{p}_s \frac{\rate{q}_s}{\rbempl{q}_s} \rbempl{q}_s 
         + \rbempl{M} - \rbempl{C}_{cap} - \rbempl{C}_{\md} \; .
\end{equation}
%
Dividing both sides by $\rbempl{C}_o$ yields

\begin{equation}
  \frac{\rate{C}_o}{\rbempl{C}_o}
       = - \frac{\bempl{p}_s \rbempl{q}_s}{\rbempl{C}_o} \frac{\rate{q}_s}{\rbempl{q}_s}
         + \frac{1}{\rbempl{C}_o} (\rbempl{M} - \rbempl{C}_{cap} - \rbempl{C}_{\md}) \; .
\end{equation}
%
Noting that $\frac{\rate{C}_o}{\rbempl{C}_o}$ and 
$\frac{\rate{q}_s}{\rbempl{q}_s}$ are
the abscissa and ordinate, respectively,
on a preferences path graph gives

\begin{equation} \label{eq:orig_line_prefs}
  y = - \frac{\bempl{p}_s \rbempl{q}_s}{\rbempl{C}_o} x
         + \frac{1}{\rbempl{C}_o} (\rbempl{M} - \rbempl{C}_{cap} - \rbempl{C}_{\md}) \; .
\end{equation}

A similar procedure can be employed to derive the equation of 
\starstar{} line through the $*$ point
after the emplacement effect.
The budget constraint at the $*$ point is

\begin{equation}
  \rbempl{M} = \aprod{p}_s \rate{q}_s + \raempl{C}_{cap} + \raempl{C}_{\md} + \rate{C}_o + \raempl{N} \; .
\end{equation}
%
Substituting Eq.~(\ref{eq:N_dot_star_empl}) for $\raempl{N}$,
substituting Eq.~(\ref{eq:G_dot}) for $\rate{G}$,
multiplying judiciously by $\rbempl{C}_o/\rbempl{C}_o$ and $\rbempl{q}_s/\rbempl{q}_s$, 
rearranging, and noting that 
$\frac{\rate{C}_o}{\rbempl{C}_o}$ is the abscissa and 
$\frac{\rate{q}_s}{\rbempl{q}_s}$ is the ordinate gives

\begin{equation} \label{eq:star_line_prefs}
  y = - \frac{\aprod{p}_s \rbempl{q}_s}{\rbempl{C}_o} x
         + \frac{1}{\rbempl{C}_o} (\rbempl{M} - \rbempl{C}_{cap} - \rbempl{C}_{\md} - \rate{G}) \; .
\end{equation}
%
Note that the slope of Eq.~(\ref{eq:star_line_prefs}) is less negative
than the slope of Eq.~(\ref{eq:orig_line_prefs}), 
because $\aprod{p}_s < \bempl{p}_s$.
The $y$-intercept of Eq.~(\ref{eq:star_line_prefs}) is less than the 
$y$-intercept of Eq.~(\ref{eq:orig_line_prefs}),
reflecting gross savings.
Both effects are seen in
preference path graphs 
(Figs.~\ref{fig:ExamplePrefsPathGraph}, \ref{fig:CarPrefsGraph}, and \ref{fig:LampPrefsGraph}).
The \circcirc{} and \starstar{} lines intersect at the coincident $\circ$ and $*$ points.

A similar derivation process can be used to find the equation of 
line representing the budget constraint
after the substitution effect (\hathat{} line through the $\wedge$ point) to be

\begin{equation} \label{eq:hat_line_prefs}
  y = - \frac{\aprod{p}_s \rbempl{q}_s}{\rbempl{C}_o} x
         + \frac{1}{\rbempl{C}_o} (\rbempl{M} - \rbempl{C}_{cap} - \rbempl{C}_{\md} 
                                   - \rate{G} + \aprod{p}_s \Delta \rasub{q}_s + \Delta \rasub{C}_o) \; .
\end{equation}

Note that the \hathat{} line (Eq.~(\ref{eq:hat_line_prefs})) has the same slope as 
the \starstar{} line (Eq.~(\ref{eq:star_line_prefs}))
but a lower $y$-intercept.

Finally, the equation of the constant expenditure line through the 
$-$ point (\barbar{}) is given by

\begin{equation} \label{eq:bar_line_prefs}
  y = - \frac{\aprod{p}_s \rbempl{q}_s}{\rbempl{C}_o} x
        + \frac{1}{\rbempl{C}_o} (\rbempl{M} - \rbempl{C}_{cap} - \rbempl{C}_{\md} 
                                   - \Delta \raempl{C}_{cap} - \Delta \raempl{C}{\md}) \; .
\end{equation}


%------------------------------
\subsubsection{Ray from the origin to the $\wedge$ point} 
\label{sec:pref_graph_ray}
%------------------------------

On preference path graphs, 
the ray from the origin to the $\wedge$ point 
(line \rr{})
defines the path along which the income effect
(lines \hatd{} and \dbar{})
operates.
The ray from the origin to the $\wedge$ point
has slope $(\rasub{C}_o/\rbempl{C}_o) / (\rasub{q}_s/\rbempl{q}_s)$
and a $y$-intercept of 0.
Therefore, the equation of line \rr{} is

\begin{equation} \label{eq:ray_prefs}
  y = \frac{\rasub{C}_o/\rbempl{C}_o}{\rasub{q}_s/\rbempl{q}_s} \, x \; .
\end{equation}


%------------------------------
\subsubsection{Indifference curves} 
\label{sec:pref_graph_indifference_curves}
%------------------------------

On a preferences path graph, 
indifference curves represent lines of constant utility
for the energy conversion device owner.
In $\rate{C}_o/\rbempl{C}_o$ vs.\ $\rate{q}_s/\rbempl{q}_s$ space, 
any indifference curve 
is given by 
Eq.~(\ref{eq:utility_Co_form})
with $\fCs$ replacing the share parameter $a$, 
as shown in Appendix~\ref{sec:utility_model}.
Recognizing that 
$\frac{\rate{C}_o}{\rbempl{C}_o}$ is on the ordinate and 
$\frac{\rate{q}_s}{\rbempl{q}_s}$ is on the abscissa
leads to substitution of 
$y$ for $\frac{\rate{C}_o}{\rbempl{C}_o}$ and 
$x$ for $\frac{\rate{q}_s}{\rbempl{q}_s}$ to obtain

\begin{equation} \label{eq:utility_line_form}
  y = \left[ \frac{1}{1 - \fCs} \left( \frac{\rate{u}}{\rbempl{u}} \right)^\rho 
            - \frac{\fCs}{1 - \fCs} (x)^\rho \right]^{(1/\rho)} \; .
\end{equation}

At any point in 
$\rate{C}_o/\rbempl{C}_o$ vs.\ $\rate{q}_s/\rbempl{q}_s$ space,
namely ($\rate{q}_{s,1}/\rbempl{q}_s$, $\rate{C}_{o,1}/\rbempl{C}_o$),
indexed utility ($\rate{u}_1/\rbempl{u}$) is given by Eq.~(\ref{eq:ces_utility}) as

\begin{equation} \label{eq:utility_ratio_1_point}
  \frac{\rate{u}_1}{\rbempl{u}} =
        \left[ \fCs \left( \frac{\rate{q}_{s,1}}{\rbempl{q}_s} \right)^\rho
        + (1-\fCs) \left( \frac{\rate{C}_{o,1}}{\rbempl{C}_o} \right)^\rho  \right]^{(1/\rho)} \; .
\end{equation}
%
Substituting Eq.~\ref{eq:utility_ratio_1_point} into Eq.~\ref{eq:utility_line_form} 
and simplifying exponents gives

\begin{equation}
  y = \left\{ \frac{1}{1 - \fCs} \left[ \fCs \left( \frac{\rate{q}_s}{\rbempl{q}_s} \right)^\rho 
        + (1-\fCs) \left( \frac{\rate{C}_o}{\rbempl{C}_o} \right)^\rho   \right] 
            - \frac{\fCs}{1 - \fCs} (x)^\rho \right\}^{(1/\rho)}  .
\end{equation}
%
Simplifying further yields
the equation of an indifference curve passing through point 
($\rate{q}_{s,1}/\rbempl{q}_s$, $\rate{C}_{o,1}/\rbempl{C}_o$):

\begin{equation} \label{eq:indiff_curve_eqn}
  y = \left\{ \left( \frac{\fCs}{1 - \fCs} \right) \left[ \left( \frac{\rate{q}_{s,1}}{\rbempl{q}_s} \right)^\rho 
                                                          - (x)^\rho  \right]
        + \left( \frac{\rate{C}_{o,1}}{\rbempl{C}_{o}} \right)^\rho \right\}^{(1/\rho)} \; .
\end{equation}
%
Note that if $x$ is $\rate{q}_{s,1}/\rbempl{q}_s$,
$y$ becomes $\rate{C}_{o,1}/\rbempl{C}_o$,
as expected.
