% The next command tells RStudio to do "Compile PDF" on HSB.Rnw,
% instead of this file, thereby eliminating the need to switch back to HSB.Rnw 
% before building the paper.
%!TEX root = ../HSB.Rnw

This appendix shows the mathematical details of rebound path graphs
for energy, expenditures, and preferences.
Rebound path graphs show clearly the impact of direct and indirect rebound effects.
Notional graphs can be found in 
Figs.~\ref{fig:ExampleEnergyPathGraph}--\ref{fig:ExamplePrefsPathGraph}.
Rebound path graphs for the car example can be found in 
Figs.~\ref{fig:CarEnergyGraph}--\ref{fig:CarPrefsGraph}.
Graphs for the lamp example can be found in
Figs.~\ref{fig:LampEnergyGraph}--\ref{fig:LampPrefsGraph}.

To construct rebound path graphs, equations of several lines and curves 
must be determined, including
%
\begin{itemize}

  \item lines of constant total energy consumption
        in energy path graphs, 
        
  \item 100\% and 0\% rebound lines in energy path graphs, 
  
  \item constant expenditure lines in expenditure path graphs,
  
  \item indifference curves in preference path graphs, and 
  
  \item constant expensiture lines in preference path graphs.

\end{itemize}
%
This appendix shows the derivation of equations for the above lines and curves.

%++++++++++++++++++++++++++++++
\subsection{Energy path graphs}
\label{sec:energy_path_graph_details}
%++++++++++++++++++++++++++++++

Energy path graphs show direct ($x$-axis) and indirect ($y$-axis)
energy consumption associated with the energy conversion device 
and the device owner.
Lines of constant total energy consumption provide the total rebound scale.
E.g., the 0\% and 100\% rebound lines are constant total energy consumption
lines which pass through the original point ($\circ$) and
the post-direct-emplacement-effect point ($a$) in an energy path graph.

The equation of a constant total energy consumption line is derived from 

\begin{equation}
  \rate{E}_{tot} = \rate{E}_{dir} + \rate{E}_{indir}
\end{equation}
%
at any rebound stage. (See Fig.~\ref{fig:flowchart}.)
Direct energy consumption is energy consumed by the energy conversion device
($\rate{E}_s$), and 
indirect energy consumption is the sum of embodied energy, 
energy associated with maintenenace and disposal, and energy associated 
with expenditures on other goods
($\rate{E}_{emb} + (\rate{C}_{md} + \rate{C}_o) I_E$).
Rearranging gives

\begin{equation}
  \rate{E}_{indir} = - \rate{E}_{dir} + \rate{E}_{tot} \; .
\end{equation}
%
For the energy path graph, 
indirect energy consumption is placed on the $y$-axis
and direct energy consumption is place on the $x$-axis, 
so we can write the euqation of any 
constant total energy consumption line on the energy path graph as

\begin{equation}
  y = -x + \rate{E}_s + \rate{E}_{emb} + (\rate{C}_{md} + \rate{C}_o) I_E \; ,
\end{equation}
%
where $\rate{E}_s$, $\rate{E}_{emb}$, $\rate{C}_{md}$, and $\rate{C}_o$
can be at any rebound stage in Fig.~\ref{fig:flowchart}.

The constant total energy consumption line 
that passes through the original point ($\circ$)
shows 100\% rebound:

\begin{equation}
  y = -x + \rbempl{E}_s + \rbempl{E}_{emb} + (\rbempl{C}_{md} + \rbempl{C}_o) I_E \; .
\end{equation}

The constant total energy consumption line 
that accounts for expected energy savings ($\Sdot$) only 
is the 0\% rebound line:

\begin{equation}
  y = -x + (\rbempl{E}_s - \Sdot)
          + \rbempl{E}_{emb} + (\rbempl{C}_{md} + \rbempl{C}_o) I_E \; .
\end{equation}
%
The above line passes through the $a$ point on an energy path graph.


%++++++++++++++++++++++++++++++
\subsection{Expenditure path graphs}
\label{sec:expenditure_path_graph_details}
%++++++++++++++++++++++++++++++

Expenditure path graphs show direct ($x$-axis) and indirect ($y$-axis)
expenses associated with the energy conversion device 
and the device owner.
Lines of constant expenditure are important, 
because they provide budget constraints for the device owner.

The equation of a constant total expenditures line is derived from 

\begin{equation}
  \rate{C}_{tot} = \rate{C}_{dir} + \rate{C}_{indir}
\end{equation}
%
at any rebound stage.
Direct expenditure is the cost of energy consumed by the energy conversion device
($\rate{C}_s = p_E \rate{E}_s$), and 
indirect expenses are the sum of capital costs, 
maintenenace and disposal costs, and 
expenditures on other goods
($\rate{C}_{cap} + \rate{C}_{md} + \rate{C}_o$).
Rearranging gives

\begin{equation}
  \rate{C}_{indir} = - \rate{C}_{dir} + \rate{C}_{tot} \; .
\end{equation}

For the expenditure path graph,
indirect expenditures are placed on the $y$-axis
and direct expenditures on energy for the energy conversion device are place on the $x$-axis,
so we can write the euqation of any
constant expenditure line on the energy path graph as

\begin{equation}
  y = -x + \rate{C}_s + \rate{C}_{cap} + \rate{C}_{md} + \rate{C}_o \; ,
\end{equation}
%
where $\rate{C}_s$, $\rate{C}_{cap}$, $\rate{C}_{md}$, and $\rate{C}_o$
can be at any rebound stage in Fig.~\ref{fig:flowchart}.

The constant total expenditure line 
that passes through the original point ($\circ$)
shows the budget constraint for the device owner:

\begin{equation}
  y = -x + \rbempl{C}_s + \rbempl{C}_{cap} + \rbempl{C}_{md} + \rbempl{C}_o \; ,
\end{equation}
%
or

\begin{equation}
  y = -x + \rbempl{M} \; .
\end{equation}

The constant total expenditure line 
that accounts for expected energy savings ($\Sdot$) 
and gross costs savings ($\rate{G} = p_E \Sdot$) only 
is given by:

\begin{equation}
  y = -x + (\rbempl{C}_s - \rate{G}) + \rbempl{C}_{cap} + \rbempl{C}_{md} + \rbempl{C}_o \; ,
\end{equation}
%
or

\begin{equation}
  y = -x + \rbempl{M} - \rate{G}\; .
\end{equation}
%
The above line passes through the $a$ point on an expenditure path graph.


%++++++++++++++++++++++++++++++
\subsection{Preference path graphs}
\label{sec:prefs_path_graph_details}
%++++++++++++++++++++++++++++++

Preferences path graphs show expenditures in 
$\rate{C}_o/\rbempl{C}_o$ vs.\ $\rate{q}_s/\rbempl{q}_s$ space.
Line of constant expenditure are derived from budget constraints.

Before the EEU, the budget constraint is

\begin{equation}
  \rbempl{M} = \bempl{p}_s \rate{q}_s + \rbempl{C}_{cap} + \rbempl{C}_{md} + \rate{C}_o \; .
\end{equation}
%
Note that both $\rbempl{C}_o$ and $\rbempl{q}_s$ 
can vary with respect to one another so are written as
$\rate{C}_o$ and $\rate{q}_s$.
To derive the equation of the line representing the original budget constraint 
in $\rate{C}_o/\rbempl{C}_o$ vs.\ $\rate{q}_s/\rbempl{q}_s$ space
(the \circcirc{} line through the $\circ$ point
in preference path graphs), 
we solve for $\rate{C}_o$ to obtain

\begin{equation}
  \rate{C}_o = - \bempl{p}_s \rate{q}_s + \rbempl{M} - \rbempl{C}_{cap} - \rbempl{C}_{md} \; .
\end{equation}
%
Multiplying judiciously by $\rbempl{C}_o/\rbempl{C}_o$ and $\rbempl{q}_s/\rbempl{q}_s$ gives

\begin{equation}
  \frac{\rate{C}_o}{\rbempl{C}_o} \rbempl{C}_o
       = - \bempl{p}_s \frac{\rate{q}_s}{\rbempl{q}_s} \rbempl{q}_s 
         + \rbempl{M} - \rbempl{C}_{cap} - \rbempl{C}_{md} \; .
\end{equation}
%
Dividing both sides by $\rbempl{C}_o$ yields

\begin{equation}
  \frac{\rate{C}_o}{\rbempl{C}_o}
       = - \frac{\bempl{p}_s \rbempl{q}_s}{\rbempl{C}_o} \frac{\rate{q}_s}{\rbempl{q}_s}
         + \frac{1}{\rbempl{C}_o} (\rbempl{M} - \rbempl{C}_{cap} - \rbempl{C}_{md}) \; .
\end{equation}
%
Noting that $\frac{\rate{C}_o}{\rbempl{C}_o}$ is the $x$-axis value and 
$\frac{\rate{q}_s}{\rbempl{q}_s}$ is the $y$-axis value gives

\begin{equation} \label{eq:orig_line_prefs}
  y = - \frac{\bempl{p}_s \rbempl{q}_s}{\rbempl{C}_o} x
         + \frac{1}{\rbempl{C}_o} (\rbempl{M} - \rbempl{C}_{cap} - \rbempl{C}_{md}) \; .
\end{equation}

A similar procedure can be employed to derive the equation of 
\starstar{} line through the $*$ point
after the emplacement effect.
The budget constraint at the $*$ point is

\begin{equation}
  \rbempl{M} = \aprod{p}_s \rate{q}_s + \raempl{C}_{cap} + \raempl{C}_{md} + \rate{C}_o + \raempl{N} \; .
\end{equation}
%
Substituting Eq.~(\ref{eq:N_dot_star_empl}) for $\raempl{N}$,
substituting Eq.~(\ref{eq:G_dot}) for $\rate{G}$,
multiplying judiciously by $\rbempl{C}_o/\rbempl{C}_o$ and $\rbempl{q}_s/\rbempl{q}_s$, 
rearranging, and noting that 
$\frac{\rate{C}_o}{\rbempl{C}_o}$ is the $x$-axis value and 
$\frac{\rate{q}_s}{\rbempl{q}_s}$ is the $y$-axis value gives

\begin{equation} \label{eq:star_line_prefs}
  y = - \frac{\aprod{p}_s \rbempl{q}_s}{\rbempl{C}_o} x
         + \frac{1}{\rbempl{C}_o} (\rbempl{M} - \rbempl{C}_{cap} - \rbempl{C}_{md} - \rate{G}) \; .
\end{equation}
%
Note that the slope of Eq.~(\ref{eq:star_line_prefs}) is a less-negative value
than the slope of Eq.~(\ref{eq:orig_line_prefs}), 
because $\aprod{p}_s < \bempl{p}_s$.
The $y$-intercept of Eq.~(\ref{eq:star_line_prefs}) is less than the 
$y$-intercept of Eq.~(\ref{eq:orig_line_prefs}),
reflecting gross savings.
Both effects are seen in
preference path graphs 
(Figs.~\ref{fig:ExamplePrefsPathGraph}, \ref{fig:CarPrefsGraph}, and \ref{fig:LampPrefsGraph}).
The \circcirc{} and \starstar{} lines intersect at the coincident $\circ$ and $*$ points.

A similar derivation process can be used to find the equation of 
line representing the budget constraint
after the substitution effect (\hathat{} line through the $\wedge$ point) as

\begin{equation} \label{eq:hat_line_prefs}
  y = - \frac{\aprod{p}_s \rbempl{q}_s}{\rbempl{C}_o} x
         + \frac{1}{\rbempl{C}_o} (\rbempl{M} - \rbempl{C}_{cap} - \rbempl{C}_{md} 
                                   - \rate{G} + \aprod{p}_s \Delta \rasub{q}_s + \Delta \rasub{C}_o) \; .
\end{equation}

Note that the \hathat{} line (Eq.~(\ref{eq:hat_line_prefs})) has the same slope as 
the \starstar{} line (Eq.~(\ref{eq:star_line_prefs}))
but a lower $y$-intercept.

Finally, the equation of the **** barbar **** line through the **** bar **** point is

**** ended here ****