% The next command tells RStudio to do "Compile PDF" on HSB.Rnw,
% instead of this file, thereby eliminating the need to switch back to HSB.Rnw 
% before building the paper.
%!TEX root = ../HSB.Rnw

This appendix shows the mathematical details of rebound path graphs
for energy, expenditures, and preferences.
Rebound path graphs show clearly the impact of direct and indirect rebound effects.
Notional graphs can be found in 
Figs.~\ref{fig:ExampleEnergyPathGraph}--\ref{fig:ExamplePrefsPathGraph}.
Rebound path graphs for the car example can be found in 
Figs.~\ref{fig:CarEnergyGraph}--\ref{fig:CarPrefsGraph}.
Graphs for the lamp example can be found in
Figs.~\ref{fig:LampEnergyGraph}--\ref{fig:LampPrefsGraph}.

To construct rebound path graphs, equations of several lines and curves 
must be determined, including
%
\begin{itemize}

  \item 0\% and 100\% rebound lines in energy path graphs, 
  
  \item lines of constant total energy consumption
        in energy path graphs, 
        
  \item constant expenditure lines in expenditure path graphs,
  
  \item indifference curves in preference path graphs, and 
  
  \item constant expensiture lines in preference path graphs.

\end{itemize}
%
This appendix shows the derivation of equations for the above lines and curves.

%++++++++++++++++++++++++++++++
\subsection{Energy path graphs}
\label{sec:energy_path_graph_details}
%++++++++++++++++++++++++++++++

Energy path graphs show direct ($x$-axis) and indirect ($y$-axis)
energy consumption associated with the energy conversion device 
and the device owner.
Lines of constant total energy consumption provide the total rebound scale.
E.g., the 0\% and 100\% rebound lines are constant total energy consumption
lines which pass through the original point ($\circ$) and
the post-direct-emplacement-effect point ($a$) in an energy path graph.

The equation of a constant total energy consumption line is derived from 

\begin{equation}
  \rate{E}_{tot} = \rate{E}_{dir} + \rate{E}_{indir}
\end{equation}
%
at any rebound stage. (See Fig.~\ref{fig:flowchart}.)
Direct energy consumption is energy consumed by the energy conversion device
($\rate{E}_s$), and 
indirect energy consumption is the sum of embodied energy, 
energy associated with maintenenace and disposal, and energy associated 
with expenditures on other goods
($\rate{E}_{emb} + (\rate{C}_{md} + \rate{C}_o) I_E$).
Rearranging gives

\begin{equation}
  \rate{E}_{indir} = - \rate{E}_{dir} + \rate{E}_{tot} \; .
\end{equation}
%
For the energy path graph, 
indirect energy consumption is placed on the $y$-axis
and direct energy consumption is place on the $x$-axis, 
so we can write the euqation of any 
constant total energy consumption line on the energy path graph to be

\begin{equation}
  y = -x + \rate{E}_s + \rate{E}_{emb} + (\rate{C}_{md} + \rate{C}_o) I_E \; ,
\end{equation}
%
where $\rate{E}_s$, $\rate{E}_{emb}$, $\rate{C}_{md}$, and $\rate{C}_o$
can be at any rebound stage in Fig.~\ref{fig:flowchart}.

The constant total energy consumption line 
that passes through the original point ($\circ$)
shows 100\% rebound:

\begin{equation}
  y = -x + \rbempl{E}_s + \rbempl{E}_{emb} + (\rbempl{C}_{md} + \rbempl{C}_o) I_E \; .
\end{equation}

The constant total energy consumption line 
that accounts for no other changes 
except for expected energy savings ($\Sdot$)
is the 0\% rebound line:

\begin{equation}
  y = -x + (\rbempl{E}_s - \Sdot)
          + \rbempl{E}_{emb} + (\rbempl{C}_{md} + \rbempl{C}_o) I_E \; .
\end{equation}
%
The above line passes through the $a$ point on an energy path graph.


%++++++++++++++++++++++++++++++
\subsection{Expenditure path graphs}
\label{sec:expenditure_path_graph_details}
%++++++++++++++++++++++++++++++

Expenditure path graphs show direct ($x$-axis) and indirect ($y$-axis)
expenses associated with the energy conversion device 
and the device owner.
Lines of constant expenditure are important, 
because they provide budget constraints for the device owner.



