% The next command tells RStudio to do "Compile PDF" on HSB.Rnw,
% instead of this file, thereby eliminating the need to switch back to HSB.Rnw 
% before building the paper.
%!TEX root = ../HSB.Rnw

This appendix derives expressions for substitution effect rebound.
Two terms comprise substitution effect rebound,
direct substitution rebound ($Re_{dsub}$) and
indirect substitution rebound ($Re_{isub}$).
Both the 
rate of energy service consumption ($\rasub{q}_s$) and
the rate of other goods consumption ($\rasub{C}_o$) 
must be determined after the substitution effect.
We show both an approximate method and an exact method for determining the 
post-substitution point ($\rasub{q}_s$ and $\rasub{C}_o$).

The EEU's energy efficiency increase
($\aprod{\eta} > \bempl{\eta}$)
causes the price of the energy service provided by the device to fall
($\aprod{p}_s < \bempl{p}_s$).
The substitution effect quantifies the amount by which
the device owner, in response,
increases the consumption rate of the energy service ($\rasub{q}_s > \rbsub{q}_s$) and
decreases the consumption rate of other goods ($\rasub{q}_o < \rbsub{q}_o$).

The increase in consumption of the energy service after its price drop
substitutes for consumption of other goods in the economy,
subject to a utility constraint.
The reduction in spending on other goods in the economy
is captured by indirect substitution rebound~($Re_{isub}$).


%++++++++++++++++++++++++++++++
\subsection{Approximate method}
\label{sec:approx_method_sub}
%++++++++++++++++++++++++++++++

This section derives each in terms of 
energy service efficiencies ($\bempl{\eta}$ and $\aprod{\eta}$) and
the service price elasticities 
of energy service consumption ($\eqsps$) and
other goods consumption ($\eqops$).
In the approximate method, 
the service price elasticities ($\eqsps$ and $\eqops$)
are (incorrectly) assumed constant along an indifference curve.

We begin with derivation of two ratios that are helpful later,
$\frac{\rasub{q}_s}{\rbsub{q}_s}$ and
$\frac{\rasub{q}_o}{\rbsub{q}_o}$.


%..............................
\paragraph{Expressions for two ratios, $\frac{\rasub{q}_s}{\rbsub{q}_s}$ and $\frac{\rasub{q}_o}{\rbsub{q}_o}$}
\label{sec:two_ratios}
%..............................

(See Appendix~\ref{sec:elasticities} for information about elasticities.)

In the approximate method, 
the relationship between energy service price and energy service consumption rate
is given by the service price elasticity of energy service consumption~($\eqsps$),
such that

\begin{equation}
  \frac{\rasub{q}_s}{\rbsub{q}_s} = \left( \frac{\aprod{p}_s}{\bempl{p}_s} \right)^{\eqsps} \; .
\end{equation}
%
Note that the energy service price elasticity of energy service consumption ($\eqsps$)
is assumed constant along an indifference curve
in the approximate method. 
A negative value for the service price elasticity of energy service consumption
is expected ($\eqsps < 0$),
such that when the energy service price decreases ($\aprod{p}_s < \bempl{p}_s$),
the rate of energy service consumption increases ($\rasub{q}_s > \rbsub{q}_s$).

Substituting Eq.~(\ref{eq:ps_pE_eta}) in the form
$\bempl{p}_s = \frac{\bempl{p}_E}{\bempl{\eta}}$ and
$\aprod{p}_s = \frac{\bempl{p}_E}{\aprod{\eta}}$
gives

\begin{equation}
  \frac{\rasub{q}_s}{\rbsub{q}_s} = \left( \frac{\aprod{\eta}}{\bempl{\eta}} \right)^{-\eqsps} \; .
                                                                        \tag{\ref{eq:q_ratio_func_of_eps}}
\end{equation}
%
Again, note that the energy service price elasticity of energy service consumption
is negative ($\eqsps < 0$), so that
as energy service efficiency increases ($\aprod{\eta} > \bempl{\eta}$),
the energy service consumption rate increases ($\rasub{q}_s > \rbsub{q}_s$).

To quantify the substitution effect on other purchases,
we introduce a second elasticity, 
the service price elasticity of other goods consumption~($\eqops$).
See Appendix~\ref{sec:elasticities} for constraints on the relationship between the
energy service price elasticity of energy service consumption~($\eqsps$) and the
energy service price elasticity of other goods consumption~($\eqops$).

The ratio of other goods consumption on before and after the substitution effect is given by

\begin{equation}
  \frac{\rasub{q}_o}{\rbsub{q}_o} = \left( \frac{\aprod{p}_s}{\bempl{p}_s} \right) ^ {\eqops} \; . 
\end{equation}
%
Note that the service price elasticity of other good consumption
is expected to be positive ($\eqops > 0$).
As the energy service price decreases~($\aprod{p}_s < \bempl{p_s}$), 
consumption of other goods is expected to decrease ($\rasub{q}_o < \rbsub{q}_o$), 
because constant utility is assumed across the substitution effect.
(See Appendix~\ref{sec:elasticities}.)
Substituting Eq.~(\ref{eq:ps_pE_eta}) in the form
$\bempl{p}_s = \frac{\bempl{p}_E}{\bempl{\eta}}$ and
$\aprod{p}_s = \frac{\bempl{p}_E}{\aprod{\eta}}$
gives

\begin{equation}
  \frac{\rasub{q}_o}{\rbsub{q}_o} = \left( \frac{\aprod{\eta}}{\bempl{\eta}} \right) ^ {-\eqops} \; . 
                                                                                              \tag{\ref{eq:qohat_qostar}}
\end{equation}

Next, we derive an expression for direct substitution rebound ($Re_{dsub}$).


%..............................
\paragraph{Expression for $Re_{dsub}$}
\label{sec:Re_dsub}
%..............................

As shown in Table~\ref{tab:subeffect}, direct substitution rebound is defined as

\begin{equation}
  Re_{dsub} \equiv \frac{\Delta \rasub{E}_s}{\Sdot} \; . \tag{\ref{eq:Re_dsub_def}}
\end{equation}
%
Expansion of the difference term and
substitution of Eq.~(\ref{eq:typ_qs_eta_Edot}) in the form
$\rasub{E}_s = \frac{\rasub{q}_s}{\aprod{\eta}}$
and
$\rbsub{E}_s = \frac{\rbsub{q}_s}{\aprod{\eta}}$
gives

\begin{equation}
   Re_{dsub} = \frac{\rasub{E}_s - \rbsub{E}_s}{\Sdot} \; ,
\end{equation}
%
and

\begin{equation}
     Re_{dsub} = \frac{\frac{\rasub{q}_s}{\aprod{\eta}} - \frac{\rbsub{q}_s}{\aprod{\eta}}}{\Sdot} \; .
\end{equation}
%
Substitution of Eq.~(\ref{eq:Sdot}) gives

\begin{equation}
  Re_{dsub} = \frac{\frac{\rasub{q}_s}{\aprod{\eta}} - \frac{\rbsub{q}_s}{\aprod{\eta}}}
              {\Sdoteqn} \; .
\end{equation}
%
Rearrangement of the numerator and canceling terms gives

\begin{equation}
  Re_{dsub} = \frac{\left( \frac{\rasub{q}_s}{\rbsub{q}_s} - 1 \right) \frac{\rbsub{q}_s}{\cancel{\aprod{\eta}}} }
              {\left( \frac{\aprod{\eta}}{\bempl{\eta}} - 1 \right)\!\frac{\bempl{\eta}}{\cancel{\aprod{\eta}}} \rbempl{E}_s} \; ,
\end{equation}
%
and

\begin{equation}
    Re_{dsub} = \frac{\frac{\rasub{q}_s}{\rbsub{q}_s} - 1}{\frac{\aprod{\eta}}{\bempl{\eta}} - 1} \; \;
                \frac{\cancelto{\rbempl{E}_s}{\frac{\rbsub{q}_s}{\bempl{\eta}}}}{\rbempl{E}_s} \; .
\end{equation}
%
Noting that $\frac{\rbsub{q}_s}{\bempl{\eta}} = \frac{\rbempl{q}_s}{\bempl{\eta}} = \rbempl{E}_s$,
canceling $\rbempl{E}_s$ terms,
and substituting Eq.~(\ref{eq:q_ratio_func_of_eps}) gives

\begin{equation}
  Re_{dsub} = \Redsubeqn \; . \tag{\ref{eq:Re_dsub}}
\end{equation}

Note that the service price elasticity of energy service consumption is
expected to be negative ($\eqsps < 0$).
For example, when $\eqsps = -0.2$ and $\frac{\aprod{\eta}}{\bempl{\eta}} = 2$,
$Re_{dsub} = 0.15$.

% To find the values for $Re_{dsub}$ in the limits where $\etaratioinline{} \to 1$ or 
% $\etaratioinline{} \to \infty$, we take derivatives of numerator and denominator
% of Eq.~(\ref{eq:Re_dsub}) with respect to $\etaratioinline{}$ and invoke L'H\^{o}pital's rule.
% 
% \begin{equation}
%   \frac{\dbydetaeta{}\left[ \left( \etaratiostacked{} - 1 \right)^{-\eqsps}  \right]}
%            {\dbydetaeta{}\left[ \etaratiostacked{} - 1\right]}
%     = -\eqsps \left( \etaratiostacked{} \right)^{-(\eqsps + 1)}
% \end{equation}
% %
% Substituting 1 and $\infty$ for $\etaratioinline$ gives
% 
% \begin{equation}
%   \lim_{\aprod{\eta}/\bempl{\eta} \to 1^{\!+}} Re_{dsub} 
%     = -\eqsps ( 1 )^{-(\eqsps + 1)}
%     = -\eqsps \; ,                               \tag{\ref{eq:lim_Redsub_1}}
% \end{equation}
% %
% and
% 
% \begin{equation}
%   \lim_{\aprod{\eta}/\bempl{\eta} \to \infty} Re_{dsub} 
%     = -\eqsps ( \infty )^{-(\eqsps + 1)}
%     = 0 \; ,                                              \tag{\ref{eq:lim_Redsub_infty}}
% \end{equation}
% %
% because $\eqsps \in [-1, 0)$ and, therefore, $(\eqsps + 1) > 0$.

With $\eqsps \in (-1, 0)$ expected,
the approximate method indicates that
direct substitution rebound will never be larger than 1.
The direct substitution effect alone
can never cause backfire. 


%..............................
\paragraph{Expression for $Re_{isub}$}
\label{sec:Re_isub}
%..............................

To derive an expression for indirect substitution rebound
under the approximate method,
we begin with the definition of $Re_{isub}$
from Table~\ref{tab:subeffect}:

\begin{equation}
  Re_{isub} \equiv \frac{\Delta \rasub{C}_o I_E}{\Sdot} \; .  \tag{\ref{eq:Re_isub_dev}}
\end{equation}
%
Expansion of the difference term to $\rasub{C}_o - \rbsub{C}_o$ and rearranging the numerator gives

\begin{equation} \label{eq:Re_isub_prelim}
  Re_{isub} = \frac{\left( \frac{\rasub{C}_o}{\rbsub{C}_o} - 1  \right) \rbsub{C}_o I_E} {\Sdot} \; .
\end{equation}
%
We assume a basket of other goods purchased in the economy,
each ($i$) with its own price ($p_{o,i}$) and rate of consumption ($\rate{q}_{o,i}$),
such that the average price of all other goods purchased in the economy~($p_o$) is given by

\begin{equation}
  p_o = \frac{\sum\limits_i \bempl{p}_{o,i} \rbempl{q}_{o,i}}{\sum\limits_i \rbempl{q}_{o,i}} \; .
\end{equation}
%
Then, the cost rate of other purchases in the economy can be given as

\begin{equation}
  \rbsub{C}_o = \bsub{p}_o \rbsub{q}_o \; ,
\end{equation}
%
and

\begin{equation}
  \rasub{C}_o = \asub{p}_o \rasub{q}_o \; .
\end{equation}
%
Assuming that the average price is unchanged across the substitution effect,
such that $\asub{p}_o = \bsub{p}_o$,
the preceding two equations can be set equal and Eq.~(\ref{eq:qohat_qostar}) can be added to find

\begin{equation} \label{eq:Cdot_o_ratio}
  \frac{\rasub{C}_o}{\rbsub{C}_o} 
      = \frac{\rasub{q}_o}{\rbsub{q}_o} 
      = \left( \frac{\aprod{\eta}}{\bempl{\eta}} \right)^{-\eqops}  \; .
\end{equation}
%
Note that Eq.~(\ref{eq:Cdot_o_ratio}) 
(along with $\rbsub{C}_o = \rbempl{C}_o$)
can be used to determine the rate of expenditures 
on other goods in the economy~($\rasub{C}_o$) by

\begin{equation}
  \rasub{C}_o = \rbempl{C}_o \left( \frac{\aprod{\eta}}{\bempl{\eta}} \right)^{-\eqops} \; .
\end{equation}

Substituting Eq.~\ref{eq:Cdot_o_ratio} into Eq.~(\ref{eq:Re_isub_prelim}) gives

\begin{equation}
  Re_{isub} = \frac{\left[ \left( \frac{\aprod{\eta}}{\bempl{\eta}} \right)^{-\eqops} - 1  \right] \rbsub{C}_o I_E} {\Sdot} \; .
\end{equation}
%
Substituting Eq.~\ref{eq:Sdot} gives

\begin{equation}
  Re_{isub} = \frac{\left[ \left(\frac{\aprod{\eta}}{\bempl{\eta}} \right)
                  ^{-\eqops} - 1  \right] \rbsub{C}_o I_E}
                  {\Sdoteqn} \; .
\end{equation}
%
Noting that $\rbsub{C}_o = \rbempl{C}_o$ and rearranging yields

\begin{equation}
  Re_{isub} = \Reisubeqn{} \; . \tag{\ref{eq:Re_isub}}
\end{equation}

Because the service price elasticity of other goods consumption is positive ($\eqops > 0$) and
the energy service efficiency ratio is greater than 1 ($\aprod{\eta} > \bempl{\eta}$),
indirect substitution rebound will be negative always ($Re_{isub} < 0$),
as expected.
Negative rebound indicates that indirect substitution reduces the energy takeback by direct substitution.

% We can take derivatives of numerator and denominator and
% invoke L'H\^{o}pital's rule
% to find limits for $Re_{isub}$ when
% $\etaratioinline{} \to 1^{\! +}$ and when
% $\etaratioinline{} \to \infty$.
% First, rearranging Eq.~(\ref{eq:Re_isub}) gives
% 
% \begin{equation}
%   Re_{isub} = \frac{\left( \etaratiostacked{} \right)^m - \etaratiostacked{}}{\left( \etaratiostacked{} \right) - 1} \frac{\rbempl{C}_o I_E}{\rbempl{E}_s} \; ,
% \end{equation}
% %
% where $m \equiv 1 - \eqops$.
% Taking the derivatives of numerator and denominator of Eq.~(\ref{eq:Re_isub})
% with respect to $\etaratioinline{}$ gives
% 
% \begin{equation}
%   \frac{\dbydetaeta{} \left[ \left( \etaratiostacked{} \right)^m - \etaratiostacked \right]}
%             {\dbydetaeta{}\left[ \left( \etaratiostacked{} \right)  - 1 \right] } \frac{\rbempl{C}_o I_E}{\rbempl{E}_s} 
%        = \left[ m \left( \etaratiostacked{} \right)^{m-1} - 1 \right] \frac{\rbempl{C}_o I_E}{\rbempl{E}_s} \; .
% \end{equation}
% %
% Substituting $\etaratioinline = 1$ and invoking L'H\^{o}pital's rule gives
% 
% \begin{equation}
%   \lim_{\etaratioinline \to 1^{+}} Re_{isub} 
%         = \left[ m ( 1 )^{m-1} - 1 \right] \frac{\rbempl{C}_o I_E}{\rbempl{E}_s}
%         = [m - 1] \frac{\rbempl{C}_o I_E}{\rbempl{E}_s} \; .
% \end{equation}
% %
% Substituting for $m$ gives
% 
% \begin{equation}
%   \lim_{\etaratioinline \to 1^{+}} Re_{isub} 
%         = -\eqops \frac{\rbempl{C}_o I_E}{\rbempl{E}_s} \; . \tag{\ref{eq:lim_Reisub_1}}
% \end{equation}
% 
% Similarly, 
% substituting $\etaratioinline = \infty$ and invoking L'H\^{o}pital's rule gives
% 
% \begin{equation}
%   \lim_{\etaratioinline \to \infty} Re_{isub} 
%         = \left[ m ( \infty )^{m-1} - 1 \right] \frac{\rbempl{C}_o I_E}{\rbempl{E}_s} \; .
% \end{equation}
% %
% Because $\eqops \in (0, 1)$ is expected **** Check with Gregor.  Is this true? ****,
% $0 < m < 1$ is expected, $m - 1 < 0$ is expected, and
% 
% \begin{equation}
%   \lim_{\aprod{\eta}/\bempl{\eta} \to \infty} Re_{isub} = -\frac{\rbempl{C}_0 I_E}{\rbempl{E}_s} \; .
% \end{equation}


%++++++++++++++++++++++++++++++
\subsection{Exact method}
\label{sec:exact_method_sub}
%++++++++++++++++++++++++++++++

The approximate method assumes that 
the energy service price elasticity of energy service consumption ($\eqsps$) and
the energy service price elasticity of other goods consumption ($\eqops$)
are constant along an indifference curve.
This assumption is an approximation that holds 
for infinitesimally small energy service price changes 
($\Delta \aempl{p}_s \approx 0$).
However, in the case of an energy efficiency upgrade (EEU), 
the energy service price change is not infinitesimal.
Rather, 
$\Delta \aempl{p}_s \approx 0$ is finite and may be large.

To determine the new consumption bundle after the substitution effect and,
ultimately, to quantify the direct and indirect rebound effects
($Re_{dsub}$ and $Re_{isub}$),
we remove the restriction that energy service price elasticities ($\eqsps$ and $\eqops$)
must be constant along an indifference curve.
Instead, recognizing that we employ a constant elasticity of substitution (CES)
utility function,
we require constancy of only
the elasticity of substitution ($\sigma$) between
the consumption rate of the energy service ($\rate{q}_s$)
and the consumption rate of other goods ($\rate{C}_o$).







