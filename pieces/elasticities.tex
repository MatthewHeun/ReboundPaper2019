% The next command tells RStudio to do "Compile PDF" on HSB.Rnw,
% instead of this file, thereby eliminating the need to switch back to HSB.Rnw 
% before building the paper.
%!TEX root = ../HSB.Rnw


%**** Gregor: place information about how to determine elasticities in this file. ---MKH ****

The rebound framework employs four elasticities.
In general their value depends on consumer preferences.
A common theoretical representation is the constant elasticity
of substitution utility map, CES (Lemoine 2020 -- to be cited properly).

Here we derive the elasticities analytically for CES utility
\begin{equation}
U(\rbempl{q}_s,\rbempl{q}_o) = 
\left( a \rbempl{q}_s^{\frac{\sigma-1}{\sigma}} + (1-a) \rbempl{q}_o^{\frac{\sigma-1}{\sigma}}\right) ^{\frac{\sigma}{\sigma-1}}
\end{equation}
Elasticities explain (compensated) demand behavior as a function of varying prices or income.
We first derive uncompensasted (Marshallian) price and the income elasticities
and then use Slutsky's equation and some identities to derive the
compensated (Hicksian) price elasticities.
Since utility is maximized subject to 
\begin{equation}
\rbempl{M} = p_s \rbempl{q}_s + p_o\rbempl{q}_o 
\end{equation}
+++ check with Matt what to subtract from M --- GS +++
at any chosen consumption bundle, the marginal rate of substitution will
equal the price ratio
\begin{equation}
 \frac {\partial U / \partial \rbempl{q}_s}{\partial U / \partial \rbempl{q}_o} =
 \frac{ p_s }{ p_o}
\end{equation}
Some algebra shows this is  
\begin{equation}
 \left( \frac {a \rbempl{q}_o}{(1-a) \rbempl{q}_s} \right ) ^ {1/\sigma} =
 \frac{ p_s }{ p_o}
\end{equation}
Substituting the income constraint variously for $\rbempl{q}_o$ and solving
for $ \rbempl{q}_s$ gives its uncompensated demand as a function of 
prices and income
\begin{equation}
 \rbempl{q}_s(p_s, p_o, \rbempl{M}) = \frac{ \rbempl{M}}{ p_s +
 \left (\frac{(1-a)p_s}{a p_o}\right)^\sigma p_o }
\end{equation}
The uncompensated (UC) price elasticity, for which empirical estimates are
often available, is
\begin{equation}
  \eqspsUC = \frac {\partial \rbempl{q}_s(p_s, p_o, \rbempl{M})}{\partial p_s} \; 
  \frac{p_s}{\rbempl{q}_s} = \frac{(\sigma - 1)p_s}{p_s +
  \left (\frac{(1-a)p_s}{a p_o}\right)^\sigma p_o} - \sigma 
\end{equation}
and the income elasticity, $\eqsM$, is unitary due to homotheticity of CES function.
This information along with knowledge about the share of income spent on the
energy service is sufficient to derive the compensated price elasticity for
the substitution effect. The Slutsky equation in elasticity form is
\begin{equation}
  \eqspsUC = \eqsps - h_{\rbempl{q}_s} \eqsM
\end{equation}
where $h_{\rbempl{q}_s}$ is the share of income spent on $h_{\rbempl{q}_s}$.
The compensated elasticity is found by simply rearranging and substituting the known 
income elasticity
\begin{equation}
  \eqsps = \eqspsUC + h_{\rbempl{q}_s}
\end{equation}
It is possible to derive the compensated cross price elasticity, $\eqops$, based
on the restriction that CES imposes on behavior.
First, note that $\eqoM$ is also unitary.
Then with Allen (1934) note that
\begin{equation}
  \eqopsUC = h_{\rbempl{q}_s} ( \sigma - \eqoM) = h_{\rbempl{q}_s} ( \sigma - 1)
  \label{Allen}
\end{equation}
But $\sigma$ is constrained in the CES. As Gortz (1977) shows the elasticity 
of substition equals
\begin{equation}
  \sigma  = \frac{h_{\rbempl{q}_s} + \eqspsUC}{h_{\rbempl{q}_s} -1 }
\end{equation}
Therefore (\ref{Allen}) expands to 
\begin{equation}
  \eqopsUC =  \frac{h_{\rbempl{q}_s} (1 + \eqspsUC)}{h_{\rbempl{q}_s}-1}
\end{equation}
which allows to use the cross-price version of the Slutsky equation
\begin{equation}
  \eqopsUC = \eqops - h_{\rbempl{q}_s} \eqoM
\end{equation}
to solve for the compensated cross price elasticity
\begin{equation}
  \eqops = \frac{h_{\rbempl{q}_s} (h_{\rbempl{q}_s} + \eqspsUC)}{h_{\rbempl{q}_s}-1}
\end{equation}
where $\eqoM = 1$.