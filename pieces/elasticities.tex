% The next command tells RStudio to do "Compile PDF" on HSB.Rnw,
% instead of this file, thereby eliminating the need to switch back to HSB.Rnw 
% before building the paper.
%!TEX root = ../HSB.Rnw


The rebound framework employs four elasticities:
%
\begin{itemize}

  \item the energy service price elasticity of energy service consumption (own price elasticity) ($\eqsps$),

  \item the energy service price elasticity of other goods consumption (cross price elasticity) ($\eqops$), 
  
  \item the income elasticity of energy service consumption ($\eqsM$), and 
  
  \item the income elasticity of other goods consumption ($\eqoM$).

\end{itemize}
%
In general, their values depend on consumer preferences.
A common theoretical representation is the constant elasticity
of substitution utility map, CES (Lemoine 2020 -- to be cited properly).\footnote{Note the Cobb-Douglas special case of CES cannot be used as it is restricted to a 100\% direct rebound effect: the uncompensated own price elasticity is 1.}

\begin{equation}
  U(\rbempl{q}_s,\rbempl{q}_o) = 
      \left( a \rbempl{q}_s^{\frac{\sigma-1}{\sigma}} + (1-a) \rbempl{q}_o^{\frac{\sigma-1}{\sigma}}\right)                ^{\frac{\sigma}{\sigma-1}}
\end{equation}

where `a' is a weight and $\sigma$ is the elasticity of substitution between goods.
+++ check with Matt what to subtract from M --- GS +++
Analytical expressions for the CES elasticities can be derived using the Slutsky equation, whereby the uncompensated (UC) own price elasticity of the energy service, $\eqspsUC$, is decomposed into compensated price and income elasticity

\begin{equation}
  \eqspsUC = \eqsps - f_{\rbempl{q}_s} \eqsM \; ,
\end{equation}
%
%and the income elasticity is unitary ($\eqsM = 1$) due to homotheticity of CES function.
%This information, along with knowledge about the share of income spent on the
%energy service, is sufficient to derive the compensated price elasticity for
%the substitution effect. 

where $f_{\rbempl{q}_s}$ is the share of income spent 
on the energy service ($\rbempl{q}_s$). The compensated elasticity is found by simply rearranging and noting that, in the case of CES utility, the income elasticity is equal to one 

\begin{equation} \label{eq:eqsps}
  \eqsps = \eqspsUC + f_{\rbempl{q}_s} \;.
\end{equation}

CES restrictions on behavior also allow straightforward derivation of the cross price elasticity ($\eqops$).
With Allen (1934) note that uncompensated cross price elasticity can generally be expressed as

\begin{equation} \label{eq:Allen}
  \eqopsUC = f_{\rbempl{q}_s} ( \sigma - \eqoM) \; .
\end{equation}
%
As Gortz (1977) shows, the elasticity 
of substition ($\sigma$) for CES utiliy is given by

\begin{equation} \label{eq:sigma}
  \sigma  = \frac{f_{\rbempl{q}_s} + \eqspsUC}{f_{\rbempl{q}_s} -1 } \; .
\end{equation}
%
Substituting Eq.~(\ref{eq:sigma}) into Eq.~(\ref{eq:Allen}) yields, 

\begin{equation} \label{eq:eps_uncompensated}
  \eqopsUC = f_{\rbempl{q}_s} \left( \frac{f_{\rbempl{q}_s} + \eqspsUC}{f_{\rbempl{q}_s} -1 } - \eqoM \right) \; .
\end{equation}
%
We set Eq.~(\ref{eq:eps_uncompensated}) equal to 
the cross-price version of the Slutsky equation,

\begin{equation}
  \eqopsUC = \eqops - f_{\rbempl{q}_s} \eqoM \; ,
\end{equation}
%
 and solve for $\eqops$ 

\begin{equation} \label{eq:eqops}
  \eqops = \frac{f_{\rbempl{q}_s} (f_{\rbempl{q}_s} + \eqspsUC)}{f_{\rbempl{q}_s}-1} \; .
\end{equation}

Note finally that the income elasticity of other goods' consumption
is also unitary ($\eqoM = 1$) for CES utility. 


Thus, the four elasticities in the rebound framework for are given by
Eq.~(\ref{eq:eqsps}), 
Eq.~(\ref{eq:eqops}), 
$\eqsM = 1$, and 
$\eqoM = 1$.
Values of only two parameters are required:
%
\begin{enumerate*}[label={(\alph*)}]
	
  \item the uncompensated energy service price elasticity of energy service consumption~($\eqspsUC$) and
  
  \item and the fraction of income spent on the energy service ($f_{\rbempl{q}_s}$).
    
\end{enumerate*}

