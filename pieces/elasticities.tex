% The next command tells RStudio to do "Compile PDF" on HSB.Rnw,
% instead of this file, thereby eliminating the need to switch back to HSB.Rnw 
% before building the paper.
%!TEX root = ../HSB.Rnw


The rebound framework employs four elasticities:
%
\begin{itemize}

  \item the energy service price elasticity of energy service consumption (own price elasticity) ($\eqsps$),

  \item the energy service price elasticity of other goods consumption (cross price elasticity) ($\eqops$), 
  
  \item the income elasticity of energy service consumption ($\eqsM$), and 
  
  \item the income elasticity of other goods consumption ($\eqoM$).

\end{itemize}
%
In general, their values depend on consumer preferences.
Analytical expressions for the elasticities can be derived using the Slutsky equation, 
whereby the uncompensated (UC) own price elasticity 
of the energy service ($\eqspsUC$) 
is decomposed into compensated price and income elasticities

\begin{equation}
  \eqspsUC = \eqsps - f_{\rbempl{q}_s} \eqsM \; ,
\end{equation}
%
where $f_{\rbempl{q}_s}$ is the share of income spent 
on the energy service ($\rbempl{q}_s$). 
Solving for the compensated elasticity 

\begin{equation} \label{eq:eqsps}
  \eqsps = \eqspsUC + f_{\rbempl{q}_s} \eqsM
\end{equation}
 shows that given empirical information on the uncompensated own price elasticity and share of income spent on goods (as is available for our examples above) preferences must only impose restrictions on the income elasticity ($\eqsM$) to determine the own price elasticity.
 
A similar argument allows straightforward derivation 
of the cross price elasticity ($\eqops$).
With \citet{Hicks1934}, 
note that uncompensated cross price elasticity can generally be expressed as

\begin{equation} \label{eq:Allen}
  \eqopsUC = f_{\rbempl{q}_s} (\sigma - \eqoM) \; ,
\end{equation}
%
where $\sigma$ is the elasticity of substitution between the energy service (s) and other goods (o). 
We set Eq.~(\ref{eq:Allen}) equal to 
the cross-price version of the Slutsky equation

\begin{equation} \label{eq:slutsky_cross}
  \eqopsUC = \eqops - f_{\rbempl{q}_s} \eqoM
\end{equation}
%
and solve for $\eqops$ to find

\begin{equation} \label{eq:eqops_general}
  \eqops = f_{\rbempl{q}_s} \sigma \; .
\end{equation}
which shows that preferences must also place a restriction on the elasticity of substitution to make progress.

A common theoretical representation of preferences is
the constant elasticity of substitution (CES)
utility map \citep{Lemoine:2020aa}%
\footnote{
  Note the Cobb-Douglas special case of CES cannot be used, because
  it is restricted to 100\% direct rebound effect: 
  the uncompensated own price elasticity ($\eqspsUC$) is 1.
}

\begin{equation}
  \rate{u}(\rbempl{q}_s,\rbempl{q}_o) = 
      \left( a \rbempl{q}_s^{\frac{\sigma-1}{\sigma}} + (1-a) \rbempl{q}_o^{\frac{\sigma-1}{\sigma}}\right)                ^{\frac{\sigma}{\sigma-1}} \; ,
\end{equation}
%
where $a$ is a weight. First note that due to homotheticity,
all income elasticities are equal to one. Therefore
the compenseated own price elasticity \ref{eq:eqsps} can be 
determined with empirically available information.
For the cross price elasticity, G{\o}rtz (1977) show that
the elasticity of substitution ($\sigma$) 
for CES utiliy is given by

\begin{equation} \label{eq:sigma}
  \sigma  = \frac{f_{\rbempl{q}_s} + \eqspsUC}{f_{\rbempl{q}_s} -1 } \; .
\end{equation}
%
Substituting Eq.~(\ref{eq:sigma}) into Eq.~(\ref{eq:Allen}) yields, 

\begin{equation} \label{eq:eps_uncompensated}
  \eqopsUC = f_{\rbempl{q}_s} \left( \frac{f_{\rbempl{q}_s} + \eqspsUC}{f_{\rbempl{q}_s} -1 } - \eqoM \right) \; .
\end{equation}

Setting Eq.~(\ref{eq:eps_uncompensated}) again equal to 
the cross-price version of the Slutsky Eq.~\ref{eq:slutsky_cross}
and solving for $\eqops$ yields a result in terms only
of empirically available parameters

\begin{equation} \label{eq:eqops}
  \eqops = \frac{f_{\rbempl{q}_s} (f_{\rbempl{q}_s} + \eqspsUC)}{f_{\rbempl{q}_s}-1} \; .
\end{equation}

Thus, the four elasticities in the rebound framework are given by
Eq.~(\ref{eq:eqsps}), 
Eq.~(\ref{eq:eqops}), 
$\eqsM = 1$, and 
$\eqoM = 1$.
Values of only two parameters are required:
%
\begin{enumerate*}[label={(\alph*)}]
	
  \item the uncompensated energy service price elasticity of energy service consumption~($\eqspsUC$) and
  
  \item and the fraction of income spent on the energy service ($f_{\rbempl{q}_s}$).
    
\end{enumerate*}

