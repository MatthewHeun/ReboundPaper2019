% The next command tells RStudio to do "Compile PDF" on HSB.Rnw,
% instead of this file, thereby eliminating the need to switch back to HSB.Rnw 
% before building the paper.
%!TEX root = ../HSB.Rnw


The rebound framework employs four elasticities:
%
\begin{itemize}

  \item the energy service price elasticity of energy service consumption ($\eqsps$),

  \item the energy service price elasticity of other goods consumption ($\eqops$), 
  
  \item the income elasticity of energy service consumption ($\eqsM$), and 
  
  \item the income elasticity of other goods consumption ($\eqoM$).

\end{itemize}
%
In general, their values depend on consumer preferences.
A common theoretical representation is the constant elasticity
of substitution utility map, CES (Lemoine 2020 -- to be cited properly).

Here we derive the elasticities analytically for CES utility

\begin{equation}
  U(\rbempl{q}_s,\rbempl{q}_o) = 
      \left( a \rbempl{q}_s^{\frac{\sigma-1}{\sigma}} + (1-a) \rbempl{q}_o^{\frac{\sigma-1}{\sigma}}\right)                ^{\frac{\sigma}{\sigma-1}}
\end{equation}

Elasticities explain (compensated) demand behavior as a function of varying prices or income.
We first derive uncompensasted (Marshallian) price and the income elasticities
and then use Slutsky's equation and some identities to derive the
compensated (Hicksian) price elasticities.
Since utility is maximized subject to 

\begin{equation}
  \rbempl{M} = p_s \rbempl{q}_s + p_o\rbempl{q}_o 
\end{equation}
%
+++ check with Matt what to subtract from M --- GS +++
at any chosen consumption bundle, the marginal rate of substitution will
equal the price ratio

\begin{equation}
 \frac {\partial U / \partial \rbempl{q}_s}{\partial U / \partial \rbempl{q}_o} =
 \frac{ p_s }{ p_o}
\end{equation}

Some algebra shows this equation can be simplified to 

\begin{equation}
 \left( \frac {a \rbempl{q}_o}{(1-a) \rbempl{q}_s} \right ) ^ {1/\sigma} =
           \frac{ p_s }{ p_o} \; .
\end{equation}

Substituting the income constraint variously for $\rbempl{q}_o$ and solving
for $ \rbempl{q}_s$ gives its uncompensated demand as a function of 
prices and income:

\begin{equation}
 \rbempl{q}_s(p_s, p_o, \rbempl{M}) = \frac{ \rbempl{M}}{ p_s +
 \left (\frac{(1-a)p_s}{a p_o}\right)^\sigma p_o }
\end{equation}

The uncompensated (UC) price elasticity, for which empirical estimates are
often available, is

\begin{equation}
  \eqspsUC = \frac {\partial \rbempl{q}_s(p_s, p_o, \rbempl{M})}{\partial p_s} \; 
  \frac{p_s}{\rbempl{q}_s} = \frac{(\sigma - 1)p_s}{p_s +
  \left (\frac{(1-a)p_s}{a p_o}\right)^\sigma p_o} - \sigma 
\end{equation}
%
and the income elasticity is unitary ($\eqsM = 1$) due to homotheticity of CES function.
This information, along with knowledge about the share of income spent on the
energy service, is sufficient to derive the compensated price elasticity for
the substitution effect. 

The Slutsky equation in elasticity form is

\begin{equation}
  \eqspsUC = \eqsps - h_{\rbempl{q}_s} \eqsM \; ,
\end{equation}
%
where $h_{\rbempl{q}_s}$ is the share of income spent 
on the energy service ($\rbempl{q}_s$).
The compensated elasticity is found by simply rearranging and substituting the known 
income elasticity ($\eqsM = 1$).

\begin{equation} \label{eq:eqsps}
  \eqsps = \eqspsUC + h_{\rbempl{q}_s}
\end{equation}

It is possible to derive the compensated cross price elasticity ($\eqops$)
using the restriction that CES imposes on behavior.
First, note that 
the income elasticity of other goods consumption is also unitary ($\eqoM = 1$).
Then with Allen (1934) note that

\begin{equation} \label{eq:Allen}
  \eqopsUC = h_{\rbempl{q}_s} ( \sigma - \eqoM) = h_{\rbempl{q}_s} ( \sigma - 1) \; .
\end{equation}
%
But $\sigma$ is constrained in the CES description of utility. 
As Gortz (1977) shows, the elasticity 
of substition ($\sigma$) is given by

\begin{equation} \label{eq:sigma}
  \sigma  = \frac{h_{\rbempl{q}_s} + \eqspsUC}{h_{\rbempl{q}_s} -1 } \; .
\end{equation}
%
Substituting Eq.~(\ref{eq:sigma}) into Eq.~(\ref{eq:Allen}) yields, 
after simplification,

\begin{equation} \label{eq:eps_uncompensated}
  \eqopsUC =  \frac{h_{\rbempl{q}_s} (1 + \eqspsUC)}{h_{\rbempl{q}_s}-1} \; .
\end{equation}
%
We set Eq.~(\ref{eq:eps_uncompensated}) equal to 
the cross-price version of the Slutsky equation,

\begin{equation}
  \eqopsUC = \eqops - h_{\rbempl{q}_s} \eqoM \; ,
\end{equation}
%
solve for $\eqops$, 
substitute unitary income elasticity ($\eqoM = 1$), and 
simplify to obtain

\begin{equation} \label{eq:eqops}
  \eqops = \frac{h_{\rbempl{q}_s} (h_{\rbempl{q}_s} + \eqspsUC)}{h_{\rbempl{q}_s}-1} \; .
\end{equation}

Thus, the four elasticities in the rebound framework are given by
Eq.~(\ref{eq:eqsps}), 
Eq.~(\ref{eq:eqops}), 
$\eqsM = 1$, and 
$\eqoM = 1$.
Values of only two parameters are required:
%
\begin{enumerate*}[label={(\alph*)}]
	
  \item the uncompensated energy service price elasticity of energy service consumption~($\eqsps$) and
  
  \item and the fraction of income spent on the energy service ($h_{\rbempl{q}_s}$).
    
\end{enumerate*}


***** 
Gregor:  I think you should derive these equations WITHOUT assuming 
unitary income elasticities from CES, 
thereby making the derivation more general. 
After the derivation, you could substitute 1 in for the income elasticities 
to show the equations we will be using. --- MKH 
****
