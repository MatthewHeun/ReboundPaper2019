% The next command tells RStudio to do "Compile PDF" on HSB.Rnw,
% instead of this file, thereby eliminating the need to switch back to HSB.Rnw 
% before building the paper.
%!TEX root = ../HSB.Rnw

% This file contains the derivation for emplacement effect rebound terms and net income.

\begin{landscape}

\linespread{1}

%%%%%%%%%%%%%%%%%%%%%%%%%%%%%%%%%%%%%%%%
%%%%%%%%%% Emplacement Effect %%%%%%%%%%
%%%%%%%%%%%%%%%%%%%%%%%%%%%%%%%%%%%%%%%%
\derivheader{\refstepcounter{table} Table~\thetable \label{tab:empleffect}. \bf{\EmplEffect}}

\sectionsep{}

%%%%%%%%%% Before Emplacement Effect %%%%%%%%%%
\derivsection{before}
{
% Original energy
\begin{equation} \label{eq:E_acct_orig}
  \Eacctorig{}
\end{equation}
}
{
% Original financial
\begin{equation} \label{eq:M_acct_orig}
  \Macctorig{}
\end{equation}
}

\sectionsep{}

%%%%%%%%%% After Emplacement Effect %%%%%%%%%%
\derivsection{after ($\aempl{ }$)}
{
% After device energy
\begin{equation} \label{eq:E_acct_aemp}
  \Eacctaempl{}
\end{equation}
}
{
% After device financial
\begin{equation} \label{eq:M_acct_aemp}
  \Macctaempl{}
\end{equation}
}

\sectionsep{}

%%%%%%%%%% Derivations for Emplacement Effect %%%%%%%%%%
\derivsection{}
% Emplacement effect: energy differences
{
%%%%%%%%%% Energy Emplacement Effect %%%%%%%%%%
~
  
Take differences to obtain the change in energy consumption, $\Delta \raempl{E} \equiv \raempl{E} - \rbempl{E}$.
%
\begin{equation}
  \Delta \raempl{E} = \Delta \raempl{E}_s
                      + \Delta \raempl{E}_{emb}
                      + (\Delta \raempl{C}_{\md}
                      + \cancelto{0}{\Delta \raempl{C}_o}) I_E
\end{equation}
%
Thus, 
%
\begin{equation}
\Delta \raempl{E} = \Delta \raempl{E}_s + \Delta \raempl{E}_{emb} + \Delta \raempl{C}_{\md} I_E \; .
\end{equation}
%
Use Eq.~(\ref{eq:Re_def}) and define
%
\begin{equation} \label{eq:Sdot_def}
\Sdot \equiv -\Delta \raempl{E}_s
\end{equation}
%
to obtain
%
\begin{equation}
Re_{empl} = 1 - \frac{-\Delta \raempl{E}}{\Sdot} 
          = 1 - \frac{-\Delta \raempl{E}_s}{\Sdot} 
              - \frac{-\Delta \raempl{E}_{emb}}{\Sdot}
              - \frac{-\Delta \raempl{C}_{\md} I_E}{\Sdot} \; .
\end{equation}
%
Define $Re_{dempl} \equiv 1 - \frac{-\Delta \raempl{E}_s}{\Sdot} = 0$, 
$Re_{iempl} \equiv Re_{emb} + Re_{\md}$, 
$Re_{emb} \equiv \frac{\Delta \raempl{E}_{emb}}{\Sdot}$, and
$Re_{\md} \equiv \frac{\Delta \raempl{C}_{\md} I_E}{\Sdot}$, 
such that
%
\begin{equation} \label{eq:Re_empl_def}
Re_{empl} = Re_{dempl} + Re_{iempl} \; .
\end{equation}
}
{
%%%%%%%%%% Financial Emplacement Effect %%%%%%%%%%
~
    
Use the monetary constraint ($\rbempl{M} = \raempl{M}$)
and constant spending on other items ($\rbempl{C}_o = \raempl{C}_o$) to cancel terms to obtain
%
\begin{align}
  p_E \rbempl{E}_s &+ \rbempl{C}_{cap} + \rbempl{C}_{\md} + \cancel{\rbempl{C}_o} + \cancelto{0}{\rbempl{N}} \nonumber \\
                   &= p_E \raempl{E}_s + \raempl{C}_{cap} + \raempl{C}_{\md} + \cancel{\raempl{C}_o}  + \raempl{N} \; .
\end{align}
%
Solving for $\Delta \raempl{N} \equiv \raempl{N} - \cancelto{0}{\rbempl{N}}$ gives 
%
\begin{equation}
  \Delta \raempl{N} = p_E(\rbempl{E}_s - \raempl{E}_s) 
                      + \rbempl{C}_{cap} - \raempl{C}_{cap}
                      + \rbempl{C}_{\md} - \raempl{C}_{\md} \; .
\end{equation}
%
Rewriting with $\Delta$ terms gives
%
\begin{equation}
  \Delta \raempl{N} = - p_E \Delta \raempl{E}_s - \Delta \raempl{C}_{cap} - \Delta \raempl{C}_{\md} \; .
\end{equation}
%
Substituting Eq.~(\ref{eq:Sdot_def}) gives
%
\begin{equation}
  \Delta \raempl{N} = \raempl{N} = p_E \Sdot - \Delta \raempl{C}_{cap} - \Delta \raempl{C}_{\md} \; .
\end{equation}
%
Gross monetary savings ($\rate{G}$) resulting from the EEU, 
before any energy takeback, is given by 
%
\begin{equation} \label{eq:G_dot}
  \rate{G} = p_E \Sdot \; .
\end{equation}

Note that Eq.~(\ref{eq:M_acct_orig}) and $\rbempl{N} = 0$ can be used to calculate $\rbempl{C}_o$ as
%
\begin{equation} \label{eq:C_dot_o}
  \rbempl{C}_o = \rate{M} - p_E \rbempl{E}_s - \rbempl{C}_{cap} - \rbempl{C}_{\md} \; .
\end{equation}
%

}
\end{landscape}
